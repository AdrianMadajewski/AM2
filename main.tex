\documentclass[leqno]{article}
\usepackage{graphicx} % Required for inserting images
\usepackage[polish]{babel} % Language support for Polish
\usepackage[utf8]{inputenc} % Input encoding
\usepackage[T1]{fontenc}
\usepackage{ragged2e}
\usepackage{amssymb} % Load before mathrsfs
\usepackage{mathrsfs} % Pakiet mathrsfs dostarcza komendę \mathscr
\usepackage{hyperref} % for clickable links
\usepackage{amsmath} % Load after amssym
\usepackage{lmodern}
\usepackage{fancyhdr} % For custom headers and footers

\pagestyle{fancy}
\fancyhf{}
\renewcommand{\headrulewidth}{0pt} % Remove header line

\setlength{\parindent}{0pt} % Remove indentation globally
\setlength{\parskip}{10pt}

\setcounter{equation}{35}

\title{\Huge{Analiza Matematyczna 2}}
\author{Adrian Madajewski}
\date{Semestr II}


% Twierdzenie
\newcounter{thcounter}
\setcounter{thcounter}{73}
\newcommand{\theorem}[1]{\noindent\refstepcounter{thcounter}\textbf{Twierdzenie \thethcounter. }\textit{#1}\label{theorem:\thethcounter}}

% Lemat
\newcounter{lematcounter}
\setcounter{lematcounter}{1}
\newcommand{\lemat}[1]{\noindent\refstepcounter{lematcounter}\textbf{Lemat \thelematcounter. }\textit{#1}\label{theorem:\thelematcounter}}

% Wniosek
\newcounter{wniosekcounter}
\setcounter{wniosekcounter}{17}
\newcommand{\wniosek}[1]{\noindent\refstepcounter{wniosekcounter}\textbf{Wniosek \thewniosekcounter. }\textit{#1}\label{wniosek:\thewniosekcounter}}

% Definicja
\newcounter{defcounter}
\setcounter{defcounter}{67}
\newcommand{\defn}{\noindent\refstepcounter{defcounter}\textbf{Definicja \thedefcounter. }\label{def:\thedefcounter}}

% Uwaga
\newcounter{uwagacounter}
\setcounter{uwagacounter}{37}
\newcommand{\uwaga}{\noindent\refstepcounter{uwagacounter}\textbf{Uwaga \theuwagacounter. }\label{uwaga:\theuwagacounter}}

% Przyklad
\newcounter{excounter}
\setcounter{excounter}{24}
\newcommand{\ex}{\noindent\refstepcounter{excounter}\textbf{Przykład \theexcounter. }\label{ex:\theexcounter}}

% Dowód
\newcommand{\proof}{\noindent\textbf{Dowód. }}

% Equation
\newcounter{eqcounter}
\setcounter{eqcounter}{38}
\newcommand{\eq}{\refstepcounter{eqcounter}\label{eq:\theeqcounter}}

\begin{document}

\maketitle

\begin{center}
    Niniejszy plik jest w całości bazowany na wykładach \\
    \vspace{10pt}
    prof. dr hab. Dariusza Bugajewskiego \\
    \vspace{10pt}
    z przedmiotu \\
    \vspace{10pt}
    Analiza Matematyczna 2 \\
    \vspace{10pt}
    na Uniwersytecie im. Adama Mickiewicza w Poznaniu
    \includegraphics[width=0.8\textwidth]{uam_logo.pdf}
\end{center}
\newpage

\begin{justify}

\setcounter{section}{7}
\section{Całka Riemanna}
\subsection{Definicja i podstawowe własności całki}
\defn{}{} Niech $[a,b]$ będzie danym przedziałem. Przez podział $P$ przedziału $[a,b]$ będziemy nazywali skończony zbiór punktów \(x_0, x_1, \ldots, x_n\), gdzie
\[
a = x_0 < x_1 < \cdots < x_n = b
\]
Będziemy pisać $\Delta x_i = x_i - x_{i-1}$ $(i = 1, \ldots , n)$. Długość największego z odcinków $[x_{i-1}, x_i]$ nazywać będziemy średnicą podziału $P$ i oznaczamy ją symbolem $\delta (P)$. $\delta (P)= \max\limits_{1 \leqslant i \leqslant n} \Delta x_i$. Niech $f$ będzie ograniczoną funkcją rzeczywistą określoną na $[a,b]$. W każdym z przedziałów $[x_{i-1}, x_i]$ wybierzmy dowolny punkt $\xi_{i}$ $(i=1, \ldots n)$ i utwórzmy sumę $R = \sum_{i=1}^{n} f(\xi_{i})\Delta x_{i}$. Sumę te nazywamy sumą Riemanna odpowiadającą podziałowi $P$, przy ustalonym wyborze punktów $\xi_i$. Przez $\mathfrak{R}(f,P)$ oznaczać będziemy zbiór wszystkich możliwych sum Riemanna odpowiadających podziałowi $P$. Utwórzmy teraz ciąg $(P_k)$ podziałów przedziału $[a, b]$:
\[
a = x_0^{(k)} < x_1^{(k)} < \cdots < x_{n(k)}^{(k)} = b;
\]
\[
\Delta_i^{(k)}=x_i^{(k)}-x_{i-1}^{(k)};
\]
\[
\delta (P_k)=\max_{1 \leqslant i \leqslant n(k)} \Delta x_i^{(k)}, k = 1,2, \ldots
\]
Ciąg $(P_k)$ nazywamy ciągiem normalnym podziałów, jeśli $\delta(P_k)\rightarrow 0$  przy $k \rightarrow \infty$. Oznaczmy przez $\mathfrak{R}(f,P_k)$ zbiór wszystkich sum Riemanna odpowiadających podziałowi $P_k$.

\noindent
\defn{} Jeśli dla dowolnego ciągu normalnego podziałów $(P_k)$ i dla dowolnych sum Riemanna $R_k \in \mathfrak{R}(f,P_k)$ istnieje skończona granica $I = \lim\limits_{k \rightarrow \infty}R_k$, to tę granicę nazywamy całką Riemanna funkcji $f$ na przedziale $[a,b]$ i oznaczamy ją symbolem 
\[
\begin{aligned}
    \int_{a}^{b} fdx \text{ lub } \int_{a}^{b} f(x)dx
\end{aligned}
\]
O funkcji $f$ mówimy wówczas, że jest całkowalna w sensie Riemanna na przedziale $[a,b]$, lub że jest ona R-całkowalna na tym przedziale. 

Powyższą definicję mozna sformułować w następujący równoważny sposób.

\noindent
\defn{} Mówimy, że funkcja $f$ jest całkowalna w sensie Riemanna na przedziale $[a,b]$, jeśli istnieje liczba $I \in \mathbb{R}$ taka, że
\[
\forall_{\varepsilon>0} \exists_{\delta > 0} \forall_P \forall_{R \in \mathfrak{R}(f, P)}\delta(P) < \delta \implies |R - I| < \varepsilon
\]
Piszemy wówczas $I = \int_{a}^{b}f(x)dx = \lim\limits_{\delta(P) \rightarrow 0}R$.

Równoważność definicji $\ref{def:69}$ i $\ref{def:70}$ można pokazać analogicznie jak w dowodzie twierdzenia 35. 

\ex{} (a) Funkcja stała $f(x)=c, c \in \mathbb{R}, x \in [a,b]$ jest całkowalna w sensie Riemanna na tym przedziale. Niech $P$ będzie dowolnym podziałem przedziału $[a,b]$:
\[
a = x_0 < x_1 < \cdots < x_n = b
\]
Dowolna suma Riemanna odpowiadającą podziałowi $P$ ma postać:
\[
R = \sum_{i=1}^{n}f(\xi_i)(x_i-x_{i-1}) = \sum_{i=1}^{n}c(x_i - x_{i-1})=c(b-a),
\]
\[
    (\xi_i \in [x_{i-1},x_i], i=1, \ldots, n)
\]
Stąd wynika, że $\int_{a}^{b} f(x)dx = c(b-a)$. 

\noindent
(b) Roważmy ponownie funkcję Dirichleta z Przykładu 18 (a), zawężoną do przedziału $[a,b]$. Dla każdego podziału $P$ przedziału $[a,b]$ można utworzyć sumę Riemanna równą zeru, jeśli wszystkie punkty $\xi_i$ będą liczbami niewymiernymi, lub równą $(b-a)$, jeśli wszystkie punkty $\xi_i$ będą liczbami wymiernymi. Jest więc jasne, że dla każdego ciągu normalnego podziałów $(P_k)$ granica $\lim\limits_{k \rightarrow \infty}R_k$, gdzie $R_k \in \mathfrak{R}(f, P_k)$, $k \in \mathbb{N}$, nie istnieje. 

\defn{}\label{def:sumy_dolne_gorne} Niech $f$ będzie ograniczoną funkcją rzeczywistą określona na $[a,b]$. Każdemu podziałowi $P$ przedziału $[a,b]$ odpowiadają liczby:
\[
M_i = \sup\limits_{x_{i-1} \leqslant x \leqslant x_i} f(x) \quad m_i = \inf\limits_{x_{i-1} \leqslant x \leqslant x_i} f(x)
\]
\[
U(f, P) = \sum_{i=1}^{n}M_i\Delta x_i \quad L(f, P) = \sum_{i=1}^{n}m_i\Delta x_i
\]
Liczby $U(f,P)$ i $L(f, P)$ nazywamy odpowiednio sumą górną i dolną lub sumami Darboux funkcji $f$ przy podziale $P$ przedziału $[a,b]$. Dalej,

\begin{equation}\label{eq:36}
    \overline{\int_{a}^{b} f(x)dx} = \inf_P U(f, P),
\end{equation}

\begin{equation}\label{eq:37}
    \underline{\int_{a}^{b} f(x)dx} = \sup_P L(f, P),
\end{equation}
gdzie kres górny i dolny są brane ze względu na wszystkie podziały $P$ przedziału $[a,b]$. Lewe strony równości (\ref{eq:36}) i (\ref{eq:37}) nazywają się odpowiednio górną i dolną całką Darboux funkcji $f$ na przedziale $[a,b]$.

Ponieważ funkcja $f$ jest ograniczona, więc istnieją liczby rzeczywistę $m$ i $M$ takie, że
\[
m \leqslant f(x) \leqslant M \quad \text{dla} \quad x \in [a,b] 
\]
Oznacza to, że przy dowolnym podziale $P$ przedziału $[a,b]$ mamy
\[
m(b-a) \leqslant L(f, P) \leqslant U(f, P) \leqslant M(b-a)
\]
a zatem zbiory $\{L(f,P) : P\}$ i $\{U(f,P) : P\}$ są ograniczone. Wynika stąd, że całki górna i dolna są określone przy dowolnej funkcji ograniczonej $f$. 

\defn{} Mówimy, że podział $P^*$ przedziału $[a,b]$ jest rozdrobnieniem (lub zagęszczeniem) podziału $P$ tego przedziału, jeśli $P \subset  P^*$, to znaczy, jeśli każdy punkt przedziału $P$ jest także punktem przedziału $P^*$. Jeśli dane są dwa podziały $P_1$,$P_2$, to podział $P^*=P_1 \cup P_2$ nazywać będziemy ich wspólnym rozdrobnieniem (lub wspólnym zagęszczeniem).

\theorem{Jeśli $P^*$ jest rozdrobnieniem podziału $P$, to}
\[
L(f,P) \leqslant L(f,P^*) \quad U(f,P) \leqslant U(f,P^*)
\]
\proof{}Załóżmy wpierw, że $P^*$ zawiera tylko o jeden punkt więcej niż $P$. Niech tym dodatkowym punktem będzie $x^*$ i niech $x_{i-1} < x^* < x_i$, gdzie $x_{i-1}, x_i$ są dwoma kolejnymi punktami przedziału $P$. Przyjmijmy
\[
\omega_1 = \inf\limits_{x_{i-1} \leqslant x \leqslant x^*} f(x) \text{,} \quad \omega_2 = \inf\limits_{x^* \leqslant x \leqslant x_i} f(x) 
\]
Wtedy $\omega_1 \geqslant m_i$ i $\omega_2 \geqslant m_i$, gdzie $m_i = \inf\limits_{x_{i-1} \leqslant x \leqslant x_i} f(x)$. Mamy więc
\begin{equation*}
\begin{gathered}
    L(f,P^*)-L(f,P) = \omega_1(x^* - x_{i-1}) + \omega_2(x_i - x^*) - m_i(x_i - x_{i-1}) \\
    = (\omega_1-m_i)(x^*-x_{i-1})+(\omega_2-m_i)(x_i-x^*) \geqslant 0
\end{gathered}
\end{equation*}
Jeśli $P^*$ zawiera o $k$ punktów więcej niż $P$, to powtarzając powyższe rozumowanie $k$ razy otrzymamy pierwszą nierówność tezy. Dowód drugiej przebiega analogicznie.

\theorem{Jeśli $f$ jest funkcją ograniczoną na przedziale $[a,b]$, to}
\[
\underline{\int_{a}^{b} f(x)dx} \leqslant \overline{\int_{a}^{b} f(x)dx}
\]
\proof{}Niech $P^*$ będzie wspólnym rozdrobnieniem podziałów $P_1$ i $P_2$ przedziału $[a,b]$. Z Twierdzenia $\ref{theorem:75}$ wynika, że
\[
L(f, P_1) \leqslant L(f, P^*) \leqslant U(f, P^*) \leqslant U(f, P_2)
\]
Stąd $L(f, P_1) \leqslant U(f, P_2)$. Traktując $P_2$ jako ustalone i obliczając kres górny ze względu na wszystkie podziały $P_1$, wobec poprzedniej nierówności otrzymujemy
\[
\underline{\int_{a}^{b} f(x)dx} \leqslant U(f, P_2)
\]
Przechodząc do kresu dolnego ze względu na wszystkie podziały $P_2$ otrzymujemy tezę dowodzonego twierdzenia.

Udowodnimy teraz dwa kryteria całkowalności funkcji w sensie Riemanna. W oparciu o drugie z tych kryteriów podamy równoważną definicję całki w sensie Riemanna.

\theorem{Na to, aby ograniczona funkcja $f$ była całkowalna w sensie Riemanna na przedziale $[a,b]$ potrzeba i wystarcza, aby dla dowolnego $\varepsilon > 0$ istniał taki podział $P$ przedziału $[a,b]$, że}
\begin{equation}\label{eq:38}
U(f, P) - L(f, P) \leqslant \varepsilon
\end{equation}
\proof{}Załóżmy wpierw, że funkcja $f$ jest całkowalna w sensie Riemanna na przedziale $[a,b]$. Wówczas dla każdego danego $\varepsilon > 0$ istnieje taki podział $P$ przedziału $[a,b]$, że nierówność
\begin{equation*}
\begin{gathered}
    |R - \int_{a}^{b} f(x)dx| < \frac{\varepsilon}{2} \text{, czyli} \\
    \int_{a}^{b} f(x)dx - \frac{\varepsilon}{2} < R < \int_{a}^{b} f(x)dx + \frac{\varepsilon}{2}
\end{gathered}
\end{equation*}
jest spełniona przy dowolnym wyborze punktów $\xi_i$ w każdym z przedziałów podziału. Ponieważ sumy Darboux są --- przy danym podziale przedziału --- odpowiednio kresem górnym i dolnym sum całkowych, zatem spełniają one nierówności
\[
\int_{a}^{b}f(x)dx - \frac{\varepsilon}{2} \leqslant L(f,P) \leqslant U(f, P) \leqslant \int_{a}^{b}f(x)dx + \frac{\varepsilon}{2}
\]
a więc $U(f, P) - L(f, P) < \varepsilon$.
Załóżmy teraz, że (\ref{eq:38}) zachodzi. Dla dowolnego podziału $P$ mamy
\[
L(f, P) \leqslant \underline{\int_{a}^{b}f(x)dx} \leqslant \overline{\int_{a}^{b}f(x)dx} \leqslant U(f, P)
\]
Jeśli $U(f, P) - L(f, P) < \varepsilon$, to wówczas
\[
0 \leqslant \overline{\int_{a}^{b}f(x)dx} - \underline{\int_{a}^{b}f(x)dx} < \varepsilon
\]
Z dowolności $\varepsilon > 0$ wynika, że $\underline{\int_{a}^{b}f(x)dx} = \overline{\int_{a}^{b}f(x)dx}$. Oznaczając ponadto $\underline{\int_{a}^{b}f(x)dx} = \overline{\int_{a}^{b}f(x)dx} = I$ mamy $L(f, P) \leqslant I \leqslant U(f, P)$.
Ustalmy $\varepsilon > 0$ i niech $P$ bedzie danym podziałem przedziału $[a,b]$, dla którego (\ref{eq:38}) zachodzi. Jeśli przez $R$ oznaczymy jedną z wartości sum Riemanna odpowiadającej podziałowi $P$, to
\[
L(f,P) \leqslant R \leqslant U(f, P)
\]
Ponieważ liczby $R$ oraz $I$ znajdują się w przedziale $[L(f, P), U(f, P)]$, zatem
\[
|R - I| \leqslant \varepsilon
\]
Wobec Twierdzenia $\ref{theorem:74}$ oraz Definicji $\ref{def:70}$ wnioskujemy, że $I = \int_{a}^{b}f(x)dx$

Jako wniosek z powyższego twierdzenia otrzymujemy następujące

\theorem{Na to by ograniczona funkcja $f$ byla całkowalna w sensie Riemanna na przedziale $[a,b]$ potrzeba i wystarcza, by}
\eq{}
\begin{equation}
\underline{\int_{a}^{b}fdx} = \overline{\int_{a}^{b}fdx}
\end{equation}
\proof{}W dowodzie Twierdzenia $\ref{theorem:76}$ pokazaliśmy, że (\ref{eq:38}) implikuje (\ref{eq:39}). Załóżmy teraz, że (\ref{eq:39}) zachodzi. Dla danej liczby $\varepsilon > 0$ istnieją podziały $P_1$ i $P_2$ przedziału $[a,b]$ takie, że
\[
    \underline{\int_{a}^{b}fdx} - \frac{\varepsilon}{2} < L(f, P_{1}), \quad U(f, P_{2}) < \overline{\int_{a}^{b}fdx} + \frac{\varepsilon}{2}
\]
Jeśli podział $P$ jest wspólnym rozdrobniemiem podziałów $P_{1}$ i $P_{2}$, to na mocy Twierdzenia $\ref{theorem:74}$ otrzymujemy
\[
    U(f, P) \leqslant U(f, P_2) < \overline{\int_{a}^{b}fdx} + \frac{\varepsilon}{2} = \underline{\int_{a}^{b}fdx} + \frac{\varepsilon}{2} < L(f, P_1) + \varepsilon \leqslant L(f, P) + \varepsilon
\]
Stąd $U(f, P) - L(f, P) \leqslant \varepsilon$, a zatem warunek (\ref{eq:38}) jest spełniony. Wobec Twierdzenia $\ref{theorem:76}$ dowód jest zakończony.

\defn{} Mówimy, że ograniczona funkcja $f$ jest całkowalna w sensie Riemanna, jeśli
\[
    \overline{\int_{a}^{b}fdx} = \underline{\int_{a}^{b}fdx}
\]
Wspólną wartość określoną powyższą równością nazywamy całką Riemanna funkcji $f$ na przedziale $[a,b]$.

Zbadamy teraz całkowalność w sensie Riemanna pewnych klas funkcji.

\theorem{Funkcja ciągła na przedziale $[a,b]$ jest na tym przedziale całkowalna w sensie Riemanna.}

\proof{}Funkcja $f$ jest jednostajnie ciągła na $[a,b]$ (por. Tw. 51), a zatem dla dowolnego $\varepsilon > 0$ istnieje $\delta > 0$ taka, 
że $|f(x) - f(t)| < \frac{\varepsilon}{b - a}$ dla wszystkich $x, t \in [a, b]$, dla których $|x - t| < \delta$. Niech $P$ będzie podziałem przedziału $[a,b]$,
dla którego $\delta(P) < \delta$. Wtedy mamy $M_i - m_i \leqslant \frac{\varepsilon}{b-a}$ dla $i = 1,\ldots,n$ i wobec tego
\[
    U(f, P) - L(f, P) = \sum_{i=1}^{n}(M_i - m_i)\Delta x_i \leqslant \frac{\varepsilon}{b-a}\sum_{i=1}^{n}\Delta x_i = \varepsilon
\]
Na mocy Twierdzenia $\ref{theorem:76}$ funkcja $f$ jest całkowalna w sensie Riemanna na $[a,b]$.

Udowodnimy teraz następujące uogólnienie powyższego twierdzenia.

\begin{theorem}
    {Jeśli $f$ jest funkcją ograniczoną i mającą tylko skończoną liczbę punktów nieciągłości na przedziale $[a,b]$, to jest ona całkowalna w sensie Riemanna na tym przedziale.}
\end{theorem}

\begin{proof}
    Ponieważ funkcja $f$ jest ograniczona, więc istnieją liczby rzeczywiste $m, M$ takie, że $m \leqslant f(x) \leqslant M$
    dla wszystkich $x \in [a,b]$. Załóżmy, że $f$ ma $k$ punktów nieciągłości na przedziale $[a,b]$. 
    Weźmy dowolne $\varepsilon > 0$ i $\delta_1 < \frac{\varepsilon}{8(M-m)k}$ (oczywiście $M \neq m$).
    Rozważmy przedziały otwarte $(x_l - \delta_1, x_l + \delta_1)$, $l = 1, \ldots, k$, gdzie $x_l$ są punktami
    nieciągłości funkcji $f$. Dopełnienie sumy tych przedziałów do przedziału $[a,b]$ składa się ze skończonej
    liczby przedziałów domkniętych, na których funkcja $f$ jest ciągła, a więc i jednostajnie ciągła. Ponieważ tych przedziałów
    jest skończenie wiele, więc dla danego $\varepsilon > 0$ istnieje liczba $\delta_2 > 0$ taka, że dla dowolnych
    punktów $x, t$ należacych do jednego z tych przedziałów, na których funkcja $f$ jest ciągła i spełniająca nierówność $|x-t| < \delta_2$ mamy
    $|f(x)-f(t)|<\frac{\varepsilon}{2(b-a)}$. Weźmy teraz liczbe $\delta = \min{(\delta_1, \delta_2)}$.
    Niech $P = \{x_0,\ldots,x_n\}$ będzię dowolnym podziałem przedziału $[a,b]$, dla którego $\delta(P) < \delta$. Ponadto
    rozbijmy zbiór indeksów $\{1, \ldots, n\}$ na dwa rozłącznę zbiory $A$ i $B$ w następujący sposób: do zbioru $A$ zaliczymy te liczby $i$, 
    dla których przedział $[x_{i-1}, x_i]$ nie ma punktów wspólnych z żadnym z skontruowanych powyżej otoczeń punktów $x_l$, $l=1,\ldots,k$, a do zbioru $B$
    pozostałe przedziały powstające z podziału $P$ przedziału $[a,b]$. Wówczas
    \[
        U(f, P) - L(f, P) = \sum_{i=1}^{n}(M_i - m_i)\Delta x_i = \sum_{i \in A}(M_i - m_i)\Delta x_i + \sum_{i \in B}(M_i - m_i)\Delta x_i
    \]
    Ponadto
    \[
        \sum_{i \in A}(M_i - m_i)\Delta x_i \leqslant \frac{\varepsilon}{2(b-a)}\sum_{i \in A}\Delta x_i \leqslant \frac{\varepsilon}{2(b-a)}(b-a)=\frac{\varepsilon}{2}
    \]
    Suma długości podprzedziałów przedziału $[a,b]$ indeksowanych przez liczby ze zbioru $B$ jest nie większa niż
    \[
        (\delta + 2\delta_i+\delta)k < 4 \frac{\varepsilon}{8(M-m)k}k = \frac{\varepsilon}{2(M-m)}
    \]
    Dlatego
    \[
        \sum_{i \in B}(M_i - m_i)\Delta x_i \leqslant (M-m)\sum_{i \in B}\Delta x_i < (M-m)\frac{\varepsilon}{2(M-m)}=\frac{\varepsilon}{2}
    \]
    Dla podziału $P$ o średnicy mniejszej niż $\delta$ otrzymujemy zatem
    \[
        U(f, P) - L(f, P) = \sum_{i=1}^{n}(M_i - m_i)\Delta x_i < \varepsilon
    \]
    co kończy dowód.
\end{proof}

\begin{uwaga}
    Twierdzenie $\ref{theorem:78}$ można istotnie uogólnić. Mianowicie dowodzi się, że jeśli $f$ jest ograniczoną funkcją
    na przedziale $[a,b]$, to jest ona całkowalna w sensie Riemanna na tym przedziale wtedy i tylko wtedy, gdy
    jest ona ciągła prawie wszędzie ma $[a,b]$, to znaczy zbiór punktów nieciągłości funkcji $f$ ma miarę Lebesgue'a równą zeru.
    (por. [7], s. 270).
    Przykładów takich funkcji dostarcza następujące
\end{uwaga}

\theorem{Funkcja monotoniczna na przedziale $[a,b]$ jest na tym przedziale całkowalna w sensie Riemanna.}

\proof{}
Załóżmy, że $f$ jest funkcją niemalejącą. Niech będzie dane dowolne $\varepsilon > 0$. Weźmy podział $P$ przedziału $[a,b]$ na $n$ równych części o długości
$\frac{b-a}{n}$. Ponieważ $f$ jest niemalejącą zatem $M_i = f(x_i)$ oraz $m_i = f(x_{i-1})$ dla $i = 1, \ldots, n$. Mamy więc
\[
    U(f, P) - L(f, P) = \sum_{i=1}^{n}(f(x_i) - f(x_{i-1}))\frac{b-a}{n} = (f(b)-f(a))\frac{b-a}{n}
\]
Biorąc $n$ tak duże, aby $(f(b) - f(a))\frac{b-a}{n} < \varepsilon$ i stosując twierdzenie $\ref{theorem:76}$ otrzymujemy tezę.
W przypadku funkcji nierosnącej dowód jest analogiczny.

\begin{theorem}
    {Jeśli $f$ jest całkowalna w sensie Riemanna na przedziale $[a,b]$, $m \leqslant f(x) \leqslant M$ dla $x \in [a,b]$ oraz $\phi$ jest funkcją ciągłą
    na $[m, M]$, to funkcja złożona $h = \phi \circ f$ jest R-całkowalna na $[a,b]$.}
\end{theorem}

\begin{proof}
    Ustalmy $\varepsilon > 0$. Ponieważ funkcja $\phi$ jest jednostajnie ciągła na $[M,m]$, więc
    istnieje $\delta > 0$ taka, że $\delta < \varepsilon$ i $|\phi(s) - \phi(t)| < \varepsilon$, jeśli $|s-t| < \delta$.
    Ponieważ $f$ jest R-całkowalna na $[a,b]$, więc istnieje podział $P = \{x_0, \ldots, x_n\}$
    przedziału $[a,b]$ taki, że $U(f, P) - L(f, P) < \delta^2$. Niech
    \[
        M_i=\sup\limits_{x_{i-1}\leqslant x \leqslant x_i}f(x), \quad m_i=\inf\limits_{x_{i-1}\leqslant x \leqslant x_i}f(x),
    \]
    \[
        M_i^*=\sup\limits_{x_{i-1}\leqslant x \leqslant x_i}h(x), \quad m_i^*=\inf\limits_{x_{i-1}\leqslant x \leqslant x_i}h(x)
    \]
    dla $i = 1, \ldots, n$. Podzielmy zbiór $\{1, \ldots, n\}$ na dwa rozłącznę zbiory $A$ i $B$ w taki sposób, że
    $i \in A$, jeśli $M_i - m_i < \delta$ oraz $i \in B$ w przypadku przeciwnym. Wówczas wobec powyższego wyboru $\delta$ mamy
    $M_i^* - m_i^* < \varepsilon$ dla $i \in A$. Natomiast dla $i \in B$ mamy $M_i^* - m_i^* \leqslant 2K$, gdzie $K=\sup{\{|\phi(t)|:m \leqslant t \leqslant M\}}$.
    Stąd otrzymujemy
    \[
        \delta \sum_{i \in B}(x_i - x_{i-1}) \leqslant \sum_{i \in B}(M_i - m_i)(x_i - x_{i-1}) < \delta^2, \quad \text{zatem} \sum_{i \in B}(x_i - x_{i-1}) < \delta
    \]
    Mamy więc
    \[
        U(h, P) - L(h, P) = \sum_{i \in A}(M_i^* - m_i^*)(x_i - x_{i-1}) + \sum_{i \in B}(M_i^* - m_i^*)(x_i - x_{i-1})
    \]
    a zatem
    \[
        U(h, P) - L(h, P) \leqslant \varepsilon(a + b + 2K)
    \]
    Ponieważ $\varepsilon$ było dowolne, zatem na mocy twierdzenia $\ref{theorem:76}$ funkcja $h$ jest R-całkowalna.
\end{proof}

Następujące twierdzenie opisuję związek całki Riemanna z operacjami arytmetycznymi. 

\begin{theorem}
    {Jeśli funkcje $f$ i $g$ są R-całkowalne na przedziale $[a,b]$, to również R-całkowalne są
    funkcje $f+g$, $\lambda f$ ($\lambda$ jest dowolną stałą rzeczywistą) i $fg$ oraz prawdziwe są równości:}
\end{theorem}
\begin{eq}
    \begin{equation}
        \int_{a}^{b} (f+g) (x) dx = \int_{a}^{b} f (x) dx + \int_{a}^{b} g (x) dx,
    \end{equation}
\end{eq}
\begin{eq}
    \begin{equation}
        \int_{a}^{b} (\lambda f) (x) dx = \lambda \int_{a}^{b} f (x) dx
    \end{equation}
\end{eq}

\proof{} Jest jasne, że dla dowolnego $R \in \mathfrak{R}(f+g, P)$ mamy $R = R_f + R_g$, gdzie $R_f \in \mathfrak{R}(f, P)$, $R_g \in \mathfrak{R}(g, P)$.
Niech $I_1 = \int_{a}^{b}f(x)dx$, $I_2 = \int_{a}^{b}f(x)dx$ oraz $I = I_1 + I_2$. Mamy
\[
    \forall_{\varepsilon > 0} \exists_{\delta > 0} \forall_P \forall_{R_f \in \mathfrak{R}(f, P)} \delta(P) < \delta \implies |R_f - I_1| < \frac{\varepsilon}{2} \quad \text{oraz}
\]
\[
    \forall_{\varepsilon > 0} \exists_{\delta > 0} \forall_P \forall_{R_g \in \mathfrak{R}(g, P)} \delta(P) < \delta \implies |R_g - I_2| < \frac{\varepsilon}{2} \quad \text{Stąd}
\]
\[
    \forall_{\varepsilon > 0} \exists_{\delta > 0} \forall_P \forall_{R \in \mathfrak{R}(f+g, P)} \delta(P) < \delta \implies |R - I| \leqslant |R_f - I_1| + |R_g - I_2| < \varepsilon
\]
Wobec powyższego jest jasne, że funkcja $f + g$ jest R-całkowalna na przedziale $[a,b]$ oraz, że spełniony jest wzór $\ref{eq:40}$.
Dowód wzoru $\ref{eq:41}$ jest analogiczny.

\noindent
Dalej przyjmując $\phi(t) = t^2$ oraz stosując do $\phi$ poprzednie twierdzenie (\ref{theorem:81}) otrzymujemy
R-całkowalność funkcji $f^2$.

\noindent
R-całkowalność iloczynu funkcji $fg$ wynika z tożsamości
\[
    fg = \frac{1}{4}[{(f+g)}^2 - {(f-g)}^2].
\]

\begin{theorem} 
{
    (a) Jeśli funkcje $f$ i $g$ są R-całkowalne na przedziale $[a,b]$ oraz $f(x) \leqslant g(x)$ dla każdego $x \in [a,b]$, to
    \[
        \int_{a}^{b}f(x)dx \leqslant \int_{a}^{b}g(x)dx
    \]
    (b) Jeśli funkcja $f$ jest R-całkowalna na przedziale $[a,b]$, to funkcja $|f|$ jest również
    R-całkowalna na tym przedziale oraz:
    \[
        \left\vert\int_{a}^{b}f(x)dx\right\vert \leqslant \int_{a}^{b}|f(x)|dx
    \]
}
\end{theorem}
\begin{proof}
    (a) Jeśli $m \leqslant f(x) \leqslant M$ dla $x \in [a,b]$, to
    \[
        m(b-a) \leqslant \int_{a}^{b}f(x)dx \leqslant M(b-a)
    \]
    Stąd, jeśli $f(x) \geqslant 0$ dla $x \in [a,b]$, to $\int_{a}^{b}f(x)dx \geqslant 0$.
    Wobec tego nierówność $f(x) \leqslant g(x)$ dla $x \in [a,b]$ implikuje
    \[
        \int_{a}^{b}f(x)dx \leqslant \int_{a}^{b}g(x)dx
    \]
    (b) Biorąc $\phi(t) = |t|$ w Twierdzeniu $\ref{theorem:81}$ otrzymujemy całkowalność funkcji $|f|$.
    Ponieważ $-|f(x)| \leqslant f(x) \leqslant |f(x)|$ dla $x \in [a,b]$, zatem na mocy (a) otrzymujemy
    \[
        \left\vert\int_{a}^{b}f(x)dx\right\vert \leqslant \int_{a}^{b}|f(x)|dx
    \]
\end{proof}
\begin{uwaga}
    (a) Punkt (a) Twierdzenia $\ref{theorem:83}$ można udowodnić bezpośrednio w oparciu o definicję
    całki Riemanna (Def. $\ref{def:69}$) oraz Wniosek 3 (b). \\
    \\
    \noindent
    (b) Twierdzenie odwrotne do Twierdzenia $\ref{theorem:83}$ (b) nie jest prawdziwe, to znaczy z R-całkowalności
    $|f|$ nie wynika R-całkowalność funkcji $f$. Dla przykładu niech
    \[ f(x) = \begin{cases}
        1 & \text{dla } x \in \mathbb{Q} \cap [a,b], \\
        -1 & \text{dla } x \in  (\mathbb{R \setminus Q}) \cap [a,b].
    \end{cases}
    \]
    Oczywiście funkcja $|f|$ jest R-całkowalna na przedziale $[a,b]$ oraz $\int_{a}^{b}|f(x)|dx = b-a$.
    Z kolei $underline{\int_{a}^{b}f(x)dx} = -(b-a)$ oraz $\overline{\int_{a}^{b}f(x)dx} = b-a$, a zatem
    wobec Twierdzenia $\ref{theorem:77}$ funkcja $f$ nie jest R-całkowalna na przedziale $[a,b]$.
\end{uwaga}

\begin{theorem}
    {Jeśli dwie funkcje $f$ i $g$ są równe na przedziale $[a,b]$ z wyjątkiem skończonego zbioru punktów ${\{x_1, \ldots, x_k\}}$
    i jedna z nich, na przykład $g$ jest R-całkowalna na tym przedziale, to druga też jest na nim R-całkowalna i zachodzi równość
    \[
        \int_{a}^{b}f(x)dx = \int_{a}^{b}g(x)dx.
    \]}
\end{theorem}
\begin{proof}
    Ponieważ $f = g + (f-g)$, więc wystarczy udowodnić, że funkcja $\phi = f - g$ jest R-całkowalna na $[a,b]$ i $\int_{a}^{b}\phi(x)dx = 0$.
    Oznaczmy $N = \max\{|\phi(x_1), \ldots, \phi(x_k)\}$. Niech $P$ będzie podziałem przedziału $[a,b]$ o średnicy $\delta$.
    Funkcja $\phi$ na co najwyżej $2k$ przedziałach podziału $P$ nie jest tożsamościowo równa zeru.
    Dlatego mamy $U(\phi, P) \leqslant 2Nk\delta$ i $L(\phi, P) = 0$, zatem $U(\phi, P) - L(\phi, P) \leqslant 2Nk\delta$. 
    Biorąc $\phi$ odpowiednio małe możemy uczynić różnicę $U(\phi, P) - L(\phi, P)$ dowolnie małą. To oznacza, że funkcja $\phi$ jest
    R-całkowalna. Ponadto jasne jest, że $\int_{a}^{b}\phi(x)dx = 0$.
\end{proof}

\begin{wniosek}
{Niech funkcja $f$ będzie określona i ograniczona na przedziale otwartym $(a,b)$. Jeśli po nadaniu jej pewnych wartości
$f(a)$ i $f(b)$ stanie się ona R-całkowalna na przedziale domkniętym $[a,b]$ --- to taką pozostanie --- gdy liczby $f(a)$ i $f(b)$ zmienimy
w sposób dowolny. Wartość całki nie ulegnie przy tym zmianie.}
\end{wniosek}

Następujący lemat pozwala przy przybliżaniu całki Riemanna sumami całkowymi ograniczyć się tylko do podziałów zawierających z góry ustalony punkt.

\begin{lemat}
{
    Niech $c \in [a,b]$ i niech $\Pi^*$ oznacza zbiór wszystkich podziałów przedziału $[a,b]$ spełniających warunek:
    \[
        P = \{x_0, \ldots, x_n\} \in \Pi^* \text{ wtedy i tylko wtedy $x_j = c$ dla pewnego $j$.} 
    \]
    Wówczas dla dowolnej funkcji $f$, ograniczonej na $[a,b]$ zachodzą równości:
    \[
        \sup_{P \in \Pi^*}L(f, P) = \sup_{P}L(f, P), \quad  \inf_{P \in \Pi^*}U(f, P) = \inf_{P}U(f, P)
    \]
}
\end{lemat}
\begin{proof}
    Ponieważ $\Pi^*$ jest podzbiorem zbioru wszystkich podziałów przedziału $[a,b]$, więc
    \begin{equation}\label{eq:42}
        \sup_{P \in \Pi^*}L(f, P) \leqslant \sup_{P}L(f, P), \quad  \inf_{P \in \Pi^*}U(f, P) \geqslant \inf_{P}U(f, P)
    \end{equation}
    Zauważmy, że dla dowolnego podziału $P$ przedziału $[a,b]$ istnieje podział od niego drobniejszy
    $P^* \in \Pi^*$. Istotnie, jeśli $P \in \Pi^*$, to przyjmujemy $P^* = P$. Jeśli natomiast $P \notin \Pi^*$,
    to przez dołączenie punktu $c$ do układu punktów wyznaczających $P$ otrzymujemy podział $P^*$ o żądanych własnościach.
    Mamy więc
    \[
        L(f, P) \leqslant L(f, P^*), \quad U(f, P) \geqslant U(f, P^*),
    \]
    skąd otrzymujemy
    \[
        L(f, P) \leqslant \sup_{P^* \in \Pi^*}L(f, P^*), \quad U(f, P) \geqslant \inf_{P^* \in \Pi^*}U(f, P^*).
    \]
    Wobec dowolności podziału $P$ mamy
    \[
        \sup_{P}L(f, P) \leqslant \sup_{P^* \in \Pi^*}L(f, P), \quad  \inf_{P}U(f, P) \geqslant \inf_{P^* \in \Pi^*}U(f, P)
    \]
    Z powyższych nierówności i z (\ref{eq:42}) otrzymujemy tezę.
\end{proof}

\begin{theorem}
{
    Niech $a < c < b$. Funkcja $f$ jest R-całkowalna na przedziale $[a,b]$ wtedy i tylko wtedy, gdy
    jest ona R-całkowalna na przedziałach $[a,c]$ i $[c,b]$. Zachodzi przy tym równość
    \[
        \int_{a}^{b}f(x)dx = \int_{a}^{c}f(x)dx + \int_{c}^{b}f(x)dx
    \]
    (addytywność całki ze względu na przedział)
}
\end{theorem}

\begin{proof}
    Załóżmy, że funkcja $f$ jest R-całkowalna na przedziale $[a,b]$. Na mocy powyższego lematu
    możemy ograniczyć się do podziałów przedziału $[a,b]$ zawierających punkt $c$. 
    Jeśli $P$ jest takim podziałem, to wówczas $P = P_1 \cup P_2$, gdzie $P_1$ jest podziałem
    przedziału $[a,c]$, a $P_2$ --- podziałem przedziału $[c,b]$ oraz mamy
    \[
        U(f, P) = U(f, P_1) + U(f, P_2), \quad L(f, P) = L(f, P_1) + L(f, P_2).
    \]
    Niech będzie dane dowolne $\varepsilon > 0$ i niech
    \[
        U(f, P) - L(f, P) < \frac{\varepsilon}{2}.
    \]
    Stąd $U(f, P_1) - L(f, P_1) < \frac{\varepsilon}{2}$ i $U(f, P_2) - L(f, P_2) < \frac{\varepsilon}{2}$. 
    Funkcja $f$ jest więc całkowalna na przedziałach $[a,c]$ i $[c, b]$ oraz zachodzą nierówności
    \[
        U(f, P_1) < \int_{a}^{c}f(x)dx + \frac{\varepsilon}{2}, \quad \int_{a}^{c}f(x)dx < L(f, P_1) + \frac{\varepsilon}{2},
    \]
    \[
        U(f, P_2) < \int_{c}^{b}f(x)dx + \frac{\varepsilon}{2}, \quad \int_{c}^{b}f(x)dx < L(f, P_2) + \frac{\varepsilon}{2},
    \]
    Wobec powyższego otrzymujemy $U(L, P) < \int_{a}^{c}f(x)dx + \int_{c}^{b}f(x)dx + \varepsilon$
    i w konsekwencji $\int_{a}^{b}f(x)dx < \int_{a}^{c}f(x)dx + \int_{c}^{b}f(x)dx + \varepsilon$.
    Ponieważ $\varepsilon > 0$ było dowolne, zatem
    \begin{equation}\label{eq:43}
        \int_{a}^{b}f(x)dx \leqslant \int_{a}^{c}f(x)dx + \int_{c}^{b}f(x)dx.
    \end{equation}
    Analogicznie $\int_{a}^{c}f(x) + \int_{c}^{b}f(x)dx < L(f, P) + \varepsilon$, skąd
    \begin{equation}\label{eq:44}
        \int_{a}^{c}f(x)dx + \int_{c}^{b}f(x)dx \leqslant \int_{a}^{b}f(x)dx,
    \end{equation}
    bowiem $\varepsilon > 0$ jest dowolne. Z nierówności (\ref{eq:43}) i (\ref{eq:44}) otrzymujemy żądaną równość.
    Uzasadnienie implikacji odwrotnej jest analogiczne.
\end{proof}

Rozszerzymy teraz zasięg Definicji $\ref{def:69}$.

\begin{defn}
    W przypadku gdy $b < a$ lub $b = a$, to całkę Riemanna z funkcji $f$ określamy wzorami
    \[
        \int_{a}^{b}f(x)dx = -\int_{b}^{a}f(x)dx \quad \text{lub odpowiednio} \quad \int_{a}^{b}f(x)dx = 0.
    \]
    W całce $\int_{a}^{b}f(x)dx$ liczbę $a$ nazywamy dolną granicą całkowania, liczbę $b$ -- górną granicą całkowania,
    bez względu na to, czy $b \geqslant a$, czy też $b < a$. 
\end{defn}

\begin{wniosek}
{
(a) Niech $a,b,c \in \mathbb{R}$ i niech $f$ będzie funkcją R-całkowalna na najwiekszym z przedziałów
domkniętych o końcach we wskazanych punktach. Wówczas obcięcie funkcji $f$ do każdego z dwóch pozostałych przedziałów domkniętych jest
funkcja R-całkowalną na odpowiednim przedziale oraz zachodzi równość
\begin{equation}\label{eq:45}
    \int_{a}^{b}f(x)dx + \int_{b}^{c}f(x)dx + \int_{c}^{a}f(x)dx = 0.
\end{equation}
}
\end{wniosek}
\begin{proof}
    Wobec symetrii równość (\ref{eq:45}) względem $a, b, c$ możemy bez straty ogólności założyć, że
    $a = \min{\{a,b,c\}}$. Jeśli $\max{\{a,b,c\} = c}$ oraz $a < b < c$, to na mocy
    Twierdzenia $\ref{theorem:85}$ mamy
    \[
        \int_{a}^{b}f(x)dx + \int_{c}^{b}f(x)dx - \int_{a}^{c}f(x)dx = 0,
    \]
    zatem wobec Definicji $\ref{def:74}$ otrzymujemy równość (\ref{eq:45}). \\
    Jeśli $\max{\{a,b,c\} = b}$ oraz $a < c < b$, to ponownie na mocy Twierdzenia
    $\ref{theorem:85}$ mamy
    \[
        \int_{a}^{c}f(x)dx + \int_{c}^{b}f(x)dx - \int_{a}^{b}f(x)dx = 0,
    \]
    stąd wobec Definicji $\ref{def:74}$ wynika równość (\ref{eq:45}). \\
    W końcu, jeśli jakiekolwiek dwa z punktów $a,b,c$ lub wszystkie trzy pokrywają się,
    to (\ref{eq:45}) jest bezpośrednią konsekwencją Definicji $\ref{def:74}$.
\end{proof}

\noindent
\textit
{
(b) Jeśli funkcja $f$ jest R-całkowalna na przedziale $[a,b]$ i $a \leqslant c < d \leqslant d$, to jest
ona również R-całkowalna na przedziale $[c,d]$.
}

\begin{defn}
    Niech każdej uporządkowanej parze $(\alpha, \beta)$ punktów $\alpha$, $\beta$ przedziału
    $[a,b]$ odpowiada dokładnie jedna liczba $I(\alpha, \beta)$, przy czym dla dowolnej trójki punktów
    $\alpha, \beta, \gamma \in [a,b]$ zachodzi równość
    \[
        I(\alpha, \gamma) = I(\alpha, \beta) + I(\beta, \gamma).
    \]
    Wówczas funkcja $I(\alpha, \beta)$ nazywa się addytywną funkcją przedziału zorientowanego
    (dla $\alpha = \gamma$ wobec powyższej równości otrzymujemy $I(\alpha, \beta) = -I(\beta, \alpha)$), 
    określoną na odcinkach zawartych w przedziale $[a,b]$. 
\end{defn}

\begin{wniosek}
{
    Jeśli funckja $f$ jest R-całkowalna na przedziale $[a,b]$ oraz $\alpha, \beta, \gamma \in [a,b]$, to
    kładąc $I(\alpha, \beta) = \int_{\alpha}^{\beta}f(x)dx$, na mocy równości $\ref{eq:45}$ otrzymujemy
    \[
        \int_{\alpha}^{\gamma}f(x)dx = \int_{\alpha}^{\beta}f(x)dx + \int_{\beta}^{\gamma}f(x)dx,
        \quad \text{czyli} \quad I(\alpha, \gamma) = I(\alpha, \beta) + I(\beta, \gamma),
    \]
    czyli całka Riemanna jest adytywną funkcją przedziału zorientowanego.
}
\end{wniosek}

Udowodnimy teraz ważne twierdzenie o funkcji górnej granicy całkowania.

\begin{theorem}
{
    Niech $f$ będzie funkcja R-całkowalną na przedziale $[a,b]$. Dla dowolnego punktu
    $x \in [a,b]$ określamy
    \[
        F(x) = \int_{a}^{x}f(t)dt.
    \]
    Wówczas funkcja $F$ jest ciągła na przedziale $[a,b]$. Ponadto, jesli funkcja $f$ jest ciągła w
    punkcie $x_0 \in [a,b]$, to funkcja $F$ jest różniczkowalna w tym punkcie oraz
    $F'(x_0) = f(x_0)$.
}
\end{theorem}

\begin{proof}
    Niech $M$ będzie takie, że $|f(t)| \leqslant M$ dla $t \in [a,b]$. 
    Wówczas, jeśli $a \leqslant x \leqslant y \leqslant b$, to
    \[
        |F(y) - F(x)| = \left|\int_{x}^{y}f(t)dt\right| \leqslant M(y - x).
    \]
    Stąd wynika natychmiast, że dla dowolnego $\varepsilon > 0$ mamy
    $|F(y) - F(x)|< \varepsilon$, jeśli tylko $|y-x| < \frac{\varepsilon}{M}$, 
    a zatem funkcja F jest ciągła. \\
    Załóżmy, że $f$ jest ciągła w punkcie $x_0$. Dla danego $\varepsilon > 0$ wybierzmy 
    $\delta > 0$ tak, aby $|f(t)-f(x_0)| < \varepsilon$ jeśli tylko $|t-x_0| < \delta$ i $a \leqslant t \leqslant b$.
    Wówczas dla $s,t \in (x_0 - \delta, x_0 + \delta)$, $s,t \in [a,b]$, $s \neq t$ mamy
    \[
        \left|\frac{F(t)-F(s)}{t-s} - f(x_0)\right| = \left|\frac{1}{t-s}\int_{s}^{t}(f(u)-f(x_0))du \right| < \varepsilon,
    \]
    Stąd wynika, że $F'(x_0) = f(x_0)$, co kończy dowód.
\end{proof}

Oznaczmy $F(x) = I(a, x)$ dla $x \in [a,b]$, gdzie $I$ oznacza addytywną funkcję przedziału zorientowanego. Mamy
\[
    I(\alpha, \beta) = I(a, \beta) - I(a, \alpha) = F(\beta) - F(\alpha)
\]
dla każdej uporządkowanej pary punktów $(\alpha, \beta)$ z przedzialu $[a,b]$.
W ten sposób każda addytywna funkcja przedziału zorientowanego ma postać
\begin{equation}\label{eq:46}
    I(\alpha, \beta) = F(\beta) - F(\alpha),
\end{equation}
gdzie $x \mapsto F(x)$ jest funkcją określoną na przedziale $[a,b]$. 
Można łatwo sprawdzić, że jest również na odwrót, to znaczy, że z dowolnej funkcji $x \mapsto F(x)$ określonej na przedziale $[a,b]$
można przy pomocy (\ref{eq:46}) otrzymać addytywną funkcję przedziału zorientowanego.

\begin{wniosek}
    Jeśli $f$ jest funkcją R-całkowalną na przedziale $[a,b]$, to na mocy (\ref{eq:46})
    funkcja $F(x) = \int_{a}^{x}f(t)dt$ generuje addytywną funkcję 
    \[
        I(\alpha, \beta) = \int_{\alpha}^{\beta}f(t)dt.
    \]
\end{wniosek}

\subsection{Całka nieoznaczona}

\begin{defn}
    Niech $f$ będzie funkcją określoną na pewnym przedziale $I$. Każdą funkcję $F$ różniczkowalną na tym przedziale
    i spełniającą w każdym punkcie $x \in I$ równość
    \[
        F'(x) = f(x)
    \]
    nazywamy funkcją pierwotną funkcji $f$. Funkcję pierwotną nazywamy również całką
    nieoznaczoną danej funkcji i oznaczamy symbolem $\int f(x)dx$ (symbol ten należy również
    rozumieć jako oznaczenie dowolnej funkcji pierwotnej funkcji $f$ na tym przedziale).
    W symbolu tym znak $f$ nazywa się znakiem całki nieoznaczonej, $f$ --- funkcją podcałkową, a $f(x)dx$ ---
    wyrażeniem podcałkowym.
\end{defn}

\begin{uwaga}
    Jeśli $F$ jest funkcją pierwotną funkcji $f$, to suma $F + c$, gdzie $c$ jest dowolną stałą, jest
    również funkcją pierwotną funkcji $f$, bowiem $(F+c)' = F' = f$. \\
    Na odwrót, dwie dowolne fukcje pierwotne $F$ i $G$ tej samej funkcji $f$ róźnią się o stałą,
    bowiem $(F-G)'= f-f = 0$. \\
    Jeśli $F$ jest więc konkretną funkcją pierwotną funkcji $f$ na przedziale $I$, to na tym przedziale
    \[
        \int f(x)dx = F(x) + c,
    \]
    to znaczy dowolna inna funkcja pierwotna funkcji $f$ może być otrzymana z danej funkcji $F$ przez dodanie stałej.

\end{uwaga}

\end{justify}
\end{document}