\documentclass[leqno]{article}
\usepackage{graphicx} % Required for inserting images
\usepackage[polish]{babel} % Language support for Polish
\usepackage[utf8]{inputenc} % Input encoding
\usepackage[T1]{fontenc}
\usepackage{ragged2e}
\usepackage{amssymb} % Load before mathrsfs
\usepackage{mathrsfs} % Pakiet mathrsfs dostarcza komendę \mathscr
\usepackage{hyperref} % for clickable links
\usepackage{amsmath} % Load after amssym
\usepackage{lmodern}
\usepackage{array, makecell}

\setlength{\parindent}{0pt} % Remove indentation globally
\setlength{\parskip}{10pt}

\setcounter{equation}{35}

\title{\Huge{Analiza Matematyczna 2}}
\author{Adrian Madajewski}
\date{Semestr II}

\DeclareMathOperator{\tg}{\text{tg}}
\DeclareMathOperator{\ctg}{\text{ctg}}
\DeclareMathOperator{\arctg}{\text{arctg}}
\DeclareMathOperator{\arcctg}{\text{arcctg}}

% Twierdzenie
\newcounter{thcounter}
\setcounter{thcounter}{73}
\newcommand{\theorem}[1]{\noindent\refstepcounter{thcounter}\textbf{Twierdzenie \thethcounter. }\textit{#1}\label{theorem:\thethcounter}}

% Lemat
\newcounter{lematcounter}
\setcounter{lematcounter}{1}
\newcommand{\lemat}[1]{\noindent\refstepcounter{lematcounter}\textbf{Lemat \thelematcounter. }\textit{#1}\label{theorem:\thelematcounter}}

% Wniosek
\newcounter{wniosekcounter}
\setcounter{wniosekcounter}{17}
\newcommand{\wniosek}[1]{\noindent\refstepcounter{wniosekcounter}\textbf{Wniosek \thewniosekcounter. }\textit{#1}\label{wniosek:\thewniosekcounter}}

% Definicja
\newcounter{defcounter}
\setcounter{defcounter}{67}
\newcommand{\defn}{\noindent\refstepcounter{defcounter}\textbf{Definicja \thedefcounter. }\label{def:\thedefcounter}}

% Uwaga
\newcounter{uwagacounter}
\setcounter{uwagacounter}{37}
\newcommand{\uwaga}{\noindent\refstepcounter{uwagacounter}\textbf{Uwaga \theuwagacounter. }\label{uwaga:\theuwagacounter}}

% Przyklad
\newcounter{excounter}
\setcounter{excounter}{24}
\newcommand{\ex}{\noindent\refstepcounter{excounter}\textbf{Przykład \theexcounter. }\label{ex:\theexcounter}}

% Dowód
\newcommand{\proof}{\noindent\textbf{Dowód. }}

% Equation
\newcounter{eqcounter}
\setcounter{eqcounter}{38}
\newcommand{\eq}{\refstepcounter{eqcounter}\label{eq:\theeqcounter}}

\begin{document}

\pagenumbering{arabic}

\maketitle

\begin{center}
    Niniejszy plik jest w całości bazowany na wykładach \\
    \vspace{10pt}
    prof. dr hab. Dariusza Bugajewskiego \\
    \vspace{10pt}
    z przedmiotu \\
    \vspace{10pt}
    Analiza Matematyczna 2 \\
    \vspace{10pt}
    na Uniwersytecie im. Adama Mickiewicza w Poznaniu
    \includegraphics[width=0.8\textwidth]{uam_logo.pdf}
\end{center}
\newpage

\begin{justify}

\setcounter{section}{7}
\section{Całka Riemanna}
\subsection{Definicja i podstawowe własności całki}
\defn{}{} Niech $[a,b]$ będzie danym przedziałem. Przez podział $P$ przedziału $[a,b]$ będziemy nazywali skończony zbiór punktów \(x_0, x_1, \ldots, x_n\), gdzie
\[
a = x_0 < x_1 < \cdots < x_n = b
\]
Będziemy pisać $\Delta x_i = x_i - x_{i-1}$ $(i = 1, \ldots , n)$. Długość największego z odcinków $[x_{i-1}, x_i]$ nazywać będziemy średnicą podziału $P$ i oznaczamy ją symbolem $\delta (P)$. $\delta (P)= \max\limits_{1 \leqslant i \leqslant n} \Delta x_i$. Niech $f$ będzie ograniczoną funkcją rzeczywistą określoną na $[a,b]$. W każdym z przedziałów $[x_{i-1}, x_i]$ wybierzmy dowolny punkt $\xi_{i}$ $(i=1, \ldots n)$ i utwórzmy sumę $R = \sum_{i=1}^{n} f(\xi_{i})\Delta x_{i}$. Sumę te nazywamy sumą Riemanna odpowiadającą podziałowi $P$, przy ustalonym wyborze punktów $\xi_i$. Przez $\mathfrak{R}(f,P)$ oznaczać będziemy zbiór wszystkich możliwych sum Riemanna odpowiadających podziałowi $P$. Utwórzmy teraz ciąg $(P_k)$ podziałów przedziału $[a, b]$:
\[
a = x_0^{(k)} < x_1^{(k)} < \cdots < x_{n(k)}^{(k)} = b;
\]
\[
\Delta_i^{(k)}=x_i^{(k)}-x_{i-1}^{(k)};
\]
\[
\delta (P_k)=\max_{1 \leqslant i \leqslant n(k)} \Delta x_i^{(k)}, k = 1,2, \ldots
\]
Ciąg $(P_k)$ nazywamy ciągiem normalnym podziałów, jeśli $\delta(P_k)\rightarrow 0$  przy $k \rightarrow \infty$. Oznaczmy przez $\mathfrak{R}(f,P_k)$ zbiór wszystkich sum Riemanna odpowiadających podziałowi $P_k$.

\noindent
\defn{} Jeśli dla dowolnego ciągu normalnego podziałów $(P_k)$ i dla dowolnych sum Riemanna $R_k \in \mathfrak{R}(f,P_k)$ istnieje skończona granica $I = \lim\limits_{k \rightarrow \infty}R_k$, to tę granicę nazywamy całką Riemanna funkcji $f$ na przedziale $[a,b]$ i oznaczamy ją symbolem 
\[
\begin{aligned}
    \int_{a}^{b} fdx \text{ lub } \int_{a}^{b} f(x)dx
\end{aligned}
\]
O funkcji $f$ mówimy wówczas, że jest całkowalna w sensie Riemanna na przedziale $[a,b]$, lub że jest ona R-całkowalna na tym przedziale. 

Powyższą definicję mozna sformułować w następujący równoważny sposób.

\noindent
\defn{} Mówimy, że funkcja $f$ jest całkowalna w sensie Riemanna na przedziale $[a,b]$, jeśli istnieje liczba $I \in \mathbb{R}$ taka, że
\[
\forall_{\varepsilon>0} \exists_{\delta > 0} \forall_P \forall_{R \in \mathfrak{R}(f, P)}\delta(P) < \delta \implies |R - I| < \varepsilon
\]
Piszemy wówczas $I = \int_{a}^{b}f(x)dx = \lim\limits_{\delta(P) \rightarrow 0}R$.

Równoważność definicji $\ref{def:69}$ i $\ref{def:70}$ można pokazać analogicznie jak w dowodzie twierdzenia 35. 

\ex{} (a) Funkcja stała $f(x)=c, c \in \mathbb{R}, x \in [a,b]$ jest całkowalna w sensie Riemanna na tym przedziale. Niech $P$ będzie dowolnym podziałem przedziału $[a,b]$:
\[
a = x_0 < x_1 < \cdots < x_n = b
\]
Dowolna suma Riemanna odpowiadającą podziałowi $P$ ma postać:
\[
R = \sum_{i=1}^{n}f(\xi_i)(x_i-x_{i-1}) = \sum_{i=1}^{n}c(x_i - x_{i-1})=c(b-a),
\]
\[
    (\xi_i \in [x_{i-1},x_i], i=1, \ldots, n)
\]
Stąd wynika, że $\int_{a}^{b} f(x)dx = c(b-a)$. 

\noindent
(b) Roważmy ponownie funkcję Dirichleta z Przykładu 18 (a), zawężoną do przedziału $[a,b]$. Dla każdego podziału $P$ przedziału $[a,b]$ można utworzyć sumę Riemanna równą zeru, jeśli wszystkie punkty $\xi_i$ będą liczbami niewymiernymi, lub równą $(b-a)$, jeśli wszystkie punkty $\xi_i$ będą liczbami wymiernymi. Jest więc jasne, że dla każdego ciągu normalnego podziałów $(P_k)$ granica $\lim\limits_{k \rightarrow \infty}R_k$, gdzie $R_k \in \mathfrak{R}(f, P_k)$, $k \in \mathbb{N}$, nie istnieje. 

\defn{}\label{def:sumy_dolne_gorne} Niech $f$ będzie ograniczoną funkcją rzeczywistą określona na $[a,b]$. Każdemu podziałowi $P$ przedziału $[a,b]$ odpowiadają liczby:
\[
M_i = \sup\limits_{x_{i-1} \leqslant x \leqslant x_i} f(x) \quad m_i = \inf\limits_{x_{i-1} \leqslant x \leqslant x_i} f(x)
\]
\[
U(f, P) = \sum_{i=1}^{n}M_i\Delta x_i \quad L(f, P) = \sum_{i=1}^{n}m_i\Delta x_i
\]
Liczby $U(f,P)$ i $L(f, P)$ nazywamy odpowiednio sumą górną i dolną lub sumami Darboux funkcji $f$ przy podziale $P$ przedziału $[a,b]$. Dalej,

\begin{equation}\label{eq:36}
    \overline{\int_{a}^{b} f(x)dx} = \inf_P U(f, P),
\end{equation}

\begin{equation}\label{eq:37}
    \underline{\int_{a}^{b} f(x)dx} = \sup_P L(f, P),
\end{equation}
gdzie kres górny i dolny są brane ze względu na wszystkie podziały $P$ przedziału $[a,b]$. Lewe strony równości (\ref{eq:36}) i (\ref{eq:37}) nazywają się odpowiednio górną i dolną całką Darboux funkcji $f$ na przedziale $[a,b]$.

Ponieważ funkcja $f$ jest ograniczona, więc istnieją liczby rzeczywistę $m$ i $M$ takie, że
\[
m \leqslant f(x) \leqslant M \quad \text{dla} \quad x \in [a,b] 
\]
Oznacza to, że przy dowolnym podziale $P$ przedziału $[a,b]$ mamy
\[
m(b-a) \leqslant L(f, P) \leqslant U(f, P) \leqslant M(b-a)
\]
a zatem zbiory $\{L(f,P) : P\}$ i $\{U(f,P) : P\}$ są ograniczone. Wynika stąd, że całki górna i dolna są określone przy dowolnej funkcji ograniczonej $f$. 

\defn{} Mówimy, że podział $P^*$ przedziału $[a,b]$ jest rozdrobnieniem (lub zagęszczeniem) podziału $P$ tego przedziału, jeśli $P \subset  P^*$, to znaczy, jeśli każdy punkt przedziału $P$ jest także punktem przedziału $P^*$. Jeśli dane są dwa podziały $P_1$,$P_2$, to podział $P^*=P_1 \cup P_2$ nazywać będziemy ich wspólnym rozdrobnieniem (lub wspólnym zagęszczeniem).

\theorem{Jeśli $P^*$ jest rozdrobnieniem podziału $P$, to}
\[
L(f,P) \leqslant L(f,P^*) \quad U(f,P) \leqslant U(f,P^*)
\]
\proof{}Załóżmy wpierw, że $P^*$ zawiera tylko o jeden punkt więcej niż $P$. Niech tym dodatkowym punktem będzie $x^*$ i niech $x_{i-1} < x^* < x_i$, gdzie $x_{i-1}, x_i$ są dwoma kolejnymi punktami przedziału $P$. Przyjmijmy
\[
\omega_1 = \inf\limits_{x_{i-1} \leqslant x \leqslant x^*} f(x) \text{,} \quad \omega_2 = \inf\limits_{x^* \leqslant x \leqslant x_i} f(x) 
\]
Wtedy $\omega_1 \geqslant m_i$ i $\omega_2 \geqslant m_i$, gdzie $m_i = \inf\limits_{x_{i-1} \leqslant x \leqslant x_i} f(x)$. Mamy więc
\begin{equation*}
\begin{gathered}
    L(f,P^*)-L(f,P) = \omega_1(x^* - x_{i-1}) + \omega_2(x_i - x^*) - m_i(x_i - x_{i-1}) \\
    = (\omega_1-m_i)(x^*-x_{i-1})+(\omega_2-m_i)(x_i-x^*) \geqslant 0
\end{gathered}
\end{equation*}
Jeśli $P^*$ zawiera o $k$ punktów więcej niż $P$, to powtarzając powyższe rozumowanie $k$ razy otrzymamy pierwszą nierówność tezy. Dowód drugiej przebiega analogicznie.

\theorem{Jeśli $f$ jest funkcją ograniczoną na przedziale $[a,b]$, to}
\[
\underline{\int_{a}^{b} f(x)dx} \leqslant \overline{\int_{a}^{b} f(x)dx}
\]
\proof{}Niech $P^*$ będzie wspólnym rozdrobnieniem podziałów $P_1$ i $P_2$ przedziału $[a,b]$. Z Twierdzenia $\ref{theorem:75}$ wynika, że
\[
L(f, P_1) \leqslant L(f, P^*) \leqslant U(f, P^*) \leqslant U(f, P_2)
\]
Stąd $L(f, P_1) \leqslant U(f, P_2)$. Traktując $P_2$ jako ustalone i obliczając kres górny ze względu na wszystkie podziały $P_1$, wobec poprzedniej nierówności otrzymujemy
\[
\underline{\int_{a}^{b} f(x)dx} \leqslant U(f, P_2)
\]
Przechodząc do kresu dolnego ze względu na wszystkie podziały $P_2$ otrzymujemy tezę dowodzonego twierdzenia.

Udowodnimy teraz dwa kryteria całkowalności funkcji w sensie Riemanna. W oparciu o drugie z tych kryteriów podamy równoważną definicję całki w sensie Riemanna.

\theorem{Na to, aby ograniczona funkcja $f$ była całkowalna w sensie Riemanna na przedziale $[a,b]$ potrzeba i wystarcza, aby dla dowolnego $\varepsilon > 0$ istniał taki podział $P$ przedziału $[a,b]$, że}
\begin{equation}\label{eq:38}
U(f, P) - L(f, P) \leqslant \varepsilon
\end{equation}
\proof{}Załóżmy wpierw, że funkcja $f$ jest całkowalna w sensie Riemanna na przedziale $[a,b]$. Wówczas dla każdego danego $\varepsilon > 0$ istnieje taki podział $P$ przedziału $[a,b]$, że nierówność
\begin{equation*}
\begin{gathered}
    |R - \int_{a}^{b} f(x)dx| < \frac{\varepsilon}{2} \text{, czyli} \\
    \int_{a}^{b} f(x)dx - \frac{\varepsilon}{2} < R < \int_{a}^{b} f(x)dx + \frac{\varepsilon}{2}
\end{gathered}
\end{equation*}
jest spełniona przy dowolnym wyborze punktów $\xi_i$ w każdym z przedziałów podziału. Ponieważ sumy Darboux są --- przy danym podziale przedziału --- odpowiednio kresem górnym i dolnym sum całkowych, zatem spełniają one nierówności
\[
\int_{a}^{b}f(x)dx - \frac{\varepsilon}{2} \leqslant L(f,P) \leqslant U(f, P) \leqslant \int_{a}^{b}f(x)dx + \frac{\varepsilon}{2}
\]
a więc $U(f, P) - L(f, P) < \varepsilon$.
Załóżmy teraz, że (\ref{eq:38}) zachodzi. Dla dowolnego podziału $P$ mamy
\[
L(f, P) \leqslant \underline{\int_{a}^{b}f(x)dx} \leqslant \overline{\int_{a}^{b}f(x)dx} \leqslant U(f, P)
\]
Jeśli $U(f, P) - L(f, P) < \varepsilon$, to wówczas
\[
0 \leqslant \overline{\int_{a}^{b}f(x)dx} - \underline{\int_{a}^{b}f(x)dx} < \varepsilon
\]
Z dowolności $\varepsilon > 0$ wynika, że $\underline{\int_{a}^{b}f(x)dx} = \overline{\int_{a}^{b}f(x)dx}$. Oznaczając ponadto $\underline{\int_{a}^{b}f(x)dx} = \overline{\int_{a}^{b}f(x)dx} = I$ mamy $L(f, P) \leqslant I \leqslant U(f, P)$.
Ustalmy $\varepsilon > 0$ i niech $P$ bedzie danym podziałem przedziału $[a,b]$, dla którego (\ref{eq:38}) zachodzi. Jeśli przez $R$ oznaczymy jedną z wartości sum Riemanna odpowiadającej podziałowi $P$, to
\[
L(f,P) \leqslant R \leqslant U(f, P)
\]
Ponieważ liczby $R$ oraz $I$ znajdują się w przedziale $[L(f, P), U(f, P)]$, zatem
\[
|R - I| \leqslant \varepsilon
\]
Wobec Twierdzenia $\ref{theorem:74}$ oraz Definicji $\ref{def:70}$ wnioskujemy, że $I = \int_{a}^{b}f(x)dx$

Jako wniosek z powyższego twierdzenia otrzymujemy następujące

\theorem{Na to by ograniczona funkcja $f$ byla całkowalna w sensie Riemanna na przedziale $[a,b]$ potrzeba i wystarcza, by}
\eq{}
\begin{equation}
\underline{\int_{a}^{b}fdx} = \overline{\int_{a}^{b}fdx}
\end{equation}
\proof{}W dowodzie Twierdzenia $\ref{theorem:76}$ pokazaliśmy, że (\ref{eq:38}) implikuje (\ref{eq:39}). Załóżmy teraz, że (\ref{eq:39}) zachodzi. Dla danej liczby $\varepsilon > 0$ istnieją podziały $P_1$ i $P_2$ przedziału $[a,b]$ takie, że
\[
    \underline{\int_{a}^{b}fdx} - \frac{\varepsilon}{2} < L(f, P_{1}), \quad U(f, P_{2}) < \overline{\int_{a}^{b}fdx} + \frac{\varepsilon}{2}
\]
Jeśli podział $P$ jest wspólnym rozdrobniemiem podziałów $P_{1}$ i $P_{2}$, to na mocy Twierdzenia $\ref{theorem:74}$ otrzymujemy
\[
    U(f, P) \leqslant U(f, P_2) < \overline{\int_{a}^{b}fdx} + \frac{\varepsilon}{2} = \underline{\int_{a}^{b}fdx} + \frac{\varepsilon}{2} < L(f, P_1) + \varepsilon \leqslant L(f, P) + \varepsilon
\]
Stąd $U(f, P) - L(f, P) \leqslant \varepsilon$, a zatem warunek (\ref{eq:38}) jest spełniony. Wobec Twierdzenia $\ref{theorem:76}$ dowód jest zakończony.

\defn{} Mówimy, że ograniczona funkcja $f$ jest całkowalna w sensie Riemanna, jeśli
\[
    \overline{\int_{a}^{b}fdx} = \underline{\int_{a}^{b}fdx}
\]
Wspólną wartość określoną powyższą równością nazywamy całką Riemanna funkcji $f$ na przedziale $[a,b]$.

Zbadamy teraz całkowalność w sensie Riemanna pewnych klas funkcji.

\theorem{Funkcja ciągła na przedziale $[a,b]$ jest na tym przedziale całkowalna w sensie Riemanna.}

\proof{}Funkcja $f$ jest jednostajnie ciągła na $[a,b]$ (por. Tw. 51), a zatem dla dowolnego $\varepsilon > 0$ istnieje $\delta > 0$ taka, 
że $|f(x) - f(t)| < \frac{\varepsilon}{b - a}$ dla wszystkich $x, t \in [a, b]$, dla których $|x - t| < \delta$. Niech $P$ będzie podziałem przedziału $[a,b]$,
dla którego $\delta(P) < \delta$. Wtedy mamy $M_i - m_i \leqslant \frac{\varepsilon}{b-a}$ dla $i = 1,\ldots,n$ i wobec tego
\[
    U(f, P) - L(f, P) = \sum_{i=1}^{n}(M_i - m_i)\Delta x_i \leqslant \frac{\varepsilon}{b-a}\sum_{i=1}^{n}\Delta x_i = \varepsilon
\]
Na mocy Twierdzenia $\ref{theorem:76}$ funkcja $f$ jest całkowalna w sensie Riemanna na $[a,b]$.

Udowodnimy teraz następujące uogólnienie powyższego twierdzenia.

\begin{theorem}
    {Jeśli $f$ jest funkcją ograniczoną i mającą tylko skończoną liczbę punktów nieciągłości na przedziale $[a,b]$, to jest ona całkowalna w sensie Riemanna na tym przedziale.}
\end{theorem}

\begin{proof}
    Ponieważ funkcja $f$ jest ograniczona, więc istnieją liczby rzeczywiste $m, M$ takie, że $m \leqslant f(x) \leqslant M$
    dla wszystkich $x \in [a,b]$. Załóżmy, że $f$ ma $k$ punktów nieciągłości na przedziale $[a,b]$. 
    Weźmy dowolne $\varepsilon > 0$ i $\delta_1 < \frac{\varepsilon}{8(M-m)k}$ (oczywiście $M \neq m$).
    Rozważmy przedziały otwarte $(x_l - \delta_1, x_l + \delta_1)$, $l = 1, \ldots, k$, gdzie $x_l$ są punktami
    nieciągłości funkcji $f$. Dopełnienie sumy tych przedziałów do przedziału $[a,b]$ składa się ze skończonej
    liczby przedziałów domkniętych, na których funkcja $f$ jest ciągła, a więc i jednostajnie ciągła. Ponieważ tych przedziałów
    jest skończenie wiele, więc dla danego $\varepsilon > 0$ istnieje liczba $\delta_2 > 0$ taka, że dla dowolnych
    punktów $x, t$ należacych do jednego z tych przedziałów, na których funkcja $f$ jest ciągła i spełniająca nierówność $|x-t| < \delta_2$ mamy
    $|f(x)-f(t)|<\frac{\varepsilon}{2(b-a)}$. Weźmy teraz liczbe $\delta = \min{(\delta_1, \delta_2)}$.
    Niech $P = \{x_0,\ldots,x_n\}$ będzię dowolnym podziałem przedziału $[a,b]$, dla którego $\delta(P) < \delta$. Ponadto
    rozbijmy zbiór indeksów $\{1, \ldots, n\}$ na dwa rozłącznę zbiory $A$ i $B$ w następujący sposób: do zbioru $A$ zaliczymy te liczby $i$, 
    dla których przedział $[x_{i-1}, x_i]$ nie ma punktów wspólnych z żadnym z skontruowanych powyżej otoczeń punktów $x_l$, $l=1,\ldots,k$, a do zbioru $B$
    pozostałe przedziały powstające z podziału $P$ przedziału $[a,b]$. Wówczas
    \[
        U(f, P) - L(f, P) = \sum_{i=1}^{n}(M_i - m_i)\Delta x_i = \sum_{i \in A}(M_i - m_i)\Delta x_i + \sum_{i \in B}(M_i - m_i)\Delta x_i
    \]
    Ponadto
    \[
        \sum_{i \in A}(M_i - m_i)\Delta x_i \leqslant \frac{\varepsilon}{2(b-a)}\sum_{i \in A}\Delta x_i \leqslant \frac{\varepsilon}{2(b-a)}(b-a)=\frac{\varepsilon}{2}
    \]
    Suma długości podprzedziałów przedziału $[a,b]$ indeksowanych przez liczby ze zbioru $B$ jest nie większa niż
    \[
        (\delta + 2\delta_i+\delta)k < 4 \frac{\varepsilon}{8(M-m)k}k = \frac{\varepsilon}{2(M-m)}
    \]
    Dlatego
    \[
        \sum_{i \in B}(M_i - m_i)\Delta x_i \leqslant (M-m)\sum_{i \in B}\Delta x_i < (M-m)\frac{\varepsilon}{2(M-m)}=\frac{\varepsilon}{2}
    \]
    Dla podziału $P$ o średnicy mniejszej niż $\delta$ otrzymujemy zatem
    \[
        U(f, P) - L(f, P) = \sum_{i=1}^{n}(M_i - m_i)\Delta x_i < \varepsilon
    \]
    co kończy dowód.
\end{proof}

\begin{uwaga}
    Twierdzenie $\ref{theorem:78}$ można istotnie uogólnić. Mianowicie dowodzi się, że jeśli $f$ jest ograniczoną funkcją
    na przedziale $[a,b]$, to jest ona całkowalna w sensie Riemanna na tym przedziale wtedy i tylko wtedy, gdy
    jest ona ciągła prawie wszędzie ma $[a,b]$, to znaczy zbiór punktów nieciągłości funkcji $f$ ma miarę Lebesgue'a równą zeru.
    (por. [7], s. 270).
    Przykładów takich funkcji dostarcza następujące
\end{uwaga}

\theorem{Funkcja monotoniczna na przedziale $[a,b]$ jest na tym przedziale całkowalna w sensie Riemanna.}

\proof{}
Załóżmy, że $f$ jest funkcją niemalejącą. Niech będzie dane dowolne $\varepsilon > 0$. Weźmy podział $P$ przedziału $[a,b]$ na $n$ równych części o długości
$\frac{b-a}{n}$. Ponieważ $f$ jest niemalejącą zatem $M_i = f(x_i)$ oraz $m_i = f(x_{i-1})$ dla $i = 1, \ldots, n$. Mamy więc
\[
    U(f, P) - L(f, P) = \sum_{i=1}^{n}(f(x_i) - f(x_{i-1}))\frac{b-a}{n} = (f(b)-f(a))\frac{b-a}{n}
\]
Biorąc $n$ tak duże, aby $(f(b) - f(a))\frac{b-a}{n} < \varepsilon$ i stosując twierdzenie $\ref{theorem:76}$ otrzymujemy tezę.
W przypadku funkcji nierosnącej dowód jest analogiczny.

\begin{theorem}
    {Jeśli $f$ jest całkowalna w sensie Riemanna na przedziale $[a,b]$, $m \leqslant f(x) \leqslant M$ dla $x \in [a,b]$ oraz $\phi$ jest funkcją ciągłą
    na $[m, M]$, to funkcja złożona $h = \phi \circ f$ jest R-całkowalna na $[a,b]$.}
\end{theorem}

\begin{proof}
    Ustalmy $\varepsilon > 0$. Ponieważ funkcja $\phi$ jest jednostajnie ciągła na $[M,m]$, więc
    istnieje $\delta > 0$ taka, że $\delta < \varepsilon$ i $|\phi(s) - \phi(t)| < \varepsilon$, jeśli $|s-t| < \delta$.
    Ponieważ $f$ jest R-całkowalna na $[a,b]$, więc istnieje podział $P = \{x_0, \ldots, x_n\}$
    przedziału $[a,b]$ taki, że $U(f, P) - L(f, P) < \delta^2$. Niech
    \[
        M_i=\sup\limits_{x_{i-1}\leqslant x \leqslant x_i}f(x), \quad m_i=\inf\limits_{x_{i-1}\leqslant x \leqslant x_i}f(x),
    \]
    \[
        M_i^*=\sup\limits_{x_{i-1}\leqslant x \leqslant x_i}h(x), \quad m_i^*=\inf\limits_{x_{i-1}\leqslant x \leqslant x_i}h(x)
    \]
    dla $i = 1, \ldots, n$. Podzielmy zbiór $\{1, \ldots, n\}$ na dwa rozłącznę zbiory $A$ i $B$ w taki sposób, że
    $i \in A$, jeśli $M_i - m_i < \delta$ oraz $i \in B$ w przypadku przeciwnym. Wówczas wobec powyższego wyboru $\delta$ mamy
    $M_i^* - m_i^* < \varepsilon$ dla $i \in A$. Natomiast dla $i \in B$ mamy $M_i^* - m_i^* \leqslant 2K$, gdzie $K=\sup{\{|\phi(t)|:m \leqslant t \leqslant M\}}$.
    Stąd otrzymujemy
    \[
        \delta \sum_{i \in B}(x_i - x_{i-1}) \leqslant \sum_{i \in B}(M_i - m_i)(x_i - x_{i-1}) < \delta^2, \quad \text{zatem} \sum_{i \in B}(x_i - x_{i-1}) < \delta
    \]
    Mamy więc
    \[
        U(h, P) - L(h, P) = \sum_{i \in A}(M_i^* - m_i^*)(x_i - x_{i-1}) + \sum_{i \in B}(M_i^* - m_i^*)(x_i - x_{i-1})
    \]
    a zatem
    \[
        U(h, P) - L(h, P) \leqslant \varepsilon(a + b + 2K)
    \]
    Ponieważ $\varepsilon$ było dowolne, zatem na mocy twierdzenia $\ref{theorem:76}$ funkcja $h$ jest R-całkowalna.
\end{proof}

Następujące twierdzenie opisuję związek całki Riemanna z operacjami arytmetycznymi. 

\begin{theorem}
    {Jeśli funkcje $f$ i $g$ są R-całkowalne na przedziale $[a,b]$, to również R-całkowalne są
    funkcje $f+g$, $\lambda f$ ($\lambda$ jest dowolną stałą rzeczywistą) i $fg$ oraz prawdziwe są równości:}
\end{theorem}
\begin{eq}
    \begin{equation}
        \int_{a}^{b} (f+g) (x) dx = \int_{a}^{b} f (x) dx + \int_{a}^{b} g (x) dx,
    \end{equation}
\end{eq}
\begin{eq}
    \begin{equation}
        \int_{a}^{b} (\lambda f) (x) dx = \lambda \int_{a}^{b} f (x) dx
    \end{equation}
\end{eq}

\proof{} Jest jasne, że dla dowolnego $R \in \mathfrak{R}(f+g, P)$ mamy $R = R_f + R_g$, gdzie $R_f \in \mathfrak{R}(f, P)$, $R_g \in \mathfrak{R}(g, P)$.
Niech $I_1 = \int_{a}^{b}f(x)dx$, $I_2 = \int_{a}^{b}f(x)dx$ oraz $I = I_1 + I_2$. Mamy
\[
    \forall_{\varepsilon > 0} \exists_{\delta > 0} \forall_P \forall_{R_f \in \mathfrak{R}(f, P)} \delta(P) < \delta \implies |R_f - I_1| < \frac{\varepsilon}{2} \quad \text{oraz}
\]
\[
    \forall_{\varepsilon > 0} \exists_{\delta > 0} \forall_P \forall_{R_g \in \mathfrak{R}(g, P)} \delta(P) < \delta \implies |R_g - I_2| < \frac{\varepsilon}{2} \quad \text{Stąd}
\]
\[
    \forall_{\varepsilon > 0} \exists_{\delta > 0} \forall_P \forall_{R \in \mathfrak{R}(f+g, P)} \delta(P) < \delta \implies |R - I| \leqslant |R_f - I_1| + |R_g - I_2| < \varepsilon
\]
Wobec powyższego jest jasne, że funkcja $f + g$ jest R-całkowalna na przedziale $[a,b]$ oraz, że spełniony jest wzór $\ref{eq:40}$.
Dowód wzoru $\ref{eq:41}$ jest analogiczny.

\noindent
Dalej przyjmując $\phi(t) = t^2$ oraz stosując do $\phi$ poprzednie twierdzenie (\ref{theorem:81}) otrzymujemy
R-całkowalność funkcji $f^2$.

\noindent
R-całkowalność iloczynu funkcji $fg$ wynika z tożsamości
\[
    fg = \frac{1}{4}[{(f+g)}^2 - {(f-g)}^2].
\]

\begin{theorem} 
{
    (a) Jeśli funkcje $f$ i $g$ są R-całkowalne na przedziale $[a,b]$ oraz $f(x) \leqslant g(x)$ dla każdego $x \in [a,b]$, to
    \[
        \int_{a}^{b}f(x)dx \leqslant \int_{a}^{b}g(x)dx
    \]
    (b) Jeśli funkcja $f$ jest R-całkowalna na przedziale $[a,b]$, to funkcja $|f|$ jest również
    R-całkowalna na tym przedziale oraz:
    \[
        \left\vert\int_{a}^{b}f(x)dx\right\vert \leqslant \int_{a}^{b}|f(x)|dx
    \]
}
\end{theorem}
\begin{proof}
    (a) Jeśli $m \leqslant f(x) \leqslant M$ dla $x \in [a,b]$, to
    \[
        m(b-a) \leqslant \int_{a}^{b}f(x)dx \leqslant M(b-a)
    \]
    Stąd, jeśli $f(x) \geqslant 0$ dla $x \in [a,b]$, to $\int_{a}^{b}f(x)dx \geqslant 0$.
    Wobec tego nierówność $f(x) \leqslant g(x)$ dla $x \in [a,b]$ implikuje
    \[
        \int_{a}^{b}f(x)dx \leqslant \int_{a}^{b}g(x)dx
    \]
    (b) Biorąc $\phi(t) = |t|$ w Twierdzeniu $\ref{theorem:81}$ otrzymujemy całkowalność funkcji $|f|$.
    Ponieważ $-|f(x)| \leqslant f(x) \leqslant |f(x)|$ dla $x \in [a,b]$, zatem na mocy (a) otrzymujemy
    \[
        \left\vert\int_{a}^{b}f(x)dx\right\vert \leqslant \int_{a}^{b}|f(x)|dx
    \]
\end{proof}
\begin{uwaga}
    (a) Punkt (a) Twierdzenia $\ref{theorem:83}$ można udowodnić bezpośrednio w oparciu o definicję
    całki Riemanna (Def. $\ref{def:69}$) oraz Wniosek 3 (b). \\
    \\
    \noindent
    (b) Twierdzenie odwrotne do Twierdzenia $\ref{theorem:83}$ (b) nie jest prawdziwe, to znaczy z R-całkowalności
    $|f|$ nie wynika R-całkowalność funkcji $f$. Dla przykładu niech
    \[ f(x) = \begin{cases}
        1 & \text{dla } x \in \mathbb{Q} \cap [a,b], \\
        -1 & \text{dla } x \in  (\mathbb{R \setminus Q}) \cap [a,b].
    \end{cases}
    \]
    Oczywiście funkcja $|f|$ jest R-całkowalna na przedziale $[a,b]$ oraz $\int_{a}^{b}|f(x)|dx = b-a$.
    Z kolei $underline{\int_{a}^{b}f(x)dx} = -(b-a)$ oraz $\overline{\int_{a}^{b}f(x)dx} = b-a$, a zatem
    wobec Twierdzenia $\ref{theorem:77}$ funkcja $f$ nie jest R-całkowalna na przedziale $[a,b]$.
\end{uwaga}

\begin{theorem}
    {Jeśli dwie funkcje $f$ i $g$ są równe na przedziale $[a,b]$ z wyjątkiem skończonego zbioru punktów ${\{x_1, \ldots, x_k\}}$
    i jedna z nich, na przykład $g$ jest R-całkowalna na tym przedziale, to druga też jest na nim R-całkowalna i zachodzi równość
    \[
        \int_{a}^{b}f(x)dx = \int_{a}^{b}g(x)dx.
    \]}
\end{theorem}
\begin{proof}
    Ponieważ $f = g + (f-g)$, więc wystarczy udowodnić, że funkcja $\phi = f - g$ jest R-całkowalna na $[a,b]$ i $\int_{a}^{b}\phi(x)dx = 0$.
    Oznaczmy $N = \max\{|\phi(x_1), \ldots, \phi(x_k)\}$. Niech $P$ będzie podziałem przedziału $[a,b]$ o średnicy $\delta$.
    Funkcja $\phi$ na co najwyżej $2k$ przedziałach podziału $P$ nie jest tożsamościowo równa zeru.
    Dlatego mamy $U(\phi, P) \leqslant 2Nk\delta$ i $L(\phi, P) = 0$, zatem $U(\phi, P) - L(\phi, P) \leqslant 2Nk\delta$. 
    Biorąc $\phi$ odpowiednio małe możemy uczynić różnicę $U(\phi, P) - L(\phi, P)$ dowolnie małą. To oznacza, że funkcja $\phi$ jest
    R-całkowalna. Ponadto jasne jest, że $\int_{a}^{b}\phi(x)dx = 0$.
\end{proof}

\begin{wniosek}
{Niech funkcja $f$ będzie określona i ograniczona na przedziale otwartym $(a,b)$. Jeśli po nadaniu jej pewnych wartości
$f(a)$ i $f(b)$ stanie się ona R-całkowalna na przedziale domkniętym $[a,b]$ --- to taką pozostanie --- gdy liczby $f(a)$ i $f(b)$ zmienimy
w sposób dowolny. Wartość całki nie ulegnie przy tym zmianie.}
\end{wniosek}

Następujący lemat pozwala przy przybliżaniu całki Riemanna sumami całkowymi ograniczyć się tylko do podziałów zawierających z góry ustalony punkt.

\begin{lemat}
{
    Niech $c \in [a,b]$ i niech $\Pi^*$ oznacza zbiór wszystkich podziałów przedziału $[a,b]$ spełniających warunek:
    \[
        P = \{x_0, \ldots, x_n\} \in \Pi^* \text{ wtedy i tylko wtedy $x_j = c$ dla pewnego $j$.} 
    \]
    Wówczas dla dowolnej funkcji $f$, ograniczonej na $[a,b]$ zachodzą równości:
    \[
        \sup_{P \in \Pi^*}L(f, P) = \sup_{P}L(f, P), \quad  \inf_{P \in \Pi^*}U(f, P) = \inf_{P}U(f, P)
    \]
}
\end{lemat}
\begin{proof}
    Ponieważ $\Pi^*$ jest podzbiorem zbioru wszystkich podziałów przedziału $[a,b]$, więc
    \begin{equation}\label{eq:42}
        \sup_{P \in \Pi^*}L(f, P) \leqslant \sup_{P}L(f, P), \quad  \inf_{P \in \Pi^*}U(f, P) \geqslant \inf_{P}U(f, P)
    \end{equation}
    Zauważmy, że dla dowolnego podziału $P$ przedziału $[a,b]$ istnieje podział od niego drobniejszy
    $P^* \in \Pi^*$. Istotnie, jeśli $P \in \Pi^*$, to przyjmujemy $P^* = P$. Jeśli natomiast $P \notin \Pi^*$,
    to przez dołączenie punktu $c$ do układu punktów wyznaczających $P$ otrzymujemy podział $P^*$ o żądanych własnościach.
    Mamy więc
    \[
        L(f, P) \leqslant L(f, P^*), \quad U(f, P) \geqslant U(f, P^*),
    \]
    skąd otrzymujemy
    \[
        L(f, P) \leqslant \sup_{P^* \in \Pi^*}L(f, P^*), \quad U(f, P) \geqslant \inf_{P^* \in \Pi^*}U(f, P^*).
    \]
    Wobec dowolności podziału $P$ mamy
    \[
        \sup_{P}L(f, P) \leqslant \sup_{P^* \in \Pi^*}L(f, P), \quad  \inf_{P}U(f, P) \geqslant \inf_{P^* \in \Pi^*}U(f, P)
    \]
    Z powyższych nierówności i z (\ref{eq:42}) otrzymujemy tezę.
\end{proof}

\begin{theorem}
{
    Niech $a < c < b$. Funkcja $f$ jest R-całkowalna na przedziale $[a,b]$ wtedy i tylko wtedy, gdy
    jest ona R-całkowalna na przedziałach $[a,c]$ i $[c,b]$. Zachodzi przy tym równość
    \[
        \int_{a}^{b}f(x)dx = \int_{a}^{c}f(x)dx + \int_{c}^{b}f(x)dx
    \]
    (addytywność całki ze względu na przedział)
}
\end{theorem}

\begin{proof}
    Załóżmy, że funkcja $f$ jest R-całkowalna na przedziale $[a,b]$. Na mocy powyższego lematu
    możemy ograniczyć się do podziałów przedziału $[a,b]$ zawierających punkt $c$. 
    Jeśli $P$ jest takim podziałem, to wówczas $P = P_1 \cup P_2$, gdzie $P_1$ jest podziałem
    przedziału $[a,c]$, a $P_2$ --- podziałem przedziału $[c,b]$ oraz mamy
    \[
        U(f, P) = U(f, P_1) + U(f, P_2), \quad L(f, P) = L(f, P_1) + L(f, P_2).
    \]
    Niech będzie dane dowolne $\varepsilon > 0$ i niech
    \[
        U(f, P) - L(f, P) < \frac{\varepsilon}{2}.
    \]
    Stąd $U(f, P_1) - L(f, P_1) < \frac{\varepsilon}{2}$ i $U(f, P_2) - L(f, P_2) < \frac{\varepsilon}{2}$. 
    Funkcja $f$ jest więc całkowalna na przedziałach $[a,c]$ i $[c, b]$ oraz zachodzą nierówności
    \[
        U(f, P_1) < \int_{a}^{c}f(x)dx + \frac{\varepsilon}{2}, \quad \int_{a}^{c}f(x)dx < L(f, P_1) + \frac{\varepsilon}{2},
    \]
    \[
        U(f, P_2) < \int_{c}^{b}f(x)dx + \frac{\varepsilon}{2}, \quad \int_{c}^{b}f(x)dx < L(f, P_2) + \frac{\varepsilon}{2},
    \]
    Wobec powyższego otrzymujemy $U(L, P) < \int_{a}^{c}f(x)dx + \int_{c}^{b}f(x)dx + \varepsilon$
    i w konsekwencji $\int_{a}^{b}f(x)dx < \int_{a}^{c}f(x)dx + \int_{c}^{b}f(x)dx + \varepsilon$.
    Ponieważ $\varepsilon > 0$ było dowolne, zatem
    \begin{equation}\label{eq:43}
        \int_{a}^{b}f(x)dx \leqslant \int_{a}^{c}f(x)dx + \int_{c}^{b}f(x)dx.
    \end{equation}
    Analogicznie $\int_{a}^{c}f(x) + \int_{c}^{b}f(x)dx < L(f, P) + \varepsilon$, skąd
    \begin{equation}\label{eq:44}
        \int_{a}^{c}f(x)dx + \int_{c}^{b}f(x)dx \leqslant \int_{a}^{b}f(x)dx,
    \end{equation}
    bowiem $\varepsilon > 0$ jest dowolne. Z nierówności (\ref{eq:43}) i (\ref{eq:44}) otrzymujemy żądaną równość.
    Uzasadnienie implikacji odwrotnej jest analogiczne.
\end{proof}

Rozszerzymy teraz zasięg Definicji $\ref{def:69}$.

\begin{defn}
    W przypadku gdy $b < a$ lub $b = a$, to całkę Riemanna z funkcji $f$ określamy wzorami
    \[
        \int_{a}^{b}f(x)dx = -\int_{b}^{a}f(x)dx \quad \text{lub odpowiednio} \quad \int_{a}^{b}f(x)dx = 0.
    \]
    W całce $\int_{a}^{b}f(x)dx$ liczbę $a$ nazywamy dolną granicą całkowania, liczbę $b$ -- górną granicą całkowania,
    bez względu na to, czy $b \geqslant a$, czy też $b < a$. 
\end{defn}

\begin{wniosek}
{
(a) Niech $a,b,c \in \mathbb{R}$ i niech $f$ będzie funkcją R-całkowalna na najwiekszym z przedziałów
domkniętych o końcach we wskazanych punktach. Wówczas obcięcie funkcji $f$ do każdego z dwóch pozostałych przedziałów domkniętych jest
funkcja R-całkowalną na odpowiednim przedziale oraz zachodzi równość
\begin{equation}\label{eq:45}
    \int_{a}^{b}f(x)dx + \int_{b}^{c}f(x)dx + \int_{c}^{a}f(x)dx = 0.
\end{equation}
}
\end{wniosek}
\begin{proof}
    Wobec symetrii równość (\ref{eq:45}) względem $a, b, c$ możemy bez straty ogólności założyć, że
    $a = \min{\{a,b,c\}}$. Jeśli $\max{\{a,b,c\} = c}$ oraz $a < b < c$, to na mocy
    Twierdzenia $\ref{theorem:85}$ mamy
    \[
        \int_{a}^{b}f(x)dx + \int_{c}^{b}f(x)dx - \int_{a}^{c}f(x)dx = 0,
    \]
    zatem wobec Definicji $\ref{def:74}$ otrzymujemy równość (\ref{eq:45}). \\
    Jeśli $\max{\{a,b,c\} = b}$ oraz $a < c < b$, to ponownie na mocy Twierdzenia
    $\ref{theorem:85}$ mamy
    \[
        \int_{a}^{c}f(x)dx + \int_{c}^{b}f(x)dx - \int_{a}^{b}f(x)dx = 0,
    \]
    stąd wobec Definicji $\ref{def:74}$ wynika równość (\ref{eq:45}). \\
    W końcu, jeśli jakiekolwiek dwa z punktów $a,b,c$ lub wszystkie trzy pokrywają się,
    to (\ref{eq:45}) jest bezpośrednią konsekwencją Definicji $\ref{def:74}$.
\end{proof}

\noindent
\textit
{
(b) Jeśli funkcja $f$ jest R-całkowalna na przedziale $[a,b]$ i $a \leqslant c < d \leqslant d$, to jest
ona również R-całkowalna na przedziale $[c,d]$.
}

\begin{defn}
    Niech każdej uporządkowanej parze $(\alpha, \beta)$ punktów $\alpha$, $\beta$ przedziału
    $[a,b]$ odpowiada dokładnie jedna liczba $I(\alpha, \beta)$, przy czym dla dowolnej trójki punktów
    $\alpha, \beta, \gamma \in [a,b]$ zachodzi równość
    \[
        I(\alpha, \gamma) = I(\alpha, \beta) + I(\beta, \gamma).
    \]
    Wówczas funkcja $I(\alpha, \beta)$ nazywa się addytywną funkcją przedziału zorientowanego
    (dla $\alpha = \gamma$ wobec powyższej równości otrzymujemy $I(\alpha, \beta) = -I(\beta, \alpha)$), 
    określoną na odcinkach zawartych w przedziale $[a,b]$. 
\end{defn}

\begin{wniosek}
{
    Jeśli funckja $f$ jest R-całkowalna na przedziale $[a,b]$ oraz $\alpha, \beta, \gamma \in [a,b]$, to
    kładąc $I(\alpha, \beta) = \int_{\alpha}^{\beta}f(x)dx$, na mocy równości $\ref{eq:45}$ otrzymujemy
    \[
        \int_{\alpha}^{\gamma}f(x)dx = \int_{\alpha}^{\beta}f(x)dx + \int_{\beta}^{\gamma}f(x)dx,
        \quad \text{czyli} \quad I(\alpha, \gamma) = I(\alpha, \beta) + I(\beta, \gamma),
    \]
    czyli całka Riemanna jest adytywną funkcją przedziału zorientowanego.
}
\end{wniosek}

Udowodnimy teraz ważne twierdzenie o funkcji górnej granicy całkowania.

\begin{theorem}
{
    Niech $f$ będzie funkcja R-całkowalną na przedziale $[a,b]$. Dla dowolnego punktu
    $x \in [a,b]$ określamy
    \[
        F(x) = \int_{a}^{x}f(t)dt.
    \]
    Wówczas funkcja $F$ jest ciągła na przedziale $[a,b]$. Ponadto, jeśli funkcja $f$ jest ciągła w
    punkcie $x_0 \in [a,b]$, to funkcja $F$ jest różniczkowalna w tym punkcie oraz
    $F'(x_0) = f(x_0)$.
}
\end{theorem}

\begin{proof}
    Niech $M$ będzie takie, że $|f(t)| \leqslant M$ dla $t \in [a,b]$. 
    Wówczas, jeśli $a \leqslant x \leqslant y \leqslant b$, to
    \[
        |F(y) - F(x)| = \left|\int_{x}^{y}f(t)dt\right| \leqslant M(y - x).
    \]
    Stąd wynika natychmiast, że dla dowolnego $\varepsilon > 0$ mamy
    $|F(y) - F(x)|< \varepsilon$, jeśli tylko $|y-x| < \frac{\varepsilon}{M}$, 
    a zatem funkcja F jest ciągła. \\
    Załóżmy, że $f$ jest ciągła w punkcie $x_0$. Dla danego $\varepsilon > 0$ wybierzmy 
    $\delta > 0$ tak, aby $|f(t)-f(x_0)| < \varepsilon$ jeśli tylko $|t-x_0| < \delta$ i $a \leqslant t \leqslant b$.
    Wówczas dla $s,t \in (x_0 - \delta, x_0 + \delta)$, $s,t \in [a,b]$, $s \neq t$ mamy
    \[
        \left|\frac{F(t)-F(s)}{t-s} - f(x_0)\right| = \left|\frac{1}{t-s}\int_{s}^{t}(f(u)-f(x_0))du \right| < \varepsilon,
    \]
    Stąd wynika, że $F'(x_0) = f(x_0)$, co kończy dowód.
\end{proof}

Oznaczmy $F(x) = I(a, x)$ dla $x \in [a,b]$, gdzie $I$ oznacza addytywną funkcję przedziału zorientowanego. Mamy
\[
    I(\alpha, \beta) = I(a, \beta) - I(a, \alpha) = F(\beta) - F(\alpha)
\]
dla każdej uporządkowanej pary punktów $(\alpha, \beta)$ z przedzialu $[a,b]$.
W ten sposób każda addytywna funkcja przedziału zorientowanego ma postać
\begin{equation}\label{eq:46}
    I(\alpha, \beta) = F(\beta) - F(\alpha),
\end{equation}
gdzie $x \mapsto F(x)$ jest funkcją określoną na przedziale $[a,b]$. 
Można łatwo sprawdzić, że jest również na odwrót, to znaczy, że z dowolnej funkcji $x \mapsto F(x)$ określonej na przedziale $[a,b]$
można przy pomocy (\ref{eq:46}) otrzymać addytywną funkcję przedziału zorientowanego.

\begin{wniosek}
    Jeśli $f$ jest funkcją R-całkowalną na przedziale $[a,b]$, to na mocy (\ref{eq:46})
    funkcja $F(x) = \int_{a}^{x}f(t)dt$ generuje addytywną funkcję 
    \[
        I(\alpha, \beta) = \int_{\alpha}^{\beta}f(t)dt.
    \]
\end{wniosek}

\subsection{Całka nieoznaczona}

\begin{defn}
    Niech $f$ będzie funkcją określoną na pewnym przedziale $I$. Każdą funkcję $F$ różniczkowalną na tym przedziale
    i spełniającą w każdym punkcie $x \in I$ równość
    \[
        F'(x) = f(x)
    \]
    nazywamy funkcją pierwotną funkcji $f$. Funkcję pierwotną nazywamy również całką
    nieoznaczoną danej funkcji i oznaczamy symbolem $\int f(x)dx$ (symbol ten należy również
    rozumieć jako oznaczenie dowolnej funkcji pierwotnej funkcji $f$ na tym przedziale).
    W symbolu tym znak $f$ nazywa się znakiem całki nieoznaczonej, $f$ --- funkcją podcałkową, a $f(x)dx$ ---
    wyrażeniem podcałkowym.
\end{defn}

\begin{uwaga}
    Jeśli $F$ jest funkcją pierwotną funkcji $f$, to suma $F + c$, gdzie $c$ jest dowolną stałą, jest
    również funkcją pierwotną funkcji $f$, bowiem $(F+c)' = F' = f$. \\
    Na odwrót, dwie dowolne fukcje pierwotne $F$ i $G$ tej samej funkcji $f$ róźnią się o stałą,
    bowiem $(F-G)'= f-f = 0$. \\
    Jeśli $F$ jest więc konkretną funkcją pierwotną funkcji $f$ na przedziale $I$, to na tym przedziale
    \[
        \int f(x)dx = F(x) + c,
    \]
    to znaczy dowolna inna funkcja pierwotna funkcji $f$ może być otrzymana z danej funkcji $F$ przez dodanie stałej.

\end{uwaga}

Bezpośrednio z Twierdzenia $\ref{theorem:86}$ otrzymujemy następujący

\begin{wniosek}
{
    Każda funkcja ciągła $f$ na przedziale $[a,b]$ ma na nim funkcję pierwotną.
}
\end{wniosek}

Dowód wniosku $\ref{wniosek:22}$ można uzyskać bez pojęcia całki Riemanna. Jest on jednak
dość długi. \\
Istnieją również funkcję nieciągłe, które posiadają funkcje pierwotne.

\begin{ex}
Niech
\[
f(x) =
\begin{cases}
    \begin{aligned}
        &2x\sin\frac{1}{x}-\cos{\frac{1}{x}} & &\text{dla } x \neq 0 \\
        &0                                   & &\text{dla } x = 0.
    \end{aligned}
\end{cases}
\]
Ponieważ nie istnieje granica funkcji $f$ w zerze, zatem $f$ nie jest funkcją ciągłą. Można łatwo sprawdzić, że funkcja 
\[
F(x) =
\begin{cases}
    \begin{aligned}
        &x^2\sin{\frac{1}{x}} & &\text{dla } x \neq 0 \\
        &0                                   & &\text{dla } x = 0,
    \end{aligned}
\end{cases}
\]
jest funkcją pierwotną funkcji $f$ (por. [7], s. 90-91).
\end{ex}

Podamy teraz przykłady funkcji całkowalnej w sensie Riemanna, która nie posiada funkcji pierwotnej.

\begin{ex}
    Niech $f : (1,3) \mapsto \mathbb{R}$ będzie określona wzorem $f(x) = [x]$.
    Jest jasne, że dla $f(x) = F'(x)$ dla $x \in (1,2) \cup (2,3)$, gdzie
    \[
    F(x) =
        \begin{cases}
            \begin{aligned}
                &x + c_1 & &\text{dla } x \in (1,2) \\
                &2x + c_2                                 & &\text{dla } x \in (2,3),
            \end{aligned}
        \end{cases}
    \]
    gdzie, $c_1, c_2 \in \mathbb{R}$. Funkcja pierwotna funkcji $f$ na przedziale $(1,3)$ 
    (z dokładnościa do stałej musiałaby mieć postać)
    \[
    F(x) =
        \begin{cases}
            \begin{aligned}
                &x + c_1 & &\text{dla } x \in (1,2] \\
                &2x + c_1 - 2 & &\text{dla } x \in (2,3),
            \end{aligned}
        \end{cases}
    \]
    gdzie $c_1 \in \mathbb{R}$. Istotnie, aby funkcja $F$ była ciągła dla $x = 2$, to 
    $\lim\limits_{x\to2^-}F(x) = \lim\limits_{x\to2^+}F(x)$, czyli $2 + c_1 = 4 + c_2$, zatem $c_2 = c_1 - 2$. 
    Można łatwo sprawdzić, że $F_{-}'(2) = 1$ oraz $F_{+}'(2) = 2$, czyli $F$ nie jest
    funkcją pierwotną funkcji $f$ na przedziale $(1,3)$.
\end{ex}

\newpage
\textbf{Tablica 1.} Całki nieoznaczona podstawowych funkcji elementarnych

\noindent
\begin{tabular}{|c|c|c|}
    \hline
    $f(x)$ & $F(x)$ & Ograniczenia ze względu na argument $x \in \mathbb{R}$ \\
    \hline
    0 & $c = $ const. & \\
    \hline
    $a = $ const. & $ax + c$ &  \\
    \hline
    $x^p$ & $\frac{1}{p+1}x^{p+1} + c$ & \makecell{$p \neq -1, x > 0$ $(p \in \mathbb{R})$, \\ $x \neq 0$ $(p \in \mathbb{Z})$, \\ $x \in \mathbb{R}$ $(p \in \mathbb{N})$}\\
    \hline
    $\frac{1}{x}$ & $\ln{|x|} + c$ & $x \neq 0$ \\
    \hline
    $a^x$ & $\frac{a^x}{\ln{a}} + c$ & $x \in \mathbb{R}$ $(a >0, a \neq 1)$ \\
    \hline
    $e^x$ & $e^x + c$ &  \\
    \hline
    $\sin{x}$ & $-\cos{x} + c$ & \\
    \hline
    $\cos{x}$ & $\sin{x} + c$ & \\
    \hline
    $\frac{1}{\cos^2{x}}$ & $\tg{x} + c$ & $x \neq \frac{\pi}{2} + k\pi$, $k \in \mathbb{Z}$ \\
    \hline
    $\frac{1}{\sin^2{x}}$ & $-\ctg{x} + c $& $ x \neq k\pi$, $k \in \mathbb{Z}$ \\
    \hline
    $\frac{1}{1+x^2}$ & \makecell{$\arctg{x} + c$ \\ $\arcctg{x} + \hat{c}$} & \\
    \hline
    $\frac{1}{\sqrt{1 - x^2}}$ & \makecell{$\arcsin{x} + c$ \\ $-\arccos{x} + \hat{c}$} & $|x| < 1$ \\
    \hline
\end{tabular}

\begin{uwaga}
    (a) Wzory zawarte w Tablicy 2 otrzymujemy przez bezpośrednie różniczkowanie funkcji $F(x)$ (zob. Tablica 1).

    \noindent
    (b) Jeżeli zakres argumentów, dla których spełniona jest równość $F(x) = f'(x)$ nie jest przedziałem
    (skończonym lub nieskończonym), to nie można twierdzić, że wyrażenie $F(x) + c$ obejmuje wszystkie funkcje pierwotne funkcji $f$
    w tym zakresie argumentów. Dla przykładu funkcja

    \[
        G(x) =
            \begin{cases}
                \begin{aligned}
                    &\ln(-x) & &\text{dla } x < 0 \\
                    &\ln{x} & &\text{dla } x > 0,
                \end{aligned}
            \end{cases}
    \]
    jest funkcją pierwotną funkcji $x \mapsto \frac{1}{x}$ $(x \neq 0)$, mimo, że nie podpada pod
    wzór $\ln|x| + c$. 
\end{uwaga}

Następujące twierdzenie podaje reguły obliczani całek nieoznaczonych.

\begin{theorem}
{   (a) Jeśli istnieją całki nieoznaczone funkcji $u, v : P \mapsto \mathbb{R}$, gdzie $P$ jest przedziałem
    oraz $\alpha, \beta \in \mathbb{R}$, to istnieje całka nieoznnaczona funkcji
    $\alpha u + \beta v$ oraz zachodzi wzór
    \begin{equation}\label{eq:47}
        \int (\alpha u(x) + \beta v(x))dx = \alpha \int u(x)dx + \beta \int v(x)dx + c \text{ dla } x \in P.
    \end{equation}
    (b) Przy założeniach punktu (a) oraz przy założeniu, że funkcje $u, v$ są różniczkowalne oraz jedna z całek występujących
    w poniższym wzorze istnieje, prawdziwy jest następujący wzór zwany wzorem na całkowanie przez części:
    \begin{equation}\label{eq:48}
        \int u(x)v'(x)dx = u(x)v(x) - \int u'(x)v(x)dx + c \text{ dla } x \in P.
    \end{equation}
    (c) Jeśli na przedziale $I$, $\int f(x)dx = F(x) + c$ oraz $\phi : P \mapsto I$ jest odwzorowaniem klasy $C^1$, to
    \begin{equation}\label{eq:49}
        \int f(\phi(t))\phi'(t)dt = F(\phi(t)) + c \text{ dla } t \in P.
    \end{equation}
}
\end{theorem}

\begin{proof}
    (a) Wzór (\ref{eq:47}) sprawdza się bezpośrednio przez różniczkowanie lewej i prawej strony z wykorzystaniem liniowości różniczkowania (zob. Tw. 54 (a), (b)).
    
    \noindent
    (b) Załóżmy, że $\int u(x)v'(x)dx = \Phi(x)$ dla $x \in P$. Ponieważ $(u(x)v(x))'=u'(x)v(x) + u(x)v'(x)$,
    $(u(x)v(x)-\Phi(x))' = u'(x)v(x)$ dla $x \in P$, a więc $uv - \Phi$ jest funkcją pierwotną funkcji $u'v$ na przedziale $P$.
    Ponadto mamy
    \[
        \int u(x)v'(x)dx = \int (u(x)v(x))'dx - \int u'(x)v(x)dx, \text{ a więc}
    \]
    \[
        \int u(x)v'(x)dx = u(x)v(x) - \int u'(x)v(x)dx + c.
    \]

    \noindent
    (c) Wzór (\ref{eq:49}) jest bezpośrednią konsekwencją reguły różniczkowania funkcji złożonej
    (zob. Tw. 55).

\end{proof}

\begin{uwaga}
    (a) Wzór (\ref{eq:49}) pokazuje, że chcąc uzyskać funkcję pierwotną funkcji $t \mapsto f(\phi(t))\phi'(t)$ można postąpić w następujacy sposób:
    \[
        \int f(\phi(t))\phi'(t)dt = \int f(x)dx = F(x) + c = F(\phi(t)) + c,
    \]
    to znaczy najpierw dokonać zmiany $\phi(t) = x$ i przejść do nowej zmiennej $x$, a następnie
    przejść do poprzedniej zmiennej podstawiając $x = \phi(t)$. Wzór ten nazywa się
    wzorem na całkowanie przez podstawienie.

    \noindent
    (b) Jeśli $\phi : P \mapsto I$ jest bijekcją, to aby obliczyć całkę nieoznaczoną $\int f(x)dx$ można obliczyć całke nieoznaczoną
    $\int f(\phi(t))\phi'(t)dt$, a następnie dokonać podstawienia $t = \phi^{-1}(x)$.

    \noindent
    (c) W szczególności wzór (\ref{eq:48}) jest prawdziwy jeśli $u,v \in C^1$.
\end{uwaga}

\begin{ex}
    (a) Obliczmy $\int \arcsin x dx$. Najpierw zastosujmy wzór (\ref{eq:48}) przyjmując $u(x) = \arcsin x$ i $v'(x) = 1$.
    Wówczas $u'(x) = \frac{1}{\sqrt{1 - x^2}}$ (dla $x \in (-1, 1)$) i $v(x) = x$ oraz
    $\int \arcsin x dx = x \arcsin x - \int \frac{x}{\sqrt{1 - x^2}}dx$. Aby
    obliczyć tę ostatnią całkę zastosujmy podstawienie $\sqrt{1 - x^2} = t$. Mamy wówczas
    $\frac{-xdx}{\sqrt{1-x^2}}=dt$ i stąd $\int \frac{x}{\sqrt{1 - x^2}}dx = -\int dt = -t + c = -\sqrt{1 - x^2} + c$, a zatem
    \[
        \int \arcsin x dx = x \arcsin x + \sqrt{1 - x^2} + c.
    \]

    \noindent
    (b) Obliczmy $\int \tg t dt$. Niech np. $P = (-\frac{\pi}{2}, \frac{\pi}{2})$, $I = (0,1]$, $\phi : P \mapsto I$
    będzie określone wzorem $\phi(t) = cost$, natomiast $f(x) = \frac{1}{x}$ dla $x \in (0, 1]$. Mamy
    \[
        \int \tg t dt = \int \frac{1}{\cos t}\sin t dt = -\int \frac{1}{x}dx = -\ln x + c = \ln \cos t + c.
    \]
    W ogólności podstawiamy $\cos t = x$. Wówczas $-\sin t dt = dx$ oraz
    \[
        \int \tg t dt = -\int \frac{dx}{x} = -\ln |x| + c = -\ln |\cos t| + c.
    \]
    Wobec Uwagi $\ref{uwaga:41}$ (b) powyższy wzór nie objemuje wszystkich funkcji pierwotnych funkcji
    $t \mapsto \tg t$, gdzie $t \neq \frac{\pi}{2} + k\pi$, $k \in \mathbb{Z}$.

    \noindent
    (c) Obliczmy $\sqrt{1 - x^2}$. Niech $I = (-1, 1)$, $P = (-\frac{\pi}{2}, \frac{\pi}{2})$,
    $f(x) = \sqrt{1 - x^2}$ dla $x \in I$ oraz $\phi(t) = sint$ dla $t \in P$. Oczywiście $\phi : P \mapsto I$
    jest bijekcją oraz $\phi^{-1}(x) = \arcsin{x}$ dla $x \in I$. Mamy
    \begin{equation*}
        \begin{split}
            \int \sqrt{1-x^2}dx &= \int \sqrt{1-\sin^2{t}}\cos tdt  = \int \cos^{2}{t}dt
            \int \frac{dt}{2} + \frac{1}{2}\int \cos{2t}dt \\
            &= \frac{1}{2}t + \frac{1}{4}\sin{2t} + c = \frac{1}{2}\arcsin{x} + \frac{1}{4}\sin(2\arcsin{x}) + c \\
            &= \frac{1}{2}\arcsin x + \frac{1}{2} \sin(\arcsin x)\cos( \arcsin x) + c \\
            &= \frac{1}{2}\arcsin{x} + \frac{1}{2}x\sqrt{1 - \sin^{2} (\arcsin x)} + c \\
            &= \frac{1}{2}\arcsin{x} + \frac{1}{2}x\sqrt{1 - x^2} + c.
        \end{split}
    \end{equation*}
  
\end{ex}

\begin{uwaga}
    Funkcjami elementarnymi nazywamy funkcje potęgowe, wykładnicze, trygonometryczne, funkcje odwrotne do nich, ich superpozycje oraz funkcje
    powstałe przez wykonanie skończonej ilości działań na nich: dodawanie, odejmowanie, mnożenie,
    dzielenie oraz składanie. 
    Podstawowe metody całkowania funkcji elementarnych oraz pewne tzw. wzory rekurencyjne można znaleźć np.
    w książce [5], s. 260-273. 
\end{uwaga}

\subsection{Rachunek całek (oznaczonych) w sensie Riemanna}
Udowodnimy wpierw następujące twierdzenie nazywane podstawowym twierdzeniem rachunku całkowego
lub wzorem Newtona-Leibniza.

\begin{theorem}
{
    Jeśli funkcja $f$ jest R-całkowalna na przedziale $[a,b]$ oraz $F$ jest funkcją pierwotną funkcji $f$ na tym przedziale, to
    \begin{equation*}
        \int_{a}^{b}f(x)dx = F(b) - F(a).
    \end{equation*}
}
\end{theorem}

\begin{proof}
    Wybierzmy dowolne $\varepsilon > 0$. Istnieje wówczas taki podział $P$ przedziału $[a,b]$, że
    $U(f, P) - L(f, P) < \varepsilon$. Wówczas
    \begin{equation*}
        U(f, P) < \varepsilon + \int_{a}^{b}f(x)dx, \quad \int_{a}^{b}f(x)dx < \varepsilon + L(f, P).
    \end{equation*}
    Na mocy Twierdzenie Lagrange'a o Wartości Średniej istnieją punkty $\xi_{j}$ ($j = 1,\ldots, n$) takie, że
    $x_{j-1} < \xi_{j} < x_{j}$ ($P = \{ x_0, x_1, \ldots, x_n\}$) oraz $F(x_j) - F(x_{j-1}) =
    f(\xi_j)(x_j - x_{j-1})$, a zatem
    \begin{equation*}
        \begin{split}
            F(b) - F(a) &= \sum_{j=1}^{n}(F(x_j) - F(x_{j-1})) \\
                        &= \sum_{j=1}^{n}f(\xi_j)(x_j - x_{j-1}) \leqslant U(f, P) < \varepsilon + \int_{a}^{b}f(x)dx, \\
            F(b) - F(a) &\geqslant L(f, P) > -\varepsilon + \int_{a}^{b}f(x)dx.            
        \end{split}
    \end{equation*}
    Stąd
    \begin{equation*}
        \left|F(b) - F(a) - \int_{a}^{b}f(x)dx \right| < \varepsilon
    \end{equation*}
    i wobec dowolności $\varepsilon > 0$ otrzymujemy żądany wzór.
\end{proof}

\begin{ex}
    Istnieją funkcje, które nie są całkowalne w sensie Riemanna, ale posiadają funkcje pierwotne. Niech
    \begin{equation*}
        f(x) = 
            \begin{cases}
                \begin{aligned}
                    &2x\cos \frac{\pi}{x^2} - \frac{2\pi}{x}\sin \frac{\pi}{x^2} &\text{ dla } x \neq 0&, \\
                    &0 &\text{ dla } x = 0&
                \end{aligned}
            \end{cases}
    \end{equation*}
    oraz
    \begin{equation*}
        F(x) = 
            \begin{cases}
                \begin{aligned}
                    &x^2 \cos \frac{\pi}{x^2} &\text{ dla } x \neq 0&, \\
                    &0 &\text{ dla } x = 0&
                \end{aligned}
            \end{cases}
    \end{equation*}
    Można łatwo sprawdzić, że $F'(x) = f(x)$ dla każdego $x \in [0,1]$. Funkcja $f$ nie jest jednak
    R-całkowalna na przedziale $[0,1]$, ponieważ nie jest na nim ograniczona. Niech bowiem
    $x_k = \frac{1}{\sqrt{\frac{1}{2}+2k}}$, $k \in \mathbb{N}$. Mamy
    \begin{equation*}
        \begin{split}
        \lim\limits_{k \to \infty}\frac{2\pi}{x_k} \sin \frac{\pi}{x^2_k} = \lim\limits_{k \to \infty}\frac{2\pi}{\frac{1}{\sqrt{\frac{1}{2}+2k}}} \sin \frac{\pi}{\frac{1}{\frac{1}{2} + 2k}} \\ 
        = \lim\limits_{k \to \infty}2\pi \sqrt{\frac{1}{2} + 2k}\sin(\frac{\pi}{2} + 2k\pi) = +\infty.
        \end{split}
    \end{equation*}
\end{ex}

Udowodnimy teraz reguły obliczania całek oznaczonych.

\begin{theorem}
{
    (o całkowaniu przez części). Niech pochodne funkcji $u$ i $v$ będą R-całkowalne
    na przedziale $[a,b]$. Wówczas zachodzi wzór (zwany wzorem na całkowanie przez części.)
    \[
        \int_{a}^{b}u(x)v'(x)dx = u(x)v(x) \big|_a^b - \int_{a}^{b}u'(x)v(x)dx,
    \]
    gdzie $u(x)v(x) \big|_a^b = u(b)v(b) - u(a)v(a)$.
}
\end{theorem}

\begin{proof}
    Istotnie, ponieważ $(uv)'(x) = u(x)v'(x) + u'(x)v(x)$, więc
    \[
        \int_{a}^{b}(uv)'(x)dx = \int_{a}^{b}u(x)v'(x)dx + \int_{a}^{b}u'(x)v(x)dx,
    \]
    a zatem na mocy wzoru Newtona-Leibniza otrzymujemy 
    \[
        u(x)v(x) \big|_a^b = \int_{a}^{b}u(x)v'(x)dx + \int_{a}^{b}u'(x)v(x).
    \]
\end{proof}

\begin{wniosek}
{
    Jeśli funkcja $f$ ma na przedziale o końcach $x_0$ i $x$ ciągłe pochodne do rzędu $n+1$ włącznie to
    \begin{equation*}
        f(x) = f(x_0) + \frac{f'(x_0)}{1!}(x-x_0) + \frac{f''(x_0)}{2!}{(x-x_0)}^2 + \ldots \frac{f^{(n)}(x_0)}{n!}{(x-x_0)}^n + r_n(x_0, x),
    \end{equation*}
    gdzie $r_n(x_0, x) = \frac{1}{n!}\int_{x_0}^{x}f^{(n+1)}(t){(x-t)}^n dt$. \\
    (Wzór Taylora dla funkcji $f$ z resztą w postaci całkowej).
}
\end{wniosek}

\begin{proof}
    Stosując Wzór Newtona-Leibniza i wzór na całkowanie przez części wykonujemy następujący ciąg przekształceń, w którym wszystkie różniczkowania i podstawienia
    wykonywane są względem $t$: 
    \begin{multline*}
        f(x) - f(x_0) = \int_{x_0}^{x}f'(t)dt = -\int_{x_0}^{x}f'(t)(x-t)'dt \\
        = -f'(t)(x-t) \big|_{x_0}^x + \int_{x_0}^{x}f''(t)(x-t)dt = f'(x_0)(x-x_0) - \frac{1}{2}\int_{x_0}^{x}f''(t)({(x-t)}^2)'dt \\
        = f'(x_0)(x-x_0) - \frac{1}{2}f''(t){(x-t)}^2 \big|_{x_0}^x + \frac{1}{2}\int_{x_0}^{x}f''(t){(x-t)}^2dt \\
        = f'(x_0)(x-x_0) + \frac{1}{2}f''(x_0){(x-x_0)}^2 - \frac{1}{2 \cdot 3}\int_{x_0}^{x}f'''(t)({(x-t)}^3)'dt \\
        = f'(x_0)(x-x_0) + \frac{1}{2}f''(x_0){(x-x_0)}^2 + \cdots + \frac{1}{n!}f^{(n)}(x_0){(x-x_0)}^n \\ 
        + \frac{1}{n!}\int_{x_0}^{x}f^{(n+1)}(t){(x-t)}^n dt.
    \end{multline*}
\end{proof}

\begin{theorem}
{
    (o całkowaniu przez podstawienie). Jeśli funkcja $f : [a,b] \mapsto \mathbb{R}$ jest ciągła, a $\phi : [\alpha, \beta] \mapsto [a,b]$ ma ciągłą
    pochodną na przedziale $[\alpha, \beta]$ oraz $\phi(\alpha) = a$, $\phi(\beta) = b$, to
    \[
        \int_{a}^{b}f(x)dx = \int_{\alpha}^{\beta}f(\phi(t))\phi'(t)dt.
    \]
}
\end{theorem}
\begin{proof}
    Ponieważ funkcje podcałkowe są ciągłe, zatem całki bo obu stronach powyższej równości istnieją.
    Jeśli $F$ jest funkcją pierwotną funkcji $f$ na przedziale $[a,b]$, to $\Phi = F \circ \phi$ jest funkcją pierwotną funkcji
    $(f \circ \phi)\phi'$ na przedziale $[\alpha, \beta]$. Na mocy Wzoru Newtona-Leibniza mamy zatem
    \[
        \int_{a}^{b}f(x)dx = F(b) - F(a) 
    \]
    oraz
    \[
        \int_{\alpha}^{\beta}f(\phi(t))\phi'(t)dt = F(\phi(\beta)) - F(\phi(\alpha)) = F(b) - F(a),
    \]
    co kończy dowód twierdzenia.
\end{proof}

Powyższe twierdzenie jest wystarczające dla wielu zastosowań. Można udowodnić następujące jego uogólnienie:

\begin{theorem}
{
    Niech $\phi : [\alpha, \beta] \mapsto [a,b]$ będzie ściśle monotinicznym przekształceniem przedziału
    $[\alpha, \beta]$ na przedział $[a,b]$ i niech pochodna $\phi'$ będzie R-całkowalna na $[a,b]$. Wówczas dla dowolnej funkcji $f$ R-całkowalnej
    na przedziale $[a,b]$ funkcja $(f \circ \phi)\phi'$ jest R-całkowalna na $[\alpha, \beta]$ oraz zachodzi równość
    \[
        \int_{\phi(\alpha)}^{\phi(\beta)}f(x)dx = \int_{\alpha}^{\beta}f(\phi(t))\phi'(t)dt.
    \]
}
\end{theorem}

Dowód powyższego twierdzenia można znaleźć w książce [9].

\begin{ex}
    Obliczmy $\int_{\frac{1}{\pi}}^{\frac{2}{\pi}}\frac{1}{x^2}\sin \frac{1}{x}dx$. Niech $f(x) = \frac{1}{x^2}\sin \frac{1}{x}$, $x \in [\frac{1}{\pi}, \frac{2}{\pi}]$
    oraz niech $\phi(t) = \frac{1}{t}$, $t \in [\frac{\pi}{2}, \pi]$. Na mocy twierdzenia $\ref{theorem:90}$ otrzymujemy
    \[
        \int_{\frac{1}{\pi}}^{\frac{2}{\pi}}\frac{1}{x^2}\sin \frac{1}{x}dx = 
        \int_{\pi}^{\frac{\pi}{2}}t^2 \sin t(-\frac{1}{t^2})dt = 
        \int_{\frac{\pi}{2}}^{\pi}\sin t dt = -\cos t \big|_{\frac{\pi}{2}}^{\pi} = 1.
    \]
\end{ex}

Udowodnimy teraz dwa twierdzenia całkowe o wartości średniej.

\begin{theorem}
{
    (I Twierdzenie Całkowe o Wartości Średniej). Niech funkcje $f$, $g$ będą R-całkowalne na przedziale
    $[a,b]$ oraz niech $m = \inf\{f(x) : x \in [a,b]\}$, $ M = \sup\{f(x) : x \in [a,b]\}$. Jeśli funkcja $g$
    jest nieujemna lub niedodatnia na $[a,b]$, to
    \begin{equation}\label{eq:50}
        \int_{a}^{b}(fg)(x)dx = \mu \int_{a}^{b}g(x)dx,
    \end{equation}
    gdzie $\mu \in [m, M]$. Jeśli ponadto funkcja $f$ jest ciągła na przedziale $[a,b]$, to istnieje punkt 
    $\xi \in [a,b]$ taki, że 
    \begin{equation}\label{eq:51}
        \int_{a}^{b}(fg)(x)dx = f(\xi)\int_{a}^{b}g(x)dx.
    \end{equation}
}
\end{theorem}

\begin{proof}
    Dla ustalenia uwagi załóżmy, że $g(x) \geqslant 0$ dla $x \in [a,b]$. Wówczas $mg(x) \leqslant f(x)g(x) \leqslant Mg(x)$ dla $x \in [a,b]$, a zatem
    \begin{equation}\label{eq:52}
        m\int_{a}^{b}g(x)dx \leqslant \int_{a}^{b}(fg)(x)dx \leqslant M\int_{a}^{b}g(x)dx.
    \end{equation}
    Jeśli $\int_{a}^{b}g(x) = 0$, to z (\ref{eq:52}) wynika natychmiast (\ref{eq:50}).
    Natomiast gdy $\int_{a}^{b}g(x) \neq 0$, to biorąc
    $\mu = (\int_{a}^{b}g(x)dx)^{-1} \int_{a}^{b}(fg)dx$ i uwzględniając (\ref{eq:52}) otrzymujemy
    $m \leqslant \mu \leqslant M$. 
    Równość (\ref{eq:51}) wynika z (\ref{eq:50}) i z tego, że funkcja ciągła na przedziale domkniętym osiąga swoje kresy i ma Właśność Darboux.
\end{proof}

\begin{wniosek}
{
    Jeśli $g(x) = 1$ dla $x \in [a,b]$, to wzór (\ref{eq:51}) przyjmuje postać
    \[
        \int_{a}^{b}f(x)dx = f(\xi)(b-a).
    \]
}
\end{wniosek}

\begin{theorem}
{
    (II Twierdzenie Całkowe o Wartości Średniej). Jeśli funkcje $f$ i $g$ są R-całkowalne na przedziale $[a,b]$ i ponadto funkcja $g$ jest monotoniczna,
    to istnieje punkt $\xi \in [a,b]$ taki, że
    \begin{equation}\label{eq:53}
        \int_{a}^{b}(fg)(x)dx = g(a)\int_{a}^{\xi}f(x)dx + g(b)\int_{\xi}^{b}f(x)dx.
    \end{equation}
}
\end{theorem}

\begin{proof}
    Dowód przeprowadzimy w szczęgolnym przypadku, gdy funkcja $f$ jest ciągła, a funkcja $g$ jest klasy $C^1$. Dowód w przypadku ogólnym można znależć np. w [2], t. II, s. 101-102 lub w [11], s. 359-363. \\
    Niech $F$ będzie dowolną funkcją pierwotną funkcji $f$, czyli $F'=f$. Całkując przez części otrzymujemy
    \[
        \int_{a}^{b}f(x)g(x)dx = \int_{a}^{b}F'(x)g(x)dx - F(x)g(x) \big|_a^b - \int_{a}^{b}F(x)g'(x)dx.
    \]
    Ponieważ funkcja $g$ jest monotoniczna, zatem jej pochodna na przedziale $[a,b]$ ma stały znak, zatem na mocy
    Twierdzenia \ref{theorem:92} mamy
    \[
        \int_{a}^{b}F(x)g'(x) = F(\xi)\int_{a}^{b}g'(x)dx
    \]
    dla pewnego $\xi \in [a,b]$. Stąd
    \begin{align*}
        \int_{a}^{b}f(x)g(x)dx &= F(b)g(b) - F(a)g(a) - F(\xi)g(b) + F(\xi)g(a) \\
                               &= g(a)(F(\xi) - F(a)) + g(b)(F(b) - F(\xi)) \\
                               &= g(a)\int_{a}^{\xi}f(x)dx + g(b)\int_{\xi}^{b}f(x)dx.
    \end{align*}
\end{proof}

\begin{uwaga}
    II Twierdzenie Całkowe o Wartości Średniej bywa podawane w różnych postaciach.
    \begin{enumerate}
        \item[(a)] 
            Jeśli w przedziale $[a,b]$ funkcja $g$ jest nierosnąca i nieujemna, a funckja $f$ jest R-całkowalna to
            \begin{equation}\label{eq:54}
                \int_{a}^{b}(fg)(x)dx = g(a)\int_{a}^{\xi}f(x)dx,
            \end{equation}
            gdzie $\xi$ jest pewnym punktem z przedziału $[a,b]$. 
        \item[(b)] 
            Analogicznie, jeśli funkcja $g$ jest niemalejąca i nieujemna, to zachodzi wzór
            \begin{equation}\label{eq:55}
                \int_{a}^{b}(fg)(x)dx = g(b)\int_{\xi}^{b}f(x)dx,
            \end{equation}
            gdzie $\xi \in [a,b]$.
    \end{enumerate}
    Wzory (\ref{eq:53}), (\ref{eq:54}), (\ref{eq:55}) nazywają się wzorami Bonneta.
\end{uwaga}

\subsection{Całka z funkcji o wartościach zespolonych}
\begin{defn}
    Niech $f_1$, $f_2$ będą funkcjami rzeczywistymi określonymi na przedziale $[a,b]$ i niech $f = (f_1, f_2)$ będzie odwzorowaniem przedziału $[a,b]$ w zbiór $\mathbb{C}$.
    Mówimy, że $f$ jest całkowalna w sensie Riemanna na przedziale $[a,b]$, jeśli funkcje $f_1$, $f_2$ są R-całkowalne na tym przedziale.
    W tym wypadku określamy
    \[
        \int_{a}^{b}f(x)dx = \Bigg(\int_{a}^{b}f_1(x)dx, \int_{a}^{b}f_2(x)dx\Bigg),
    \]
    lub równoważnie
    \[
        \int_{a}^{b}f(x)dx = \int_{a}^{b}\text{Re} f(x)dx + i \int_{a}^{b} \text{Im} f(x)dx.
    \]
\end{defn}

Jest oczywiste, że Twierdzenie \ref{theorem:82} (dla sumy oraz iloczynu przez liczbę rzeczywistą funkcji R-całkowalnych)
jest prawdziwe także dla funkcji o wartościach w $\mathbb{C}$. To samo dotyczy twierdzeń 
\ref{theorem:85}, \ref{theorem:86}, \ref{theorem:88}, \ref{theorem:90} (por. [1], s. 272-273).
Aby się o tym przekonać, należy jedynie zastosować poprzednie rezultaty do poszczególnych współrzędnych.
Dla przykłady sformułujemy Podstawowe Twierdzenie Rachunku Całkowego.

\begin{theorem}
{
    Niech $f$ i $F$ będą funkcjami określonymi na przedziale $[a,b]$ o wartościach w $\mathbb{C}$. Jeśli odwzorowanie $f$ jest R-całkowalne na tym przedziale
    oraz $F'(x) = f(x)$ dla $x \in [a,b]$, to
    \[
        \int_{a}^{b}f(x)dx = F(b) - F(a).
    \]
}
\end{theorem}

Prawdziwy jest również analog Twierdzenia \ref{theorem:83} (b), jednakże jego dowód jest bardziej subtelny.

\begin{theorem}
{
    Jeśli odwzorowanie $f : [a,b] \mapsto \mathbb{C}$ jest R-całkowalne, to funkcja $|f|$ jest również R-całkowalna oraz
    \[
        \Big| \int_{a}^{b}f(x)dx \Big| \leqslant \int_{a}^{b} |f(x)|dx.
    \]
}
\end{theorem}

\begin{proof}
    Niech $f_1$, $f_2$ będą składowymi odwzorowania $f$ (to znaczy $f = (f_1, f_2)$).
    Wówczas $|f| = \sqrt{f_1^2+f_2^2}$. Każda z funkcji $f_1^2$, $f_2^2$ jest R-całkowalna, więc z ciągłości
    pierwiastka i Twierdzenia \ref{theorem:81} wynika, że funkcja $|f|$ jest również całkowalna.
    Niech $y = (y_1, y_2)$, gdzie $y_i = \int_{a}^{b}f_i(x)dx$ dla $i = 1,2$. Wówczas
    $y = \int_{a}^{b}f(x)dx$ oraz
    \[
        |y|^2 = y_1^2 + y_2^2 = y_1 \int_{a}^{b}f_1(x)dx + y_2\int_{a}^{b}f_2(x)dx = 
        \int_{a}^{b}(y_1f_1(x) + y_2f_2(x))dx.
    \]
    Na podstawie nierówności Schwarza mamy 
    \[
        y_1f_1(x) + y_2f_2(x) \leqslant |y||f(x)| \text{ dla } x \in [a,b],
    \]
    a zatem
    \[
        |y|^2 \leqslant |y|\int_{a}^{b}|f(x)|dx.
    \]
    Dzieląc ostatnią nierówność przez $|y| \neq 0$, otrzymujemy tezę (dla $y = 0$ twierdzenie jest oczywiste).
\end{proof}

W przypadku funkcji R-całkowalnych określonych na przedziale $[a,b]$ o wartościach w $\mathbb{C}$, Wniosek \ref{wniosek:24}
nie zachodzi (por. Rozdział 7.6). Prawdziwe jest natomiast następujące

\begin{theorem}
{
    Jeśli $f : [a,b] \mapsto \mathbb{C}$ jest funkcja R-całkowalną oraz $f([a,b]) \subset B(x^0, r)$,
    gdzie $B(x^0, r)$ oznacza kule domkniętą o środku w punkcie $x^0 \in \mathbb{C}$ i promieniu $r$, to
    \begin{equation}\label{eq:56}
        \frac{1}{b-a}\int_{a}^{b}f(x)dx \in B(x^0, r).
    \end{equation}
}
\end{theorem}

\begin{proof}
Rozważmy funkcję $g(t) = f(t) - x^0$, $t \in [a,b]$. Mamy
\begin{align*}
    \Bigg|\frac{1}{b-a}\int_{a}^{b}f(t)dt - x^0\Bigg| &= \frac{1}{b-a}\Bigg|\int_{a}^{b}(f(t)-x^0)dt\Bigg| \\
    &\leqslant \frac{1}{b-a}\int_{a}^{b}|f(t)-x^0|dt \leqslant \frac{1}{b-a}(b-a)r = r,
\end{align*}
a zatem (\ref{eq:56}) zachodzi.
\end{proof}

\subsection{Zastosowania całki Riemanna}
Wiele zastosowań całki Riemanna opiera się na następującym twierdzeniu, które podaje warunek na to, aby
addytywna funkcja przedziału była generowana przez całkę. 

\begin{theorem}
{
    Jeśli dla addytywnej funkcji $J(\alpha, \beta)$ określonej dla punktów $\alpha, \beta \in [a,b]$ istnieje
    funkcja R-całkowalna na $[a,b]$ i taka, że 
    \[
        \inf_{x \in [\alpha, \beta]} f(x)(\beta - \alpha) \leqslant J(\alpha, \beta) \leqslant \sup_{x \in [\alpha, \beta]} f(x)(\beta - \alpha)
    \]
    dla dowolnych $a \leqslant \alpha < \beta \leqslant b$, to
    \[
        J(a,b) = \int_{a}^{b}f(x)dx.
    \]
}
\end{theorem}

\begin{proof}
    Niech $ P = \{x_0, \ldots, x_n\}$ będzie dowolnym podziałem przedziału $[a,b]$ i niech
    $m_i = \inf\{f(x) : x \in [x_{i-1}, x_i]\}$ oraz $M_i = \sup\{f(x) : x \in [x_{i-1}, x_i]\}$ ($i = 1, \ldots, n$).
    Dla dowolnego przedziału $[x_{i-1}, x_i]$ mamy 
    \[
        m_i \Delta x_i \leqslant J(x_{i-1}, x_i) \leqslant M_i \Delta x_i.
    \]
    Sumując powyższe nierówności i korzystając z addytywności funkcji $J(\alpha, \beta)$ otrzymujemy
    \[
        L(f, P) = \sum_{i=1}^{n}m_i \Delta x_i \leqslant J(a,b) \leqslant \sum_{i = 1}^{n}M_i \Delta x_i = U(f, P).
    \]
    Niech będzie dane dowolne $\varepsilon > 0$. Wówczas istnieje taki podział $P$ przedziału $[a,b]$, że
    $U(f, P) - L(f, P) < \varepsilon$. Mamy zatem
    \[
        \int_{a}^{b}f(x)dx - \varepsilon < L(f, P) \leqslant J(a,b) \leqslant U(f, P) < \int_{a}^{b}f(x)dx + \varepsilon,
    \]
    czyli $\big|J(a,b) - \int_{a}^{b}f(x)dx \big| < \varepsilon$, a zatem wobec dowolności $\varepsilon > 0$ otrzymujemy tezę.
\end{proof}

Omówimy teraz kilka geometrycznych zastosowań całki Riemanna. Zajmiemy się wpierw zagadnieniem długości krzywej.

\begin{defn}
    Ciągłe odwzorowanie $\gamma$ przedziau $[a,b]$ w zbiór $\mathbb{R}$ lub $\mathbb{C}$ nazywamy krzywą
    lub drogą w $\mathbb{R}$ lub $\mathbb{C}$. Punkty $A = \gamma(a)$, $B = \gamma(b)$, nazywają się odpowiednio
    początkiem i końcem drogi. Jeśli $\gamma(a) = \gamma(b)$, to powiemy, że $\gamma$ jest krzywą zamkniętą.
    Zbiór $\gamma([a,b])$ nazywamy obrazem krzywej $\gamma$. 
\end{defn}

\begin{uwaga}
    Zauważmy, że jeden i ten sam zbiór może być obrazem wielu różnych krzywych.
    Ponadto może on okazać się nie tym, co w naszym potocznym wyobrażeniu jest linią.
    Istnieją przykłady krzywych, które wypełniają cały kwadrat jednostkowy (tak zwane ,,Krzywe Peano'',
    zob. R. Eugelking, K. Sieklucki, Geometria i topologia, cz-II, s. 134-135).
\end{uwaga}

\begin{defn}
Krzywą $\gamma : [a,b] \mapsto \mathbb{R}$ (lub $\mathbb{C}$) nazywamy łukiem, jeśli odwzorowanie $\gamma$ jest wzajemnie
jednoznacznie. Krzywą zamkniętą $\gamma : [a,b] \mapsto \mathbb{R}$ (lub $\mathbb{C}$) nazywamy łukiem zamkniętym, jeśli funkcja $\gamma$ jest wzajemnie jednoznaczna na przedziale $[a,b)$.
\end{defn}

\begin{defn}
    Z każdym podziałem $P = \{x_0, x_1, \ldots x_n\}$ przedziału $[a,b]$ i z dowolną krzywą $\gamma$ w
    $\mathbb{R}$ lub $\mathbb{C}$ wiążemy liczbę
    \[
        V(P, \gamma) = \sum_{i=1}^{n}|\gamma(x_i)-\gamma(x_{i-1})|,
    \]
    gdzie $i$-ty składnik powyższej sumy oznacza odległość pomiędzy puntkami $\gamma(x_{i-1})$ oraz
    $\gamma(x_{i})$. Zatem $V(P, \gamma)$ jest długością łamanej o wierzchołkach $\gamma(x_1), \ldots \gamma(x_n)$ występujacych w takim porządku. \\
    Długością krzywej $\gamma$ nazywamy liczbę
    \[
        L(\gamma) = l(a,b) = \sup_P V(P, \gamma),
    \]
    gdzie kres górny jest wzięty po zbiorze wszytkich podziałów przedziału $[a,b]$.
    Jeśli $L(\gamma) < +\infty$, to powiemy, że $\gamma$ jest krzywą prostowalną. 
\end{defn}

W pewnych przypadkach możemy obliczać $L(\gamma)$ jako całkę Riemanna.

\begin{defn}
Krzywa $\gamma : [a,b] \mapsto \mathbb{R}$ (lub $\mathbb{C})$ nazywa się krzywą danej klasy gładkości,
jeśli funkcja $\gamma$ należy do tej klasy (to znaczy do klasy $C^{(n)}$ przy pewnym $n$).
Krzywą klasy $C^1$ nazywamy krzywą gładką. Natomiast mówimy, że krzywa jest kawałkami gładka, jeśli przedzial $[a,b]$ można 
podzielić na skończoną liczbę przedziałów tak, że na każdym z nich funkcja $\gamma$ jest klasy $C^1$.
\end{defn}

\begin{theorem}
{
    Jeśli krzywa $\gamma : [a,b] \mapsto \mathbb{R}$ (lub $\mathbb{C}$) jest gładka, to $\gamma$ jest prostowalna oraz
    \begin{equation}\label{eq:57}
        L(\gamma) = \int_{a}^{b}|\gamma'(t)|dt.
    \end{equation}
}
\end{theorem}

\begin{proof}
    Jest jasne, że dla $a \leqslant \alpha < \beta \leqslant b$ zachodzi równość $L(\alpha, \gamma) = L(\alpha, \beta) + L(\beta, \gamma)$.
    Ponadto wobec Twierdzenia Lagrange'a (Tw. 62) mamy
    \[
        \inf_{t \in [\alpha, \beta]} |\gamma'(t)|(\beta - \alpha) \leqslant L(\alpha, \beta) \leqslant \sup_{t \in [\alpha, \beta]} |\gamma'(t)|(\beta - \alpha).
    \]
    Wzór (\ref{eq:57}) jest więc konsekwencją Twierdzenia \ref{theorem:97} (to, że krzywa $\gamma$ jest prostowalna wynika z Twierdzenia Lagrange'a i założenia, że
    $\gamma \in C^1$; mamy bowiem $L(\gamma) \leqslant \sup\limits_{t \in [a,b]} \gamma'(t)(b-a) < +\infty)$.
\end{proof}

\begin{uwaga}
    \begin{itemize}
        \item [(a)] 
            Niech będzie dana krzywa na płaszczyźnie (w $\mathbb{C}$) o równaniach
            $x = x(t)$, $y = y(t)$ ($\gamma(t) = (x(t), y(t))), t \in [a,b]$. Jeśli krzywa $\gamma$ jest gładka, to na mocy
            \ref{eq:57} jej długość wyrażą się wzorem
            \begin{equation}\label{eq:58}
                L(\gamma) = \int_{a}^{b}\sqrt{x'(t)^2 + y'(t)^2}dt.
            \end{equation}
        \item [(b)] 
            Długość wykresu funkcji $y = f(x)$, $x \in [a,b]$, klasy $C^1$ wyraża się wzorem
            \[
                L(a,b) = \int_{a}^{b}\sqrt{1 + f'(x)^2}dx.
            \]
            Wynika to ze wzoru (\ref{eq:58}) bowiem $\{(x, f(x)) : x \in [a,b]\} = \gamma([a,b])$, gdzie
            $\gamma(t)=(t, f(t))$ dla $t \in [a,b]$. 
        \item [(c)]
            \textbf{Współrzędne biegunowe} \\
            Przez dowolny punkt $O$ zwany biegunem poprowadzimy oś $S$, mającą początek w tym punkcie.
            Współrzędnymi biegunowymi punktu $P$ nazywamy liczbe $r$ będącą długościa wektora $\overrightarrow{OP}$ i liczbę
            $\phi \in [0, 2\pi)$ będącą miarą łukową kąta skierowanego $\angle{(S, \overrightarrow{OP})}$. Liczbę $\phi$ nazywamy amplitudą
            punktu $P$ (biegunowi, to znaczy punktowi $O$, można przyporządkować dowolną amplitudę), natomiast $r$ - promieniem wodzącym punktu $P$. \\
            Obierzmy układ kartezjański prostokątny i układ biegunowy tak, by biegun leżał w początku układu kartezjańskiego, a oś biegunowa pokrywała się z osią odciętych.
            Oznaczmy przez $(x, y)$ współrzędne prostokątne oraz przez $(r, \phi)$ - współrzędne biegunowe tego samego punktu w obu układach. Wówczas zachodzą następujące związki:
            \begin{align*}
                x &= r\cos \phi \\
                y &= r \sin \phi \\
                r &= \sqrt{x^2+y^2} \text{ dla } x^2 + y^2 > 0 \\
                & \cos \phi = \frac{x}{\sqrt{x^2+y^2}}, & \sin \phi = \frac{y}{\sqrt{x^2+y^2}}.
            \end{align*}
            Załóżmy, że funkcja $g$ klasy $C^1$ jest określona we współrzędnych biegunowych, to znaczy $r = g(\phi)$, gdzie $\phi_1 \leqslant \phi \leqslant \phi_2$. 
            Za pomocą wzorów $x = r\cos \phi = g(\phi)\cos \phi$, $y = r\sin \phi = g(\phi)\sin \phi$ otrzymujemy przedstawienie parametryczne funkcji $g$, zatem
            \[
                L(\phi_1, \phi_2) = \int_{\phi_1}^{\phi_2} \sqrt{g(\phi)^2 + g'(\phi)^2}d\phi
            \]
        \item [(d)]
            Gdy krzywa jest kawałkami gładka, to aby obliczyć jej długość dzielimy ją na skończoną liczbę krzywych gładkich i do każdej z nich stosujemy wzór (\ref{eq:57}), a następnie dodajemy otrzymane liczby.
    \end{itemize}
\end{uwaga}

Zajmiemy się teraz zastosowaniem całek Riemanna do obliczania pola powierzchni.

\begin{defn}
    Rozpatrzmy na płaszczyźnie dowolną figurę $P$, która jest obszarem ograniczonym. 
    Załóżmy, że brzeg figury $P$ jest obrazem gładkiej krzywej zamkniętej $\gamma : [a,b] \mapsto \mathbb{C}$ (lub składa się z obrazów kilku takich krzywych).
    Zakładamy, że $\gamma'(t) \neq 0$ dla każdego $t \in [a,b]$. Jordan udowodnił, że rozważana krzywa zamknięta rozcina płaszczyznę na dwa obszary,
    wewnętrzny i zewnętrzny, dla których jest ona wspólnym brzegiem. Rozważmy wszystkie możliwe wielokąty $A$, całkowicie zawarte w $P$ i wielokąty $B$, całkowicie zawierające obszar $P$.
    Jeśli $|A|$ i $|B|$ oznaczają pola tych wielokątów, to $|A| \leqslant |B|$. Zbiór $\{|A| : A \subset P\}$ jest ograniczony z góry przez którąkolwiek z liczb $|B|$, a zatem na mocy
    Aksjomatu Dedekinda posiada on kres górny $|P_*|$. Natomiast zbiór $\{|B| : P \subset B\}$ jest ograniczony z dołu przez liczbę $|P_*|$ i posiada kres dolny $|P^*|$. Kres
    górny $|P_*|$ nazywamy wewnętrzną, a kres dolny $|P^*|$ --- zewnętrzną miarą Jordana figury $P$. \\
    Jeśli $|P_*| = |P^*|$, to mówimy, że figura $P$ jest mierzalna w sensie Jordana, a wspólną wartość tych kresów nazywamy miarą Jordana lub polem figury $P$ i oznaczamy symbolem $|P|$. \\
    Będziemy mówili, że figura ma pole równe zeru, jeśli można pokryć ją obszarem wielokątnym o dowolnie małym polu. 
\end{defn}

\end{justify}
\end{document}