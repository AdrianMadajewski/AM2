\documentclass[leqno]{article}
\usepackage{graphicx} % Required for inserting images
\usepackage[polish]{babel} % Language support for Polish
\usepackage[utf8]{inputenc} % Input encoding
\usepackage[T1]{fontenc}
\usepackage{ragged2e}
\usepackage{amssymb} % Load before mathrsfs
\usepackage{mathrsfs} % Pakiet mathrsfs dostarcza komendę \mathscr
\usepackage{hyperref} % for clickable links
\usepackage{amsmath} % Load after amssymb

\setlength{\parindent}{0pt} % Remove indentation globally
\setlength{\parskip}{10pt}

\setcounter{equation}{35}

\title{Analiza Matematyczna 2}
\author{Adrian Madajewski}
\date{Semestr II}

% Definicja
\newcounter{defcounter}
\setcounter{defcounter}{67}
\newcommand{\defn}{\noindent\refstepcounter{defcounter}\textbf{Definicja \thedefcounter. }\label{def:\thedefcounter}}

% Uwaga
\newcounter{uwagacounter}
\setcounter{uwagacounter}{37}
\newcommand{\uwaga}{\noindent\refstepcounter{uwagacounter}\textbf{Uwaga \theuwagacounter. }\label{uwaga:\theuwagacounter}}

% Przyklad
\newcounter{excounter}
\setcounter{excounter}{24}
\newcommand{\ex}{\noindent\refstepcounter{excounter}\textbf{Przykład \theexcounter. }\label{ex:\theexcounter}}

% Twierdzenie
\newcounter{thcounter}
\setcounter{thcounter}{73}
\newcommand{\theorem}[1]{\noindent\refstepcounter{thcounter}\textbf{Twierdzenie \thethcounter. }\textit{#1}\label{theorem:\thethcounter}}

% Dowód
\newcommand{\proof}{\noindent\textbf{Dowód. }}

% Equation
\newcounter{eqcounter}
\setcounter{eqcounter}{38}
\newcommand{\eq}{\refstepcounter{eqcounter}\label{eq:\theeqcounter}}

\begin{document}

\maketitle

\begin{justify}

\setcounter{section}{7}
\section{Całka Riemanna}
\subsection{Definicja i podstawowe własności całki}
\defn{}{} Niech $[a,b]$ będzie danym przedziałem. Przez podział $P$ przedziału $[a,b]$ będziemy nazywali skończony zbiór punktów \(x_0, x_1, \ldots, x_n\), gdzie
\[
a = x_0 < x_1 < \cdots < x_n = b
\]
Będziemy pisać $\Delta x_i = x_i - x_{i-1}$ $(i = 1, \ldots , n)$. Długość największego z odcinków $[x_{i-1}, x_i]$ nazywać będziemy średnicą podziału $P$ i oznaczamy ją symbolem $\delta (P)$. $\delta (P)= \max\limits_{1 \leq i \leq n} \Delta x_i$. Niech $f$ będzie ograniczoną funkcją rzeczywistą określoną na $[a,b]$. W każdym z przedziałów $[x_{i-1}, x_i]$ wybierzmy dowolny punkt $\xi_{i}$ $(i=1, \ldots n)$ i utwórzmy sumę $R = \sum_{i=1}^{n} f(\xi_{i})\Delta x_{i}$. Sumę te nazywamy sumą Riemanna odpowiadającą podziałowi $P$, przy ustalonym wyborze punktów $\xi_i$. Przez $\mathfrak{R}(f,P)$ oznaczać będziemy zbiór wszystkich możliwych sum Riemanna odpowiadających podziałowi $P$. Utwórzmy teraz ciąg $(P_k)$ podziałów przedziału $[a, b]$:
\[
a = x_0^{(k)} < x_1^{(k)} < \cdots < x_{n(k)}^{(k)} = b;
\]
\[
\Delta_i^{(k)}=x_i^{(k)}-x_{i-1}^{(k)};
\]
\[
\delta (P_k)=\max_{1 \leq i \leq n(k)} \Delta x_i^{(k)}, k = 1,2, \ldots
\]
Ciąg $(P_k)$ nazywamy ciągiem normalnym podziałów, jeśli $\delta(P_k)\rightarrow 0$  przy $k \rightarrow \infty$. Oznaczmy przez $\mathfrak{R}(f,P_k)$ zbiór wszystkich sum Riemanna odpowiadających podziałowi $P_k$.

\noindent
\defn{} Jeśli dla dowolnego ciągu normalnego podziałów $(P_k)$ i dla dowolnych sum Riemanna $R_k \in \mathfrak{R}(f,P_k)$ istnieje skończona granica $I = \lim\limits_{k \rightarrow \infty}R_k$, to tę granicę nazywamy całką Riemanna funkcji $f$ na przedziale $[a,b]$ i oznaczamy ją symbolem 
\[
\begin{aligned}
    \int_{a}^{b} fdx \text{ lub } \int_{a}^{b} f(x)dx
\end{aligned}
\]
O funkcji $f$ mówimy wówczas, że jest całkowalna w sensie Riemanna na przedziale $[a,b]$, lub że jest ona R-całkowalna na tym przedziale. 

Powyższą definicję mozna sformułować w następujący równoważny sposób.

\noindent
\defn{} Mówimy, że funkcja $f$ jest całkowalna w sensie Riemanna na przedziale $[a,b]$, jeśli istnieje liczba $I \in \mathbb{R}$ taka, że
\[
\forall_{\epsilon>0} \exists_{\delta > 0} \forall_P \forall_{R \in \mathfrak{R}(f, P)}\delta(P) < \delta \implies |R - I| < \epsilon
\]
Piszemy wówczas $I = \int_{a}^{b}f(x)dx = \lim\limits_{\delta(P) \rightarrow 0}R$.

Równoważność definicji $\ref{def:69}$ i $\ref{def:70}$ można pokazać analogicznie jak w dowodzie twierdzenia 35. 

\ex{} (a) Funkcja stała $f(x)=c, c \in \mathbb{R}, x \in [a,b]$ jest całkowalna w sensie Riemanna na tym przedziale. Niech $P$ będzie dowolnym podziałem przedziału $[a,b]$:
\[
a = x_0 < x_1 < \cdots < x_n = b
\]
Dowolna suma Riemanna odpowiadającą podziałowi $P$ ma postać:
\[
R = \sum_{i=1}^{n}f(\xi_i)(x_i-x_{i-1}) = \sum_{i=1}^{n}c(x_i - x_{i-1})=c(b-a),
\]
\[
    (\xi_i \in [x_{i-1},x_i], i=1, \ldots, n)
\]
Stąd wynika, że $\int_{a}^{b} f(x)dx = c(b-a)$. 

\noindent
(b) Roważmy ponownie funkcję Dirichleta z Przykładu 18 (a), zawężoną do przedziału $[a,b]$. Dla każdego podziału $P$ przedziału $[a,b]$ można utworzyć sumę Riemanna równą zeru, jeśli wszystkie punkty $\xi_i$ będą liczbami niewymiernymi, lub równą $(b-a)$, jeśli wszystkie punkty $\xi_i$ będą liczbami wymiernymi. Jest więc jasne, że dla każdego ciągu normalnego podziałów $(P_k)$ granica $\lim\limits_{k \rightarrow \infty}R_k$, gdzie $R_k \in \mathfrak{R}(f, P_k)$, $k \in \mathbb{N}$, nie istnieje. 

\defn{}\label{def:sumy_dolne_gorne} Niech $f$ będzie ograniczoną funkcją rzeczywistą określona na $[a,b]$. Każdemu podziałowi $P$ przedziału $[a,b]$ odpowiadają liczby:
\[
M_i = \sup\limits_{x_{i-1} \leq x \leq x_i} f(x) \quad m_i = \inf\limits_{x_{i-1} \leq x \leq x_i} f(x)
\]
\[
U(f, P) = \sum_{i=1}^{n}M_i\Delta x_i \quad L(f, P) = \sum_{i=1}^{n}m_i\Delta x_i
\]
Liczby $U(f,P)$ i $L(f, P)$ nazywamy odpowiednio sumą górną i dolną lub sumami Darboux funkcji $f$ przy podziale $P$ przedziału $[a,b]$. Dalej,

\begin{equation}\label{eq:36}
    \overline{\int_{a}^{b} f(x)dx} = \inf_P U(f, P),
\end{equation}

\begin{equation}\label{eq:37}
    \underline{\int_{a}^{b} f(x)dx} = \sup_P L(f, P),
\end{equation}
gdzie kres górny i dolny są brane ze względu na wszystkie podziały $P$ przedziału $[a,b]$. Lewe strony równości (\ref{eq:36}) i (\ref{eq:37}) nazywają się odpowiednio górną i dolną całką Darboux funkcji $f$ na przedziale $[a,b]$.

Ponieważ funkcja $f$ jest ograniczona, więc istnieją liczby rzeczywistę $m$ i $M$ takie, że
\[
m \leq f(x) \leq M \quad \text{dla} \quad x \in [a,b] 
\]
Oznacza to, że przy dowolnym podziale $P$ przedziału $[a,b]$ mamy
\[
m(b-a) \leq L(f, P) \leq U(f, P) \leq M(b-a)
\]
a zatem zbiory $\{L(f,P) : P\}$ i $\{U(f,P) : P\}$ są ograniczone. Wynika stąd, że całki górna i dolna są określone przy dowolnej funkcji ograniczonej $f$. 

\defn{} Mówimy, że podział $P^*$ przedziału $[a,b]$ jest rozdrobnieniem (lub zagęszczeniem) podziału $P$ tego przedziału, jeśli $P \subset  P^*$, to znaczy, jeśli każdy punkt przedziału $P$ jest także punktem przedziału $P^*$. Jeśli dane są dwa podziały $P_1$,$P_2$, to podział $P^*=P_1 \cup P_2$ nazywać będziemy ich wspólnym rozdrobnieniem (lub wspólnym zagęszczeniem).
\theorem{Jeśli $P^*$ jest rozdrobnieniem podziału $P$, to}
\[
L(f,P) \leq L(f,P^*) \quad U(f,P) \leq U(f,P^*)
\]
\proof{}Załóżmy wpierw, że $P^*$ zawiera tylko o jeden punkt więcej niż $P$. Niech tym dodatkowym punktem będzie $x^*$ i niech $x_{i-1} < x^* < x_i$, gdzie $x_{i-1}, x_i$ są dwoma kolejnymi punktami przedziału $P$. Przyjmijmy
\[
\omega_1 = \inf\limits_{x_{i-1} \leq x \leq x^*} f(x) \text{,} \quad \omega_2 = \inf\limits_{x^* \leq x \leq x_i} f(x) 
\]
Wtedy $\omega_1 \geq m_i$ i $\omega_2 \geq m_i$, gdzie $m_i = \inf\limits_{x_{i-1} \leq x \leq x_i} f(x)$. Mamy więc
\begin{equation*}
\begin{gathered}
    L(f,P^*)-L(f,P) = \omega_1(x^* - x_{i-1}) + \omega_2(x_i - x^*) - m_i(x_i - x_{i-1}) \\
    = (\omega_1-m_i)(x^*-x_{i-1})+(\omega_2-m_i)(x_i-x^*) \geq 0
\end{gathered}
\end{equation*}
Jeśli $P^*$ zawiera o $k$ punktów więcej niż $P$, to powtarzając powyższe rozumowanie $k$ razy otrzymamy pierwszą nierówność tezy. Dowód drugiej przebiega analogicznie.

\theorem{Jeśli $f$ jest funkcją ograniczoną na przedziale $[a,b]$, to}
\[
\underline{\int_{a}^{b} f(x)dx} \leq \overline{\int_{a}^{b} f(x)dx}
\]
\proof{}Niech $P^*$ będzie wspólnym rozdrobnieniem podziałów $P_1$ i $P_2$ przedziału $[a,b]$. Z Twierdzenia $\ref{theorem:75}$ wynika, że
\[
L(f, P_1) \leq L(f, P^*) \leq U(f, P^*) \leq U(f, P_2)
\]
Stąd $L(f, P_1) \leq U(f, P_2)$. Traktując $P_2$ jako ustalone i obliczając kres górny ze względu na wszystkie podziały $P_1$, wobec poprzedniej nierówności otrzymujemy
\[
\underline{\int_{a}^{b} f(x)dx} \leq U(f, P_2)
\]
Przechodząc do kresu dolnego ze względu na wszystkie podziały $P_2$ otrzymujemy tezę dowodzonego twierdzenia.

Udowodnimy teraz dwa kryteria całkowalności funkcji w sensie Riemanna. W oparciu o drugie z tych kryteriów podamy równoważną definicję całki w sensie Riemanna.

\theorem{Na to, aby ograniczona funkcja $f$ była całkowalna w sensie Riemanna na przedziale $[a,b]$ potrzeba i wystarcza, aby dla dowolnego $\epsilon > 0$ istniał taki podział $P$ przedziału $[a,b]$, że}
\begin{equation}\label{eq:38}
U(f, P) - L(f, P) \leq \epsilon
\end{equation}
\proof{}Załóżmy wpierw, że funkcja $f$ jest całkowalna w sensie Riemanna na przedziale $[a,b]$. Wówczas dla każdego danego $\epsilon > 0$ istnieje taki podział $P$ przedziału $[a,b]$, że nierówność
\begin{equation*}
\begin{gathered}
    |R - \int_{a}^{b} f(x)dx| < \frac{\epsilon}{2} \text{, czyli} \\
    \int_{a}^{b} f(x)dx - \frac{\epsilon}{2} < R < \int_{a}^{b} f(x)dx + \frac{\epsilon}{2}
\end{gathered}
\end{equation*}
jest spełniona przy dowolnym wyborze punktów $\xi_i$ w każdym z przedziałów podziału. Ponieważ sumy Darboux są --- przy danym podziale przedziału --- odpowiednio kresem górnym i dolnym sum całkowych, zatem spełniają one nierówności
\[
\int_{a}^{b}f(x)dx - \frac{\epsilon}{2} \leq L(f,P) \leq U(f, P) \leq \int_{a}^{b}f(x)dx + \frac{\epsilon}{2}
\]
a więc $U(f, P) - L(f, P) < \epsilon$.
Załóżmy teraz, że (\ref{eq:38}) zachodzi. Dla dowolnego podziału $P$ mamy
\[
L(f, P) \leq \underline{\int_{a}^{b}f(x)dx} \leq \overline{\int_{a}^{b}f(x)dx} \leq U(f, P)
\]
Jeśli $U(f, P) - L(f, P) < \epsilon$, to wówczas
\[
0 \leq \overline{\int_{a}^{b}f(x)dx} - \underline{\int_{a}^{b}f(x)dx} < \epsilon
\]
Z dowolności $\epsilon > 0$ wynika, że $\underline{\int_{a}^{b}f(x)dx} = \overline{\int_{a}^{b}f(x)dx}$. Oznaczając ponadto $\underline{\int_{a}^{b}f(x)dx} = \overline{\int_{a}^{b}f(x)dx} = I$ mamy $L(f, P) \leq I \leq U(f, P)$.
Ustalmy $\epsilon > 0$ i niech $P$ bedzie danym podziałem przedziału $[a,b]$, dla którego (\ref{eq:38}) zachodzi. Jeśli przez $R$ oznaczymy jedną z wartości sum Riemanna odpowiadającej podziałowi $P$, to
\[
L(f,P) \leq R \leq U(f, P)
\]
Ponieważ liczby $R$ oraz $I$ znajdują się w przedziale $[L(f, P), U(f, P)]$, zatem
\[
|R - I| \leq \epsilon
\]
Wobec Twierdzenia $\ref{theorem:74}$ oraz Definicji $\ref{def:70}$ wnioskujemy, że $I = \int_{a}^{b}f(x)dx$

Jako wniosek z powyższego twierdzenia otrzymujemy następujące

\theorem{Na to by ograniczona funkcja $f$ byla całkowalna w sensie Riemanna na przedziale $[a,b]$ potrzeba i wystarcza, by}
\eq{}
\begin{equation}
\underline{\int_{a}^{b}fdx} = \overline{\int_{a}^{b}fdx}
\end{equation}
\proof{}W dowodzie Twierdzenia $\ref{theorem:76}$ pokazaliśmy, że (\ref{eq:38}) implikuje (\ref{eq:39}). Załóżmy teraz, że (\ref{eq:39}) zachodzi. Dla danej liczby $\epsilon > 0$ istnieją podziały $P_1$ i $P_2$ przedziału $[a,b]$ takie, że
\[
    \underline{\int_{a}^{b}fdx} - \frac{\epsilon}{2} < L(f, P_{1}), \quad U(f, P_{2}) < \overline{\int_{a}^{b}fdx} + \frac{\epsilon}{2}
\]
Jeśli podział $P$ jest wspólnym rozdrobniemiem podziałów $P_{1}$ i $P_{2}$, to na mocy Twierdzenia $\ref{theorem:74}$ otrzymujemy
\[
    U(f, P) \leq U(f, P_2) < \overline{\int_{a}^{b}fdx} + \frac{\epsilon}{2} = \underline{\int_{a}^{b}fdx} + \frac{\epsilon}{2} < L(f, P_1) + \epsilon \leq L(f, P) + \epsilon
\]
Stąd $U(f, P) - L(f, P) \leq \epsilon$, a zatem warunek (\ref{eq:38}) jest spełniony. Wobec Twierdzenia $\ref{theorem:76}$ dowód jest zakończony.

\defn{} Mówimy, że ograniczona funkcja $f$ jest całkowalna w sensie Riemanna, jeśli
\[
    \overline{\int_{a}^{b}fdx} = \underline{\int_{a}^{b}fdx}
\]
Wspólną wartość określoną powyższą równością nazywamy całką Riemanna funkcji $f$ na przedziale $[a,b]$.

Zbadamy teraz całkowalność w sensie Riemanna pewnych klas funkcji.

\theorem{Funkcja ciągła na przedziale $[a,b]$ jest na tym przedziale całkowalna w sensie Riemanna.}

\proof{}Funkcja $f$ jest jednostajnie ciągła na $[a,b]$ (por. Tw. 51), a zatem dla dowolnego $\epsilon > 0$ istnieje $\delta > 0$ taka, 
że $|f(x) - f(t)| < \frac{\epsilon}{b - a}$ dla wszystkich $x, t \in [a, b]$, dla których $|x - t| < \delta$. Niech $P$ będzie podziałem przedziału $[a,b]$,
dla którego $\delta(P) < \delta$. Wtedy mamy $M_i - m_i \leq \frac{\epsilon}{b-a}$ dla $i = 1,\ldots,n$ i wobec tego
\[
    U(f, P) - L(f, P) = \sum_{i=1}^{n}(M_i - m_i)\Delta x_i \leq \frac{\epsilon}{b-a}\sum_{i=1}^{n}\Delta x_i = \epsilon
\]
Na mocy Twierdzenia $\ref{theorem:76}$ funkcja $f$ jest całkowalna w sensie Riemanna na $[a,b]$.

Udowodnimy teraz następujące uogólnienie powyższego twierdzenia.

\begin{theorem}
    {Jeśli $f$ jest funkcją ograniczoną i mającą tylko skończoną liczbę punktów nieciągłości na przedziale $[a,b]$, to jest ona całkowalna w sensie Riemanna na tym przedziale.}
\end{theorem}

\begin{proof}
    Ponieważ funkcja $f$ jest ograniczona, więc istnieją liczby rzeczywiste $m, M$ takie, że $m \leq f(x) \leq M$
    dla wszystkich $x \in [a,b]$. Załóżmy, że $f$ ma $k$ punktów nieciągłości na przedziale $[a,b]$. 
    Weźmy dowolne $\epsilon > 0$ i $\delta_1 < \frac{\epsilon}{8(M-m)k}$ (oczywiście $M \neq m$).
    Rozważmy przedziały otwarte $(x_l - \delta_1, x_l + \delta_1)$, $l = 1, \ldots, k$, gdzie $x_l$ są punktami
    nieciągłości funkcji $f$. Dopełnienie sumy tych przedziałów do przedziału $[a,b]$ składa się ze skończonej
    liczby przedziałów domkniętych, na których funkcja $f$ jest ciągła, a więc i jednostajnie ciągła. Ponieważ tych przedziałów
    jest skończenie wiele, więc dla danego $\epsilon > 0$ istnieje liczba $\delta_2 > 0$ taka, że dla dowolnych
    punktów $x, t$ należacych do jednego z tych przedziałów, na których funkcja $f$ jest ciągła i spełniająca nierówność $|x-t| < \delta_2$ mamy
    $|f(x)-f(t)|<\frac{\epsilon}{2(b-a)}$. Weźmy teraz liczbe $\delta = \min{(\delta_1, \delta_2)}$.
    Niech $P = \{x_0,\ldots,x_n\}$ będzię dowolnym podziałem przedziału $[a,b]$, dla którego $\delta(P) < \delta$. Ponadto
    rozbijmy zbiór indeksów $\{1, \ldots, n\}$ na dwa rozłącznę zbiory $A$ i $B$ w następujący sposób: do zbioru $A$ zaliczymy te liczby $i$, 
    dla których przedział $[x_{i-1}, x_i]$ nie ma punktów wspólnych z żadnym z skontruowanych powyżej otoczeń punktów $x_l$, $l=1,\ldots,k$, a do zbioru $B$
    pozostałe przedziały powstające z podziału $P$ przedziału $[a,b]$. Wówczas
    \[
        U(f, P) - L(f, P) = \sum_{i=1}^{n}(M_i - m_i)\Delta x_i = \sum_{i \in A}(M_i - m_i)\Delta x_i + \sum_{i \in B}(M_i - m_i)\Delta x_i
    \]
    Ponadto
    \[
        \sum_{i \in A}(M_i - m_i)\Delta x_i \leq \frac{\epsilon}{2(b-a)}\sum_{i \in A}\Delta x_i \leq \frac{\epsilon}{2(b-a)}(b-a)=\frac{\epsilon}{2}
    \]
    Suma długości podprzedziałów przedziału $[a,b]$ indeksowanych przez liczby ze zbioru $B$ jest nie większa niż
    \[
        (\delta + 2\delta_i+\delta)k < 4 \frac{\epsilon}{8(M-m)k}k = \frac{\epsilon}{2(M-m)}
    \]
    Dlatego
    \[
        \sum_{i \in B}(M_i - m_i)\Delta x_i \leq (M-m)\sum_{i \in B}\Delta x_i < (M-m)\frac{\epsilon}{2(M-m)}=\frac{\epsilon}{2}
    \]
    Dla podziału $P$ o średnicy mniejszej niż $\delta$ otrzymujemy zatem
    \[
        U(f, P) - L(f, P) = \sum_{i=1}^{n}(M_i - m_i)\Delta x_i < \epsilon
    \]
    co kończy dowód.
\end{proof}

\begin{uwaga}
    Twierdzenie $\ref{theorem:78}$ można istotnie uogólnić. Mianowicie dowodzi się, że jeśli $f$ jest ograniczoną funkcją
    na przedziale $[a,b]$, to jest ona całkowalna w sensie Riemanna na tym przedziale wtedy i tylko wtedy, gdy
    jest ona ciągła prawie wszędzie ma $[a,b]$, to znaczy zbiór punktów nieciągłości funkcji $f$ ma miarę Lebesgue'a równą zeru.
    (por. [7], s. 270).
    Przykładów takich funkcji dostarcza następujące
\end{uwaga}

\theorem{Funkcja monotoniczna na przedziale $[a,b]$ jest na tym przedziale całkowalna w sensie Riemanna.}

\proof{}
Załóżmy, że $f$ jest funkcją niemalejącą. Niech będzie dane dowolne $\epsilon > 0$. Weźmy podział $P$ przedziału $[a,b]$ na $n$ równych części o długości
$\frac{b-a}{n}$. Ponieważ $f$ jest niemalejącą zatem $M_i = f(x_i)$ oraz $m_i = f(x_{i-1})$ dla $i = 1, \ldots, n$. Mamy więc
\[
    U(f, P) - L(f, P) = \sum_{i=1}^{n}(f(x_i) - f(x_{i-1}))\frac{b-a}{n} = (f(b)-f(a))\frac{b-a}{n}
\]
Biorąc $n$ tak duże, aby $(f(b) - f(a))\frac{b-a}{n} < \epsilon$ i stosując twierdzenie $\ref{theorem:76}$ otrzymujemy tezę.
W przypadku funkcji nierosnącej dowód jest analogiczny.

\begin{theorem}
    {Jeśli $f$ jest całkowalna w sensie Riemanna na przedziale $[a,b]$, $m \leq f(x) \leq M$ dla $x \in [a,b]$ oraz $\phi$ jest funkcją ciągłą
    na $[m, M]$, to funkcja złożona $h = \phi \circ f$ jest R-całkowalna na $[a,b]$.}
\end{theorem}

\begin{proof}
    Ustalmy $\epsilon > 0$. Ponieważ funkcja $\phi$ jest jednostajnie ciągła na $[M,m]$, więc
    istnieje $\delta > 0$ taka, że $\delta < \epsilon$ i $|\phi(s) - \phi(t)| < \epsilon$, jeśli $|s-t| < \delta$.
    Ponieważ $f$ jest R-całkowalna na $[a,b]$, więc istnieje podział $P = \{x_0, \ldots, x_n\}$
    przedziału $[a,b]$ taki, że $U(f, P) - L(f, P) < \delta^2$. Niech
    \[
        M_i=\sup\limits_{x_{i-1}\leq x \leq x_i}f(x), \quad m_i=\inf\limits_{x_{i-1}\leq x \leq x_i}f(x),
    \]
    \[
        M_i^*=\sup\limits_{x_{i-1}\leq x \leq x_i}h(x), \quad m_i^*=\inf\limits_{x_{i-1}\leq x \leq x_i}h(x)
    \]
    dla $i = 1, \ldots, n$. Podzielmy zbiór $\{1, \ldots, n\}$ na dwa rozłącznę zbiory $A$ i $B$ w taki sposób, że
    $i \in A$, jeśli $M_i - m_i < \delta$ oraz $i \in B$ w przypadku przeciwnym. Wówczas wobec powyższego wyboru $\delta$ mamy
    $M_i^* - m_i^* < \epsilon$ dla $i \in A$. Natomiast dla $i \in B$ mamy $M_i^* - m_i^* \leq 2K$, gdzie $K=\sup{\{|\phi(t)|:m \leq t \leq M\}}$.
    Stąd otrzymujemy
    \[
        \delta \sum_{i \in B}(x_i - x_{i-1}) \leq \sum_{i \in B}(M_i - m_i)(x_i - x_{i-1}) < \delta^2, \quad \text{zatem} \sum_{i \in B}(x_i - x_{i-1}) < \delta
    \]
    Mamy więc
    \[
        U(h, P) - L(h, P) = \sum_{i \in A}(M_i^* - m_i^*)(x_i - x_{i-1}) + \sum_{i \in B}(M_i^* - m_i^*)(x_i - x_{i-1})
    \]
    a zatem
    \[
        U(h, P) - L(h, P) \leq \epsilon(a + b + 2K)
    \]
    Ponieważ $\epsilon$ było dowolne, zatem na mocy twierdzenia $\ref{theorem:76}$ funkcja $h$ jest R-całkowalna.
\end{proof}

Następujące twierdzenie opisuję związek całki Riemanna z operacjami arytmetycznymi. 

\begin{theorem}
    {Jeśli funkcje $f$ i $g$ są R-całkowalne na przedziale $[a,b]$, to również R-całkowalne są
    funkcje $f+g$, $\lambda f$ ($\lambda$ jest dowolną stałą rzeczywistą) i $fg$ oraz prawdziwe są równości:}
\end{theorem}
\begin{eq}
    \begin{equation}
        \int_{a}^{b} (f+g) (x) dx = \int_{a}^{b} f (x) dx + \int_{a}^{b} g (x) dx,
    \end{equation}
\end{eq}
\begin{eq}
    \begin{equation}
        \int_{a}^{b} (\lambda f) (x) dx = \lambda \int_{a}^{b} f (x) dx
    \end{equation}
\end{eq}

\proof{} Jest jasne, że dla dowolnego $R \in \mathfrak{R}(f+g, P)$ mamy $R = R_f + R_g$, gdzie $R_f \in \mathfrak{R}(f, P)$, $R_g \in \mathfrak{R}(g, P)$.
Niech $I_1 = \int_{a}^{b}f(x)dx$, $I_2 = \int_{a}^{b}f(x)dx$ oraz $I = I_1 + I_2$. Mamy
\[
    \forall_{\epsilon > 0} \exists_{\delta > 0} \forall_P \forall_{R_f \in \mathfrak{R}(f, P)} \delta(P) < \delta \implies |R_f - I_1| < \frac{\epsilon}{2} \quad \text{oraz}
\]
\[
    \forall_{\epsilon > 0} \exists_{\delta > 0} \forall_P \forall_{R_g \in \mathfrak{R}(g, P)} \delta(P) < \delta \implies |R_g - I_2| < \frac{\epsilon}{2} \quad \text{Stąd}
\]
\[
    \forall_{\epsilon > 0} \exists_{\delta > 0} \forall_P \forall_{R \in \mathfrak{R}(f+g, P)} \delta(P) < \delta \implies |R - I| \leq |R_f - I_1| + |R_g - I_2| < \epsilon
\]
Wobec powyższego jest jasne, że funkcja $f + g$ jest R-całkowalna na przedziale $[a,b]$ oraz, że spełniony jest wzór $\ref{eq:40}$.
Dowód wzoru $\ref{eq:41}$ jest analogiczny.

\noindent
Dalej przyjmując $\phi(t) = t^2$ oraz stosując do $\phi$ poprzednie twierdzenie (\ref{theorem:81}) otrzymujemy
R-całkowalność funkcji $f^2$.

\noindent
R-całkowalność iloczynu funkcji $fg$ wynika z tożsamości
\[
    fg = \frac{1}{4}[{(f+g)}^2 - {(f-g)}^2].
\]

\end{justify}
\end{document}