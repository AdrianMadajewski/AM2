\documentclass[leqno]{article}
\usepackage{graphicx} % Required for inserting images
\usepackage[polish]{babel} % Language support for Polish
\usepackage[utf8]{inputenc} % Input encoding
\usepackage[T1]{fontenc}
\usepackage{ragged2e}
\usepackage{amssymb} % Load before mathrsfs
\usepackage{mathrsfs} % Pakiet mathrsfs dostarcza komendę \mathscr
\usepackage{hyperref} % for clickable links
\usepackage{amsmath} % Load after amssym
\usepackage{lmodern}
\usepackage{array, makecell}

\setlength{\parindent}{0pt} % Remove indentation globally
\setlength{\parskip}{10pt}

\setcounter{equation}{35}

\title{\Huge{Analiza Matematyczna 2}}
\author{Adrian Madajewski}
\date{Semestr II}

\DeclareMathOperator{\tg}{\text{tg}}
\DeclareMathOperator{\ctg}{\text{ctg}}
\DeclareMathOperator{\arctg}{\text{arctg}}
\DeclareMathOperator{\arcctg}{\text{arcctg}}

% Define a custom command for fraction without line
\newcommand{\fracnoline}[2]{\genfrac{}{}{0pt}{}{#1}{#2}}

% Twierdzenie
\newcounter{thcounter}
\setcounter{thcounter}{73}
\newcommand{\theorem}[1]{\noindent\refstepcounter{thcounter}\textbf{Twierdzenie \thethcounter. }\textit{#1}\label{theorem:\thethcounter}}

% Lemat
\newcounter{lematcounter}
\setcounter{lematcounter}{1}
\newcommand{\lemat}[1]{\noindent\refstepcounter{lematcounter}\textbf{Lemat \thelematcounter. }\textit{#1}\label{theorem:\thelematcounter}}

% Wniosek
\newcounter{wniosekcounter}
\setcounter{wniosekcounter}{17}
\newcommand{\wniosek}[1]{\noindent\refstepcounter{wniosekcounter}\textbf{Wniosek \thewniosekcounter. }\textit{#1}\label{wniosek:\thewniosekcounter}}

% Definicja
\newcounter{defcounter}
\setcounter{defcounter}{67}
\newcommand{\defn}{\noindent\refstepcounter{defcounter}\textbf{Definicja \thedefcounter. }\label{def:\thedefcounter}}

% Uwaga
\newcounter{uwagacounter}
\setcounter{uwagacounter}{37}
\newcommand{\uwaga}{\noindent\refstepcounter{uwagacounter}\textbf{Uwaga \theuwagacounter. }\label{uwaga:\theuwagacounter}}

% Przyklad
\newcounter{excounter}
\setcounter{excounter}{24}
\newcommand{\ex}{\noindent\refstepcounter{excounter}\textbf{Przykład \theexcounter. }\label{ex:\theexcounter}}

% Dowód
\newcommand{\proof}{\noindent\textbf{Dowód. }}

% Equation
\newcounter{eqcounter}
\setcounter{eqcounter}{38}
\newcommand{\eq}{\refstepcounter{eqcounter}\label{eq:\theeqcounter}}

\begin{document}

\pagenumbering{arabic}

\maketitle

\begin{center}
    Niniejszy plik jest w całości bazowany na wykładach \\
    \vspace{10pt}
    prof. dr hab. Dariusza Bugajewskiego \\
    \vspace{10pt}
    z przedmiotu \\
    \vspace{10pt}
    Analiza Matematyczna 2 \\
    \vspace{10pt}
    na Uniwersytecie im. Adama Mickiewicza w Poznaniu
    \includegraphics[width=0.8\textwidth]{uam_logo.pdf}
\end{center}
\newpage

\begin{justify}

\setcounter{section}{7}
\section{Całka Riemanna}
\subsection{Definicja i podstawowe własności całki}
\defn{}{} Niech $[a,b]$ będzie danym przedziałem. Przez podział $P$ przedziału $[a,b]$ będziemy nazywali skończony zbiór punktów \(x_0, x_1, \ldots, x_n\), gdzie
\[
a = x_0 < x_1 < \cdots < x_n = b
\]
Będziemy pisać $\Delta x_i = x_i - x_{i-1}$ $(i = 1, \ldots , n)$. Długość największego z odcinków $[x_{i-1}, x_i]$ nazywać będziemy średnicą podziału $P$ i oznaczamy ją symbolem $\delta (P)$. $\delta (P)= \max\limits_{1 \leqslant i \leqslant n} \Delta x_i$. Niech $f$ będzie ograniczoną funkcją rzeczywistą określoną na $[a,b]$. W każdym z przedziałów $[x_{i-1}, x_i]$ wybierzmy dowolny punkt $\xi_{i}$ $(i=1, \ldots n)$ i utwórzmy sumę $R = \sum_{i=1}^{n} f(\xi_{i})\Delta x_{i}$. Sumę te nazywamy sumą Riemanna odpowiadającą podziałowi $P$, przy ustalonym wyborze punktów $\xi_i$. Przez $\mathfrak{R}(f,P)$ oznaczać będziemy zbiór wszystkich możliwych sum Riemanna odpowiadających podziałowi $P$. Utwórzmy teraz ciąg $(P_k)$ podziałów przedziału $[a, b]$:
\[
a = x_0^{(k)} < x_1^{(k)} < \cdots < x_{n(k)}^{(k)} = b;
\]
\[
\Delta_i^{(k)}=x_i^{(k)}-x_{i-1}^{(k)};
\]
\[
\delta (P_k)=\max_{1 \leqslant i \leqslant n(k)} \Delta x_i^{(k)}, k = 1,2, \ldots
\]
Ciąg $(P_k)$ nazywamy ciągiem normalnym podziałów, jeśli $\delta(P_k)\rightarrow 0$  przy $k \rightarrow \infty$. Oznaczmy przez $\mathfrak{R}(f,P_k)$ zbiór wszystkich sum Riemanna odpowiadających podziałowi $P_k$.

\noindent
\defn{} Jeśli dla dowolnego ciągu normalnego podziałów $(P_k)$ i dla dowolnych sum Riemanna $R_k \in \mathfrak{R}(f,P_k)$ istnieje skończona granica $I = \lim\limits_{k \rightarrow \infty}R_k$, to tę granicę nazywamy całką Riemanna funkcji $f$ na przedziale $[a,b]$ i oznaczamy ją symbolem 
\[
\begin{aligned}
    \int_{a}^{b} fdx \text{ lub } \int_{a}^{b} f(x)dx
\end{aligned}
\]
O funkcji $f$ mówimy wówczas, że jest całkowalna w sensie Riemanna na przedziale $[a,b]$, lub że jest ona R-całkowalna na tym przedziale. 

Powyższą definicję mozna sformułować w następujący równoważny sposób.

\noindent
\defn{} Mówimy, że funkcja $f$ jest całkowalna w sensie Riemanna na przedziale $[a,b]$, jeśli istnieje liczba $I \in \mathbb{R}$ taka, że
\[
\forall_{\varepsilon>0} \exists_{\delta > 0} \forall_P \forall_{R \in \mathfrak{R}(f, P)}\delta(P) < \delta \implies |R - I| < \varepsilon
\]
Piszemy wówczas $I = \int_{a}^{b}f(x)dx = \lim\limits_{\delta(P) \rightarrow 0}R$.

Równoważność definicji $\ref{def:69}$ i $\ref{def:70}$ można pokazać analogicznie jak w dowodzie twierdzenia 35. 

\ex{} (a) Funkcja stała $f(x)=c, c \in \mathbb{R}, x \in [a,b]$ jest całkowalna w sensie Riemanna na tym przedziale. Niech $P$ będzie dowolnym podziałem przedziału $[a,b]$:
\[
a = x_0 < x_1 < \cdots < x_n = b
\]
Dowolna suma Riemanna odpowiadającą podziałowi $P$ ma postać:
\[
R = \sum_{i=1}^{n}f(\xi_i)(x_i-x_{i-1}) = \sum_{i=1}^{n}c(x_i - x_{i-1})=c(b-a),
\]
\[
    (\xi_i \in [x_{i-1},x_i], i=1, \ldots, n)
\]
Stąd wynika, że $\int_{a}^{b} f(x)dx = c(b-a)$. 

\noindent
(b) Roważmy ponownie funkcję Dirichleta z Przykładu 18 (a), zawężoną do przedziału $[a,b]$. Dla każdego podziału $P$ przedziału $[a,b]$ można utworzyć sumę Riemanna równą zeru, jeśli wszystkie punkty $\xi_i$ będą liczbami niewymiernymi, lub równą $(b-a)$, jeśli wszystkie punkty $\xi_i$ będą liczbami wymiernymi. Jest więc jasne, że dla każdego ciągu normalnego podziałów $(P_k)$ granica $\lim\limits_{k \rightarrow \infty}R_k$, gdzie $R_k \in \mathfrak{R}(f, P_k)$, $k \in \mathbb{N}$, nie istnieje. 

\defn{}\label{def:sumy_dolne_gorne} Niech $f$ będzie ograniczoną funkcją rzeczywistą określona na $[a,b]$. Każdemu podziałowi $P$ przedziału $[a,b]$ odpowiadają liczby:
\[
M_i = \sup\limits_{x_{i-1} \leqslant x \leqslant x_i} f(x) \quad m_i = \inf\limits_{x_{i-1} \leqslant x \leqslant x_i} f(x)
\]
\[
U(f, P) = \sum_{i=1}^{n}M_i\Delta x_i \quad L(f, P) = \sum_{i=1}^{n}m_i\Delta x_i
\]
Liczby $U(f,P)$ i $L(f, P)$ nazywamy odpowiednio sumą górną i dolną lub sumami Darboux funkcji $f$ przy podziale $P$ przedziału $[a,b]$. Dalej,

\begin{equation}\label{eq:36}
    \overline{\int_{a}^{b} f(x)dx} = \inf_P U(f, P),
\end{equation}

\begin{equation}\label{eq:37}
    \underline{\int_{a}^{b} f(x)dx} = \sup_P L(f, P),
\end{equation}
gdzie kres górny i dolny są brane ze względu na wszystkie podziały $P$ przedziału $[a,b]$. Lewe strony równości (\ref{eq:36}) i (\ref{eq:37}) nazywają się odpowiednio górną i dolną całką Darboux funkcji $f$ na przedziale $[a,b]$.

Ponieważ funkcja $f$ jest ograniczona, więc istnieją liczby rzeczywistę $m$ i $M$ takie, że
\[
m \leqslant f(x) \leqslant M \quad \text{dla} \quad x \in [a,b] 
\]
Oznacza to, że przy dowolnym podziale $P$ przedziału $[a,b]$ mamy
\[
m(b-a) \leqslant L(f, P) \leqslant U(f, P) \leqslant M(b-a)
\]
a zatem zbiory $\{L(f,P) : P\}$ i $\{U(f,P) : P\}$ są ograniczone. Wynika stąd, że całki górna i dolna są określone przy dowolnej funkcji ograniczonej $f$. 

\defn{} Mówimy, że podział $P^*$ przedziału $[a,b]$ jest rozdrobnieniem (lub zagęszczeniem) podziału $P$ tego przedziału, jeśli $P \subset  P^*$, to znaczy, jeśli każdy punkt przedziału $P$ jest także punktem przedziału $P^*$. Jeśli dane są dwa podziały $P_1$,$P_2$, to podział $P^*=P_1 \cup P_2$ nazywać będziemy ich wspólnym rozdrobnieniem (lub wspólnym zagęszczeniem).

\theorem{Jeśli $P^*$ jest rozdrobnieniem podziału $P$, to}
\[
L(f,P) \leqslant L(f,P^*) \quad U(f,P) \leqslant U(f,P^*)
\]
\proof{}Załóżmy wpierw, że $P^*$ zawiera tylko o jeden punkt więcej niż $P$. Niech tym dodatkowym punktem będzie $x^*$ i niech $x_{i-1} < x^* < x_i$, gdzie $x_{i-1}, x_i$ są dwoma kolejnymi punktami przedziału $P$. Przyjmijmy
\[
\omega_1 = \inf\limits_{x_{i-1} \leqslant x \leqslant x^*} f(x) \text{,} \quad \omega_2 = \inf\limits_{x^* \leqslant x \leqslant x_i} f(x) 
\]
Wtedy $\omega_1 \geqslant m_i$ i $\omega_2 \geqslant m_i$, gdzie $m_i = \inf\limits_{x_{i-1} \leqslant x \leqslant x_i} f(x)$. Mamy więc
\begin{equation*}
\begin{gathered}
    L(f,P^*)-L(f,P) = \omega_1(x^* - x_{i-1}) + \omega_2(x_i - x^*) - m_i(x_i - x_{i-1}) \\
    = (\omega_1-m_i)(x^*-x_{i-1})+(\omega_2-m_i)(x_i-x^*) \geqslant 0
\end{gathered}
\end{equation*}
Jeśli $P^*$ zawiera o $k$ punktów więcej niż $P$, to powtarzając powyższe rozumowanie $k$ razy otrzymamy pierwszą nierówność tezy. Dowód drugiej przebiega analogicznie.

\theorem{Jeśli $f$ jest funkcją ograniczoną na przedziale $[a,b]$, to}
\[
\underline{\int_{a}^{b} f(x)dx} \leqslant \overline{\int_{a}^{b} f(x)dx}
\]
\proof{}Niech $P^*$ będzie wspólnym rozdrobnieniem podziałów $P_1$ i $P_2$ przedziału $[a,b]$. Z Twierdzenia $\ref{theorem:75}$ wynika, że
\[
L(f, P_1) \leqslant L(f, P^*) \leqslant U(f, P^*) \leqslant U(f, P_2)
\]
Stąd $L(f, P_1) \leqslant U(f, P_2)$. Traktując $P_2$ jako ustalone i obliczając kres górny ze względu na wszystkie podziały $P_1$, wobec poprzedniej nierówności otrzymujemy
\[
\underline{\int_{a}^{b} f(x)dx} \leqslant U(f, P_2)
\]
Przechodząc do kresu dolnego ze względu na wszystkie podziały $P_2$ otrzymujemy tezę dowodzonego twierdzenia.

Udowodnimy teraz dwa kryteria całkowalności funkcji w sensie Riemanna. W oparciu o drugie z tych kryteriów podamy równoważną definicję całki w sensie Riemanna.

\theorem{Na to, aby ograniczona funkcja $f$ była całkowalna w sensie Riemanna na przedziale $[a,b]$ potrzeba i wystarcza, aby dla dowolnego $\varepsilon > 0$ istniał taki podział $P$ przedziału $[a,b]$, że}
\begin{equation}\label{eq:38}
U(f, P) - L(f, P) \leqslant \varepsilon
\end{equation}
\proof{}Załóżmy wpierw, że funkcja $f$ jest całkowalna w sensie Riemanna na przedziale $[a,b]$. Wówczas dla każdego danego $\varepsilon > 0$ istnieje taki podział $P$ przedziału $[a,b]$, że nierówność
\begin{equation*}
\begin{gathered}
    |R - \int_{a}^{b} f(x)dx| < \frac{\varepsilon}{2} \text{, czyli} \\
    \int_{a}^{b} f(x)dx - \frac{\varepsilon}{2} < R < \int_{a}^{b} f(x)dx + \frac{\varepsilon}{2}
\end{gathered}
\end{equation*}
jest spełniona przy dowolnym wyborze punktów $\xi_i$ w każdym z przedziałów podziału. Ponieważ sumy Darboux są --- przy danym podziale przedziału --- odpowiednio kresem górnym i dolnym sum całkowych, zatem spełniają one nierówności
\[
\int_{a}^{b}f(x)dx - \frac{\varepsilon}{2} \leqslant L(f,P) \leqslant U(f, P) \leqslant \int_{a}^{b}f(x)dx + \frac{\varepsilon}{2}
\]
a więc $U(f, P) - L(f, P) < \varepsilon$.
Załóżmy teraz, że (\ref{eq:38}) zachodzi. Dla dowolnego podziału $P$ mamy
\[
L(f, P) \leqslant \underline{\int_{a}^{b}f(x)dx} \leqslant \overline{\int_{a}^{b}f(x)dx} \leqslant U(f, P)
\]
Jeśli $U(f, P) - L(f, P) < \varepsilon$, to wówczas
\[
0 \leqslant \overline{\int_{a}^{b}f(x)dx} - \underline{\int_{a}^{b}f(x)dx} < \varepsilon
\]
Z dowolności $\varepsilon > 0$ wynika, że $\underline{\int_{a}^{b}f(x)dx} = \overline{\int_{a}^{b}f(x)dx}$. Oznaczając ponadto $\underline{\int_{a}^{b}f(x)dx} = \overline{\int_{a}^{b}f(x)dx} = I$ mamy $L(f, P) \leqslant I \leqslant U(f, P)$.
Ustalmy $\varepsilon > 0$ i niech $P$ bedzie danym podziałem przedziału $[a,b]$, dla którego (\ref{eq:38}) zachodzi. Jeśli przez $R$ oznaczymy jedną z wartości sum Riemanna odpowiadającej podziałowi $P$, to
\[
L(f,P) \leqslant R \leqslant U(f, P)
\]
Ponieważ liczby $R$ oraz $I$ znajdują się w przedziale $[L(f, P), U(f, P)]$, zatem
\[
|R - I| \leqslant \varepsilon
\]
Wobec Twierdzenia $\ref{theorem:74}$ oraz Definicji $\ref{def:70}$ wnioskujemy, że $I = \int_{a}^{b}f(x)dx$

Jako wniosek z powyższego twierdzenia otrzymujemy następujące

\theorem{Na to by ograniczona funkcja $f$ byla całkowalna w sensie Riemanna na przedziale $[a,b]$ potrzeba i wystarcza, by}
\eq{}
\begin{equation}
\underline{\int_{a}^{b}fdx} = \overline{\int_{a}^{b}fdx}
\end{equation}
\proof{}W dowodzie Twierdzenia $\ref{theorem:76}$ pokazaliśmy, że (\ref{eq:38}) implikuje (\ref{eq:39}). Załóżmy teraz, że (\ref{eq:39}) zachodzi. Dla danej liczby $\varepsilon > 0$ istnieją podziały $P_1$ i $P_2$ przedziału $[a,b]$ takie, że
\[
    \underline{\int_{a}^{b}fdx} - \frac{\varepsilon}{2} < L(f, P_{1}), \quad U(f, P_{2}) < \overline{\int_{a}^{b}fdx} + \frac{\varepsilon}{2}
\]
Jeśli podział $P$ jest wspólnym rozdrobniemiem podziałów $P_{1}$ i $P_{2}$, to na mocy Twierdzenia $\ref{theorem:74}$ otrzymujemy
\[
    U(f, P) \leqslant U(f, P_2) < \overline{\int_{a}^{b}fdx} + \frac{\varepsilon}{2} = \underline{\int_{a}^{b}fdx} + \frac{\varepsilon}{2} < L(f, P_1) + \varepsilon \leqslant L(f, P) + \varepsilon
\]
Stąd $U(f, P) - L(f, P) \leqslant \varepsilon$, a zatem warunek (\ref{eq:38}) jest spełniony. Wobec Twierdzenia $\ref{theorem:76}$ dowód jest zakończony.

\defn{} Mówimy, że ograniczona funkcja $f$ jest całkowalna w sensie Riemanna, jeśli
\[
    \overline{\int_{a}^{b}fdx} = \underline{\int_{a}^{b}fdx}
\]
Wspólną wartość określoną powyższą równością nazywamy całką Riemanna funkcji $f$ na przedziale $[a,b]$.

Zbadamy teraz całkowalność w sensie Riemanna pewnych klas funkcji.

\theorem{Funkcja ciągła na przedziale $[a,b]$ jest na tym przedziale całkowalna w sensie Riemanna.}

\proof{}Funkcja $f$ jest jednostajnie ciągła na $[a,b]$ (por. Tw. 51), a zatem dla dowolnego $\varepsilon > 0$ istnieje $\delta > 0$ taka, 
że $|f(x) - f(t)| < \frac{\varepsilon}{b - a}$ dla wszystkich $x, t \in [a, b]$, dla których $|x - t| < \delta$. Niech $P$ będzie podziałem przedziału $[a,b]$,
dla którego $\delta(P) < \delta$. Wtedy mamy $M_i - m_i \leqslant \frac{\varepsilon}{b-a}$ dla $i = 1,\ldots,n$ i wobec tego
\[
    U(f, P) - L(f, P) = \sum_{i=1}^{n}(M_i - m_i)\Delta x_i \leqslant \frac{\varepsilon}{b-a}\sum_{i=1}^{n}\Delta x_i = \varepsilon
\]
Na mocy Twierdzenia $\ref{theorem:76}$ funkcja $f$ jest całkowalna w sensie Riemanna na $[a,b]$.

Udowodnimy teraz następujące uogólnienie powyższego twierdzenia.

\begin{theorem}
    {Jeśli $f$ jest funkcją ograniczoną i mającą tylko skończoną liczbę punktów nieciągłości na przedziale $[a,b]$, to jest ona całkowalna w sensie Riemanna na tym przedziale.}
\end{theorem}

\begin{proof}
    Ponieważ funkcja $f$ jest ograniczona, więc istnieją liczby rzeczywiste $m, M$ takie, że $m \leqslant f(x) \leqslant M$
    dla wszystkich $x \in [a,b]$. Załóżmy, że $f$ ma $k$ punktów nieciągłości na przedziale $[a,b]$. 
    Weźmy dowolne $\varepsilon > 0$ i $\delta_1 < \frac{\varepsilon}{8(M-m)k}$ (oczywiście $M \neq m$).
    Rozważmy przedziały otwarte $(x_l - \delta_1, x_l + \delta_1)$, $l = 1, \ldots, k$, gdzie $x_l$ są punktami
    nieciągłości funkcji $f$. Dopełnienie sumy tych przedziałów do przedziału $[a,b]$ składa się ze skończonej
    liczby przedziałów domkniętych, na których funkcja $f$ jest ciągła, a więc i jednostajnie ciągła. Ponieważ tych przedziałów
    jest skończenie wiele, więc dla danego $\varepsilon > 0$ istnieje liczba $\delta_2 > 0$ taka, że dla dowolnych
    punktów $x, t$ należacych do jednego z tych przedziałów, na których funkcja $f$ jest ciągła i spełniająca nierówność $|x-t| < \delta_2$ mamy
    $|f(x)-f(t)|<\frac{\varepsilon}{2(b-a)}$. Weźmy teraz liczbe $\delta = \min{(\delta_1, \delta_2)}$.
    Niech $P = \{x_0,\ldots,x_n\}$ będzię dowolnym podziałem przedziału $[a,b]$, dla którego $\delta(P) < \delta$. Ponadto
    rozbijmy zbiór indeksów $\{1, \ldots, n\}$ na dwa rozłącznę zbiory $A$ i $B$ w następujący sposób: do zbioru $A$ zaliczymy te liczby $i$, 
    dla których przedział $[x_{i-1}, x_i]$ nie ma punktów wspólnych z żadnym z skontruowanych powyżej otoczeń punktów $x_l$, $l=1,\ldots,k$, a do zbioru $B$
    pozostałe przedziały powstające z podziału $P$ przedziału $[a,b]$. Wówczas
    \[
        U(f, P) - L(f, P) = \sum_{i=1}^{n}(M_i - m_i)\Delta x_i = \sum_{i \in A}(M_i - m_i)\Delta x_i + \sum_{i \in B}(M_i - m_i)\Delta x_i
    \]
    Ponadto
    \[
        \sum_{i \in A}(M_i - m_i)\Delta x_i \leqslant \frac{\varepsilon}{2(b-a)}\sum_{i \in A}\Delta x_i \leqslant \frac{\varepsilon}{2(b-a)}(b-a)=\frac{\varepsilon}{2}
    \]
    Suma długości podprzedziałów przedziału $[a,b]$ indeksowanych przez liczby ze zbioru $B$ jest nie większa niż
    \[
        (\delta + 2\delta_i+\delta)k < 4 \frac{\varepsilon}{8(M-m)k}k = \frac{\varepsilon}{2(M-m)}
    \]
    Dlatego
    \[
        \sum_{i \in B}(M_i - m_i)\Delta x_i \leqslant (M-m)\sum_{i \in B}\Delta x_i < (M-m)\frac{\varepsilon}{2(M-m)}=\frac{\varepsilon}{2}
    \]
    Dla podziału $P$ o średnicy mniejszej niż $\delta$ otrzymujemy zatem
    \[
        U(f, P) - L(f, P) = \sum_{i=1}^{n}(M_i - m_i)\Delta x_i < \varepsilon
    \]
    co kończy dowód.
\end{proof}

\begin{uwaga}
    Twierdzenie $\ref{theorem:78}$ można istotnie uogólnić. Mianowicie dowodzi się, że jeśli $f$ jest ograniczoną funkcją
    na przedziale $[a,b]$, to jest ona całkowalna w sensie Riemanna na tym przedziale wtedy i tylko wtedy, gdy
    jest ona ciągła prawie wszędzie ma $[a,b]$, to znaczy zbiór punktów nieciągłości funkcji $f$ ma miarę Lebesgue'a równą zeru.
    (por. [7], s. 270).
    Przykładów takich funkcji dostarcza następujące
\end{uwaga}

\theorem{Funkcja monotoniczna na przedziale $[a,b]$ jest na tym przedziale całkowalna w sensie Riemanna.}

\proof{}
Załóżmy, że $f$ jest funkcją niemalejącą. Niech będzie dane dowolne $\varepsilon > 0$. Weźmy podział $P$ przedziału $[a,b]$ na $n$ równych części o długości
$\frac{b-a}{n}$. Ponieważ $f$ jest niemalejącą zatem $M_i = f(x_i)$ oraz $m_i = f(x_{i-1})$ dla $i = 1, \ldots, n$. Mamy więc
\[
    U(f, P) - L(f, P) = \sum_{i=1}^{n}(f(x_i) - f(x_{i-1}))\frac{b-a}{n} = (f(b)-f(a))\frac{b-a}{n}
\]
Biorąc $n$ tak duże, aby $(f(b) - f(a))\frac{b-a}{n} < \varepsilon$ i stosując twierdzenie $\ref{theorem:76}$ otrzymujemy tezę.
W przypadku funkcji nierosnącej dowód jest analogiczny.

\begin{theorem}
    {Jeśli $f$ jest całkowalna w sensie Riemanna na przedziale $[a,b]$, $m \leqslant f(x) \leqslant M$ dla $x \in [a,b]$ oraz $\phi$ jest funkcją ciągłą
    na $[m, M]$, to funkcja złożona $h = \phi \circ f$ jest R-całkowalna na $[a,b]$.}
\end{theorem}

\begin{proof}
    Ustalmy $\varepsilon > 0$. Ponieważ funkcja $\phi$ jest jednostajnie ciągła na $[M,m]$, więc
    istnieje $\delta > 0$ taka, że $\delta < \varepsilon$ i $|\phi(s) - \phi(t)| < \varepsilon$, jeśli $|s-t| < \delta$.
    Ponieważ $f$ jest R-całkowalna na $[a,b]$, więc istnieje podział $P = \{x_0, \ldots, x_n\}$
    przedziału $[a,b]$ taki, że $U(f, P) - L(f, P) < \delta^2$. Niech
    \[
        M_i=\sup\limits_{x_{i-1}\leqslant x \leqslant x_i}f(x), \quad m_i=\inf\limits_{x_{i-1}\leqslant x \leqslant x_i}f(x),
    \]
    \[
        M_i^*=\sup\limits_{x_{i-1}\leqslant x \leqslant x_i}h(x), \quad m_i^*=\inf\limits_{x_{i-1}\leqslant x \leqslant x_i}h(x)
    \]
    dla $i = 1, \ldots, n$. Podzielmy zbiór $\{1, \ldots, n\}$ na dwa rozłącznę zbiory $A$ i $B$ w taki sposób, że
    $i \in A$, jeśli $M_i - m_i < \delta$ oraz $i \in B$ w przypadku przeciwnym. Wówczas wobec powyższego wyboru $\delta$ mamy
    $M_i^* - m_i^* < \varepsilon$ dla $i \in A$. Natomiast dla $i \in B$ mamy $M_i^* - m_i^* \leqslant 2K$, gdzie $K=\sup{\{|\phi(t)|:m \leqslant t \leqslant M\}}$.
    Stąd otrzymujemy
    \[
        \delta \sum_{i \in B}(x_i - x_{i-1}) \leqslant \sum_{i \in B}(M_i - m_i)(x_i - x_{i-1}) < \delta^2, \quad \text{zatem} \sum_{i \in B}(x_i - x_{i-1}) < \delta
    \]
    Mamy więc
    \[
        U(h, P) - L(h, P) = \sum_{i \in A}(M_i^* - m_i^*)(x_i - x_{i-1}) + \sum_{i \in B}(M_i^* - m_i^*)(x_i - x_{i-1})
    \]
    a zatem
    \[
        U(h, P) - L(h, P) \leqslant \varepsilon(a + b + 2K)
    \]
    Ponieważ $\varepsilon$ było dowolne, zatem na mocy twierdzenia $\ref{theorem:76}$ funkcja $h$ jest R-całkowalna.
\end{proof}

Następujące twierdzenie opisuję związek całki Riemanna z operacjami arytmetycznymi. 

\begin{theorem}
    {Jeśli funkcje $f$ i $g$ są R-całkowalne na przedziale $[a,b]$, to również R-całkowalne są
    funkcje $f+g$, $\lambda f$ ($\lambda$ jest dowolną stałą rzeczywistą) i $fg$ oraz prawdziwe są równości:}
\end{theorem}
\begin{eq}
    \begin{equation}
        \int_{a}^{b} (f+g) (x) dx = \int_{a}^{b} f (x) dx + \int_{a}^{b} g (x) dx,
    \end{equation}
\end{eq}
\begin{eq}
    \begin{equation}
        \int_{a}^{b} (\lambda f) (x) dx = \lambda \int_{a}^{b} f (x) dx
    \end{equation}
\end{eq}

\proof{} Jest jasne, że dla dowolnego $R \in \mathfrak{R}(f+g, P)$ mamy $R = R_f + R_g$, gdzie $R_f \in \mathfrak{R}(f, P)$, $R_g \in \mathfrak{R}(g, P)$.
Niech $I_1 = \int_{a}^{b}f(x)dx$, $I_2 = \int_{a}^{b}f(x)dx$ oraz $I = I_1 + I_2$. Mamy
\[
    \forall_{\varepsilon > 0} \exists_{\delta > 0} \forall_P \forall_{R_f \in \mathfrak{R}(f, P)} \delta(P) < \delta \implies |R_f - I_1| < \frac{\varepsilon}{2} \quad \text{oraz}
\]
\[
    \forall_{\varepsilon > 0} \exists_{\delta > 0} \forall_P \forall_{R_g \in \mathfrak{R}(g, P)} \delta(P) < \delta \implies |R_g - I_2| < \frac{\varepsilon}{2} \quad \text{Stąd}
\]
\[
    \forall_{\varepsilon > 0} \exists_{\delta > 0} \forall_P \forall_{R \in \mathfrak{R}(f+g, P)} \delta(P) < \delta \implies |R - I| \leqslant |R_f - I_1| + |R_g - I_2| < \varepsilon
\]
Wobec powyższego jest jasne, że funkcja $f + g$ jest R-całkowalna na przedziale $[a,b]$ oraz, że spełniony jest wzór $\ref{eq:40}$.
Dowód wzoru $\ref{eq:41}$ jest analogiczny.

\noindent
Dalej przyjmując $\phi(t) = t^2$ oraz stosując do $\phi$ poprzednie twierdzenie (\ref{theorem:81}) otrzymujemy
R-całkowalność funkcji $f^2$.

\noindent
R-całkowalność iloczynu funkcji $fg$ wynika z tożsamości
\[
    fg = \frac{1}{4}[{(f+g)}^2 - {(f-g)}^2].
\]

\begin{theorem} 
{
    (a) Jeśli funkcje $f$ i $g$ są R-całkowalne na przedziale $[a,b]$ oraz $f(x) \leqslant g(x)$ dla każdego $x \in [a,b]$, to
    \[
        \int_{a}^{b}f(x)dx \leqslant \int_{a}^{b}g(x)dx
    \]
    (b) Jeśli funkcja $f$ jest R-całkowalna na przedziale $[a,b]$, to funkcja $|f|$ jest również
    R-całkowalna na tym przedziale oraz:
    \[
        \left\vert\int_{a}^{b}f(x)dx\right\vert \leqslant \int_{a}^{b}|f(x)|dx
    \]
}
\end{theorem}
\begin{proof}
    (a) Jeśli $m \leqslant f(x) \leqslant M$ dla $x \in [a,b]$, to
    \[
        m(b-a) \leqslant \int_{a}^{b}f(x)dx \leqslant M(b-a)
    \]
    Stąd, jeśli $f(x) \geqslant 0$ dla $x \in [a,b]$, to $\int_{a}^{b}f(x)dx \geqslant 0$.
    Wobec tego nierówność $f(x) \leqslant g(x)$ dla $x \in [a,b]$ implikuje
    \[
        \int_{a}^{b}f(x)dx \leqslant \int_{a}^{b}g(x)dx
    \]
    (b) Biorąc $\phi(t) = |t|$ w Twierdzeniu $\ref{theorem:81}$ otrzymujemy całkowalność funkcji $|f|$.
    Ponieważ $-|f(x)| \leqslant f(x) \leqslant |f(x)|$ dla $x \in [a,b]$, zatem na mocy (a) otrzymujemy
    \[
        \left\vert\int_{a}^{b}f(x)dx\right\vert \leqslant \int_{a}^{b}|f(x)|dx
    \]
\end{proof}
\begin{uwaga}
    (a) Punkt (a) Twierdzenia $\ref{theorem:83}$ można udowodnić bezpośrednio w oparciu o definicję
    całki Riemanna (Def. $\ref{def:69}$) oraz Wniosek 3 (b). \\
    \\
    \noindent
    (b) Twierdzenie odwrotne do Twierdzenia $\ref{theorem:83}$ (b) nie jest prawdziwe, to znaczy z R-całkowalności
    $|f|$ nie wynika R-całkowalność funkcji $f$. Dla przykładu niech
    \[ f(x) = \begin{cases}
        1 & \text{dla } x \in \mathbb{Q} \cap [a,b], \\
        -1 & \text{dla } x \in  (\mathbb{R \setminus Q}) \cap [a,b].
    \end{cases}
    \]
    Oczywiście funkcja $|f|$ jest R-całkowalna na przedziale $[a,b]$ oraz $\int_{a}^{b}|f(x)|dx = b-a$.
    Z kolei $underline{\int_{a}^{b}f(x)dx} = -(b-a)$ oraz $\overline{\int_{a}^{b}f(x)dx} = b-a$, a zatem
    wobec Twierdzenia $\ref{theorem:77}$ funkcja $f$ nie jest R-całkowalna na przedziale $[a,b]$.
\end{uwaga}

\begin{theorem}
    {Jeśli dwie funkcje $f$ i $g$ są równe na przedziale $[a,b]$ z wyjątkiem skończonego zbioru punktów ${\{x_1, \ldots, x_k\}}$
    i jedna z nich, na przykład $g$ jest R-całkowalna na tym przedziale, to druga też jest na nim R-całkowalna i zachodzi równość
    \[
        \int_{a}^{b}f(x)dx = \int_{a}^{b}g(x)dx.
    \]}
\end{theorem}
\begin{proof}
    Ponieważ $f = g + (f-g)$, więc wystarczy udowodnić, że funkcja $\phi = f - g$ jest R-całkowalna na $[a,b]$ i $\int_{a}^{b}\phi(x)dx = 0$.
    Oznaczmy $N = \max\{|\phi(x_1), \ldots, \phi(x_k)\}$. Niech $P$ będzie podziałem przedziału $[a,b]$ o średnicy $\delta$.
    Funkcja $\phi$ na co najwyżej $2k$ przedziałach podziału $P$ nie jest tożsamościowo równa zeru.
    Dlatego mamy $U(\phi, P) \leqslant 2Nk\delta$ i $L(\phi, P) = 0$, zatem $U(\phi, P) - L(\phi, P) \leqslant 2Nk\delta$. 
    Biorąc $\phi$ odpowiednio małe możemy uczynić różnicę $U(\phi, P) - L(\phi, P)$ dowolnie małą. To oznacza, że funkcja $\phi$ jest
    R-całkowalna. Ponadto jasne jest, że $\int_{a}^{b}\phi(x)dx = 0$.
\end{proof}

\begin{wniosek}
{Niech funkcja $f$ będzie określona i ograniczona na przedziale otwartym $(a,b)$. Jeśli po nadaniu jej pewnych wartości
$f(a)$ i $f(b)$ stanie się ona R-całkowalna na przedziale domkniętym $[a,b]$ --- to taką pozostanie --- gdy liczby $f(a)$ i $f(b)$ zmienimy
w sposób dowolny. Wartość całki nie ulegnie przy tym zmianie.}
\end{wniosek}

Następujący lemat pozwala przy przybliżaniu całki Riemanna sumami całkowymi ograniczyć się tylko do podziałów zawierających z góry ustalony punkt.

\begin{lemat}
{
    Niech $c \in [a,b]$ i niech $\Pi^*$ oznacza zbiór wszystkich podziałów przedziału $[a,b]$ spełniających warunek:
    \[
        P = \{x_0, \ldots, x_n\} \in \Pi^* \text{ wtedy i tylko wtedy $x_j = c$ dla pewnego $j$.} 
    \]
    Wówczas dla dowolnej funkcji $f$, ograniczonej na $[a,b]$ zachodzą równości:
    \[
        \sup_{P \in \Pi^*}L(f, P) = \sup_{P}L(f, P), \quad  \inf_{P \in \Pi^*}U(f, P) = \inf_{P}U(f, P)
    \]
}
\end{lemat}
\begin{proof}
    Ponieważ $\Pi^*$ jest podzbiorem zbioru wszystkich podziałów przedziału $[a,b]$, więc
    \begin{equation}\label{eq:42}
        \sup_{P \in \Pi^*}L(f, P) \leqslant \sup_{P}L(f, P), \quad  \inf_{P \in \Pi^*}U(f, P) \geqslant \inf_{P}U(f, P)
    \end{equation}
    Zauważmy, że dla dowolnego podziału $P$ przedziału $[a,b]$ istnieje podział od niego drobniejszy
    $P^* \in \Pi^*$. Istotnie, jeśli $P \in \Pi^*$, to przyjmujemy $P^* = P$. Jeśli natomiast $P \notin \Pi^*$,
    to przez dołączenie punktu $c$ do układu punktów wyznaczających $P$ otrzymujemy podział $P^*$ o żądanych własnościach.
    Mamy więc
    \[
        L(f, P) \leqslant L(f, P^*), \quad U(f, P) \geqslant U(f, P^*),
    \]
    skąd otrzymujemy
    \[
        L(f, P) \leqslant \sup_{P^* \in \Pi^*}L(f, P^*), \quad U(f, P) \geqslant \inf_{P^* \in \Pi^*}U(f, P^*).
    \]
    Wobec dowolności podziału $P$ mamy
    \[
        \sup_{P}L(f, P) \leqslant \sup_{P^* \in \Pi^*}L(f, P), \quad  \inf_{P}U(f, P) \geqslant \inf_{P^* \in \Pi^*}U(f, P)
    \]
    Z powyższych nierówności i z (\ref{eq:42}) otrzymujemy tezę.
\end{proof}

\begin{theorem}
{
    Niech $a < c < b$. Funkcja $f$ jest R-całkowalna na przedziale $[a,b]$ wtedy i tylko wtedy, gdy
    jest ona R-całkowalna na przedziałach $[a,c]$ i $[c,b]$. Zachodzi przy tym równość
    \[
        \int_{a}^{b}f(x)dx = \int_{a}^{c}f(x)dx + \int_{c}^{b}f(x)dx
    \]
    (addytywność całki ze względu na przedział)
}
\end{theorem}

\begin{proof}
    Załóżmy, że funkcja $f$ jest R-całkowalna na przedziale $[a,b]$. Na mocy powyższego lematu
    możemy ograniczyć się do podziałów przedziału $[a,b]$ zawierających punkt $c$. 
    Jeśli $P$ jest takim podziałem, to wówczas $P = P_1 \cup P_2$, gdzie $P_1$ jest podziałem
    przedziału $[a,c]$, a $P_2$ --- podziałem przedziału $[c,b]$ oraz mamy
    \[
        U(f, P) = U(f, P_1) + U(f, P_2), \quad L(f, P) = L(f, P_1) + L(f, P_2).
    \]
    Niech będzie dane dowolne $\varepsilon > 0$ i niech
    \[
        U(f, P) - L(f, P) < \frac{\varepsilon}{2}.
    \]
    Stąd $U(f, P_1) - L(f, P_1) < \frac{\varepsilon}{2}$ i $U(f, P_2) - L(f, P_2) < \frac{\varepsilon}{2}$. 
    Funkcja $f$ jest więc całkowalna na przedziałach $[a,c]$ i $[c, b]$ oraz zachodzą nierówności
    \[
        U(f, P_1) < \int_{a}^{c}f(x)dx + \frac{\varepsilon}{2}, \quad \int_{a}^{c}f(x)dx < L(f, P_1) + \frac{\varepsilon}{2},
    \]
    \[
        U(f, P_2) < \int_{c}^{b}f(x)dx + \frac{\varepsilon}{2}, \quad \int_{c}^{b}f(x)dx < L(f, P_2) + \frac{\varepsilon}{2},
    \]
    Wobec powyższego otrzymujemy $U(L, P) < \int_{a}^{c}f(x)dx + \int_{c}^{b}f(x)dx + \varepsilon$
    i w konsekwencji $\int_{a}^{b}f(x)dx < \int_{a}^{c}f(x)dx + \int_{c}^{b}f(x)dx + \varepsilon$.
    Ponieważ $\varepsilon > 0$ było dowolne, zatem
    \begin{equation}\label{eq:43}
        \int_{a}^{b}f(x)dx \leqslant \int_{a}^{c}f(x)dx + \int_{c}^{b}f(x)dx.
    \end{equation}
    Analogicznie $\int_{a}^{c}f(x) + \int_{c}^{b}f(x)dx < L(f, P) + \varepsilon$, skąd
    \begin{equation}\label{eq:44}
        \int_{a}^{c}f(x)dx + \int_{c}^{b}f(x)dx \leqslant \int_{a}^{b}f(x)dx,
    \end{equation}
    bowiem $\varepsilon > 0$ jest dowolne. Z nierówności (\ref{eq:43}) i (\ref{eq:44}) otrzymujemy żądaną równość.
    Uzasadnienie implikacji odwrotnej jest analogiczne.
\end{proof}

Rozszerzymy teraz zasięg Definicji $\ref{def:69}$.

\begin{defn}
    W przypadku gdy $b < a$ lub $b = a$, to całkę Riemanna z funkcji $f$ określamy wzorami
    \[
        \int_{a}^{b}f(x)dx = -\int_{b}^{a}f(x)dx \quad \text{lub odpowiednio} \quad \int_{a}^{b}f(x)dx = 0.
    \]
    W całce $\int_{a}^{b}f(x)dx$ liczbę $a$ nazywamy dolną granicą całkowania, liczbę $b$ -- górną granicą całkowania,
    bez względu na to, czy $b \geqslant a$, czy też $b < a$. 
\end{defn}

\begin{wniosek}
{
(a) Niech $a,b,c \in \mathbb{R}$ i niech $f$ będzie funkcją R-całkowalna na najwiekszym z przedziałów
domkniętych o końcach we wskazanych punktach. Wówczas obcięcie funkcji $f$ do każdego z dwóch pozostałych przedziałów domkniętych jest
funkcja R-całkowalną na odpowiednim przedziale oraz zachodzi równość
\begin{equation}\label{eq:45}
    \int_{a}^{b}f(x)dx + \int_{b}^{c}f(x)dx + \int_{c}^{a}f(x)dx = 0.
\end{equation}
}
\end{wniosek}
\begin{proof}
    Wobec symetrii równość (\ref{eq:45}) względem $a, b, c$ możemy bez straty ogólności założyć, że
    $a = \min{\{a,b,c\}}$. Jeśli $\max{\{a,b,c\} = c}$ oraz $a < b < c$, to na mocy
    Twierdzenia $\ref{theorem:85}$ mamy
    \[
        \int_{a}^{b}f(x)dx + \int_{c}^{b}f(x)dx - \int_{a}^{c}f(x)dx = 0,
    \]
    zatem wobec Definicji $\ref{def:74}$ otrzymujemy równość (\ref{eq:45}). \\
    Jeśli $\max{\{a,b,c\} = b}$ oraz $a < c < b$, to ponownie na mocy Twierdzenia
    $\ref{theorem:85}$ mamy
    \[
        \int_{a}^{c}f(x)dx + \int_{c}^{b}f(x)dx - \int_{a}^{b}f(x)dx = 0,
    \]
    stąd wobec Definicji $\ref{def:74}$ wynika równość (\ref{eq:45}). \\
    W końcu, jeśli jakiekolwiek dwa z punktów $a,b,c$ lub wszystkie trzy pokrywają się,
    to (\ref{eq:45}) jest bezpośrednią konsekwencją Definicji $\ref{def:74}$.
\end{proof}

\noindent
\textit
{
(b) Jeśli funkcja $f$ jest R-całkowalna na przedziale $[a,b]$ i $a \leqslant c < d \leqslant d$, to jest
ona również R-całkowalna na przedziale $[c,d]$.
}

\begin{defn}
    Niech każdej uporządkowanej parze $(\alpha, \beta)$ punktów $\alpha$, $\beta$ przedziału
    $[a,b]$ odpowiada dokładnie jedna liczba $I(\alpha, \beta)$, przy czym dla dowolnej trójki punktów
    $\alpha, \beta, \gamma \in [a,b]$ zachodzi równość
    \[
        I(\alpha, \gamma) = I(\alpha, \beta) + I(\beta, \gamma).
    \]
    Wówczas funkcja $I(\alpha, \beta)$ nazywa się addytywną funkcją przedziału zorientowanego
    (dla $\alpha = \gamma$ wobec powyższej równości otrzymujemy $I(\alpha, \beta) = -I(\beta, \alpha)$), 
    określoną na odcinkach zawartych w przedziale $[a,b]$. 
\end{defn}

\begin{wniosek}
{
    Jeśli funckja $f$ jest R-całkowalna na przedziale $[a,b]$ oraz $\alpha, \beta, \gamma \in [a,b]$, to
    kładąc $I(\alpha, \beta) = \int_{\alpha}^{\beta}f(x)dx$, na mocy równości $\ref{eq:45}$ otrzymujemy
    \[
        \int_{\alpha}^{\gamma}f(x)dx = \int_{\alpha}^{\beta}f(x)dx + \int_{\beta}^{\gamma}f(x)dx,
        \quad \text{czyli} \quad I(\alpha, \gamma) = I(\alpha, \beta) + I(\beta, \gamma),
    \]
    czyli całka Riemanna jest adytywną funkcją przedziału zorientowanego.
}
\end{wniosek}

Udowodnimy teraz ważne twierdzenie o funkcji górnej granicy całkowania.

\begin{theorem}
{
    Niech $f$ będzie funkcja R-całkowalną na przedziale $[a,b]$. Dla dowolnego punktu
    $x \in [a,b]$ określamy
    \[
        F(x) = \int_{a}^{x}f(t)dt.
    \]
    Wówczas funkcja $F$ jest ciągła na przedziale $[a,b]$. Ponadto, jeśli funkcja $f$ jest ciągła w
    punkcie $x_0 \in [a,b]$, to funkcja $F$ jest różniczkowalna w tym punkcie oraz
    $F'(x_0) = f(x_0)$.
}
\end{theorem}

\begin{proof}
    Niech $M$ będzie takie, że $|f(t)| \leqslant M$ dla $t \in [a,b]$. 
    Wówczas, jeśli $a \leqslant x \leqslant y \leqslant b$, to
    \[
        |F(y) - F(x)| = \left|\int_{x}^{y}f(t)dt\right| \leqslant M(y - x).
    \]
    Stąd wynika natychmiast, że dla dowolnego $\varepsilon > 0$ mamy
    $|F(y) - F(x)|< \varepsilon$, jeśli tylko $|y-x| < \frac{\varepsilon}{M}$, 
    a zatem funkcja F jest ciągła. \\
    Załóżmy, że $f$ jest ciągła w punkcie $x_0$. Dla danego $\varepsilon > 0$ wybierzmy 
    $\delta > 0$ tak, aby $|f(t)-f(x_0)| < \varepsilon$ jeśli tylko $|t-x_0| < \delta$ i $a \leqslant t \leqslant b$.
    Wówczas dla $s,t \in (x_0 - \delta, x_0 + \delta)$, $s,t \in [a,b]$, $s \neq t$ mamy
    \[
        \left|\frac{F(t)-F(s)}{t-s} - f(x_0)\right| = \left|\frac{1}{t-s}\int_{s}^{t}(f(u)-f(x_0))du \right| < \varepsilon,
    \]
    Stąd wynika, że $F'(x_0) = f(x_0)$, co kończy dowód.
\end{proof}

Oznaczmy $F(x) = I(a, x)$ dla $x \in [a,b]$, gdzie $I$ oznacza addytywną funkcję przedziału zorientowanego. Mamy
\[
    I(\alpha, \beta) = I(a, \beta) - I(a, \alpha) = F(\beta) - F(\alpha)
\]
dla każdej uporządkowanej pary punktów $(\alpha, \beta)$ z przedzialu $[a,b]$.
W ten sposób każda addytywna funkcja przedziału zorientowanego ma postać
\begin{equation}\label{eq:46}
    I(\alpha, \beta) = F(\beta) - F(\alpha),
\end{equation}
gdzie $x \mapsto F(x)$ jest funkcją określoną na przedziale $[a,b]$. 
Można łatwo sprawdzić, że jest również na odwrót, to znaczy, że z dowolnej funkcji $x \mapsto F(x)$ określonej na przedziale $[a,b]$
można przy pomocy (\ref{eq:46}) otrzymać addytywną funkcję przedziału zorientowanego.

\begin{wniosek}
    Jeśli $f$ jest funkcją R-całkowalną na przedziale $[a,b]$, to na mocy (\ref{eq:46})
    funkcja $F(x) = \int_{a}^{x}f(t)dt$ generuje addytywną funkcję 
    \[
        I(\alpha, \beta) = \int_{\alpha}^{\beta}f(t)dt.
    \]
\end{wniosek}

\subsection{Całka nieoznaczona}

\begin{defn}
    Niech $f$ będzie funkcją określoną na pewnym przedziale $I$. Każdą funkcję $F$ różniczkowalną na tym przedziale
    i spełniającą w każdym punkcie $x \in I$ równość
    \[
        F'(x) = f(x)
    \]
    nazywamy funkcją pierwotną funkcji $f$. Funkcję pierwotną nazywamy również całką
    nieoznaczoną danej funkcji i oznaczamy symbolem $\int f(x)dx$ (symbol ten należy również
    rozumieć jako oznaczenie dowolnej funkcji pierwotnej funkcji $f$ na tym przedziale).
    W symbolu tym znak $f$ nazywa się znakiem całki nieoznaczonej, $f$ --- funkcją podcałkową, a $f(x)dx$ ---
    wyrażeniem podcałkowym.
\end{defn}

\begin{uwaga}
    Jeśli $F$ jest funkcją pierwotną funkcji $f$, to suma $F + c$, gdzie $c$ jest dowolną stałą, jest
    również funkcją pierwotną funkcji $f$, bowiem $(F+c)' = F' = f$. \\
    Na odwrót, dwie dowolne fukcje pierwotne $F$ i $G$ tej samej funkcji $f$ róźnią się o stałą,
    bowiem $(F-G)'= f-f = 0$. \\
    Jeśli $F$ jest więc konkretną funkcją pierwotną funkcji $f$ na przedziale $I$, to na tym przedziale
    \[
        \int f(x)dx = F(x) + c,
    \]
    to znaczy dowolna inna funkcja pierwotna funkcji $f$ może być otrzymana z danej funkcji $F$ przez dodanie stałej.

\end{uwaga}

Bezpośrednio z Twierdzenia $\ref{theorem:86}$ otrzymujemy następujący

\begin{wniosek}
{
    Każda funkcja ciągła $f$ na przedziale $[a,b]$ ma na nim funkcję pierwotną.
}
\end{wniosek}

Dowód wniosku $\ref{wniosek:22}$ można uzyskać bez pojęcia całki Riemanna. Jest on jednak
dość długi. \\
Istnieją również funkcję nieciągłe, które posiadają funkcje pierwotne.

\begin{ex}
Niech
\[
f(x) =
\begin{cases}
    \begin{aligned}
        &2x\sin\frac{1}{x}-\cos{\frac{1}{x}} & &\text{dla } x \neq 0 \\
        &0                                   & &\text{dla } x = 0.
    \end{aligned}
\end{cases}
\]
Ponieważ nie istnieje granica funkcji $f$ w zerze, zatem $f$ nie jest funkcją ciągłą. Można łatwo sprawdzić, że funkcja 
\[
F(x) =
\begin{cases}
    \begin{aligned}
        &x^2\sin{\frac{1}{x}} & &\text{dla } x \neq 0 \\
        &0                                   & &\text{dla } x = 0,
    \end{aligned}
\end{cases}
\]
jest funkcją pierwotną funkcji $f$ (por. [7], s. 90-91).
\end{ex}

Podamy teraz przykłady funkcji całkowalnej w sensie Riemanna, która nie posiada funkcji pierwotnej.

\begin{ex}
    Niech $f : (1,3) \mapsto \mathbb{R}$ będzie określona wzorem $f(x) = [x]$.
    Jest jasne, że dla $f(x) = F'(x)$ dla $x \in (1,2) \cup (2,3)$, gdzie
    \[
    F(x) =
        \begin{cases}
            \begin{aligned}
                &x + c_1 & &\text{dla } x \in (1,2) \\
                &2x + c_2                                 & &\text{dla } x \in (2,3),
            \end{aligned}
        \end{cases}
    \]
    gdzie, $c_1, c_2 \in \mathbb{R}$. Funkcja pierwotna funkcji $f$ na przedziale $(1,3)$ 
    (z dokładnościa do stałej musiałaby mieć postać)
    \[
    F(x) =
        \begin{cases}
            \begin{aligned}
                &x + c_1 & &\text{dla } x \in (1,2] \\
                &2x + c_1 - 2 & &\text{dla } x \in (2,3),
            \end{aligned}
        \end{cases}
    \]
    gdzie $c_1 \in \mathbb{R}$. Istotnie, aby funkcja $F$ była ciągła dla $x = 2$, to 
    $\lim\limits_{x\to2^-}F(x) = \lim\limits_{x\to2^+}F(x)$, czyli $2 + c_1 = 4 + c_2$, zatem $c_2 = c_1 - 2$. 
    Można łatwo sprawdzić, że $F_{-}'(2) = 1$ oraz $F_{+}'(2) = 2$, czyli $F$ nie jest
    funkcją pierwotną funkcji $f$ na przedziale $(1,3)$.
\end{ex}

\newpage
\textbf{Tablica 1.} Całki nieoznaczona podstawowych funkcji elementarnych

\noindent
\begin{tabular}{|c|c|c|}
    \hline
    $f(x)$ & $F(x)$ & Ograniczenia ze względu na argument $x \in \mathbb{R}$ \\
    \hline
    0 & $c = $ const. & \\
    \hline
    $a = $ const. & $ax + c$ &  \\
    \hline
    $x^p$ & $\frac{1}{p+1}x^{p+1} + c$ & \makecell{$p \neq -1, x > 0$ $(p \in \mathbb{R})$, \\ $x \neq 0$ $(p \in \mathbb{Z})$, \\ $x \in \mathbb{R}$ $(p \in \mathbb{N})$}\\
    \hline
    $\frac{1}{x}$ & $\ln{|x|} + c$ & $x \neq 0$ \\
    \hline
    $a^x$ & $\frac{a^x}{\ln{a}} + c$ & $x \in \mathbb{R}$ $(a >0, a \neq 1)$ \\
    \hline
    $e^x$ & $e^x + c$ &  \\
    \hline
    $\sin{x}$ & $-\cos{x} + c$ & \\
    \hline
    $\cos{x}$ & $\sin{x} + c$ & \\
    \hline
    $\frac{1}{\cos^2{x}}$ & $\tg{x} + c$ & $x \neq \frac{\pi}{2} + k\pi$, $k \in \mathbb{Z}$ \\
    \hline
    $\frac{1}{\sin^2{x}}$ & $-\ctg{x} + c $& $ x \neq k\pi$, $k \in \mathbb{Z}$ \\
    \hline
    $\frac{1}{1+x^2}$ & \makecell{$\arctg{x} + c$ \\ $\arcctg{x} + \hat{c}$} & \\
    \hline
    $\frac{1}{\sqrt{1 - x^2}}$ & \makecell{$\arcsin{x} + c$ \\ $-\arccos{x} + \hat{c}$} & $|x| < 1$ \\
    \hline
\end{tabular}

\begin{uwaga}
    (a) Wzory zawarte w Tablicy 2 otrzymujemy przez bezpośrednie różniczkowanie funkcji $F(x)$ (zob. Tablica 1).

    \noindent
    (b) Jeżeli zakres argumentów, dla których spełniona jest równość $F(x) = f'(x)$ nie jest przedziałem
    (skończonym lub nieskończonym), to nie można twierdzić, że wyrażenie $F(x) + c$ obejmuje wszystkie funkcje pierwotne funkcji $f$
    w tym zakresie argumentów. Dla przykładu funkcja

    \[
        G(x) =
            \begin{cases}
                \begin{aligned}
                    &\ln(-x) & &\text{dla } x < 0 \\
                    &\ln{x} & &\text{dla } x > 0,
                \end{aligned}
            \end{cases}
    \]
    jest funkcją pierwotną funkcji $x \mapsto \frac{1}{x}$ $(x \neq 0)$, mimo, że nie podpada pod
    wzór $\ln|x| + c$. 
\end{uwaga}

Następujące twierdzenie podaje reguły obliczani całek nieoznaczonych.

\begin{theorem}
{   (a) Jeśli istnieją całki nieoznaczone funkcji $u, v : P \mapsto \mathbb{R}$, gdzie $P$ jest przedziałem
    oraz $\alpha, \beta \in \mathbb{R}$, to istnieje całka nieoznnaczona funkcji
    $\alpha u + \beta v$ oraz zachodzi wzór
    \begin{equation}\label{eq:47}
        \int (\alpha u(x) + \beta v(x))dx = \alpha \int u(x)dx + \beta \int v(x)dx + c \text{ dla } x \in P.
    \end{equation}
    (b) Przy założeniach punktu (a) oraz przy założeniu, że funkcje $u, v$ są różniczkowalne oraz jedna z całek występujących
    w poniższym wzorze istnieje, prawdziwy jest następujący wzór zwany wzorem na całkowanie przez części:
    \begin{equation}\label{eq:48}
        \int u(x)v'(x)dx = u(x)v(x) - \int u'(x)v(x)dx + c \text{ dla } x \in P.
    \end{equation}
    (c) Jeśli na przedziale $I$, $\int f(x)dx = F(x) + c$ oraz $\phi : P \mapsto I$ jest odwzorowaniem klasy $C^1$, to
    \begin{equation}\label{eq:49}
        \int f(\phi(t))\phi'(t)dt = F(\phi(t)) + c \text{ dla } t \in P.
    \end{equation}
}
\end{theorem}

\begin{proof}
    (a) Wzór (\ref{eq:47}) sprawdza się bezpośrednio przez różniczkowanie lewej i prawej strony z wykorzystaniem liniowości różniczkowania (zob. Tw. 54 (a), (b)).
    
    \noindent
    (b) Załóżmy, że $\int u(x)v'(x)dx = \Phi(x)$ dla $x \in P$. Ponieważ $(u(x)v(x))'=u'(x)v(x) + u(x)v'(x)$,
    $(u(x)v(x)-\Phi(x))' = u'(x)v(x)$ dla $x \in P$, a więc $uv - \Phi$ jest funkcją pierwotną funkcji $u'v$ na przedziale $P$.
    Ponadto mamy
    \[
        \int u(x)v'(x)dx = \int (u(x)v(x))'dx - \int u'(x)v(x)dx, \text{ a więc}
    \]
    \[
        \int u(x)v'(x)dx = u(x)v(x) - \int u'(x)v(x)dx + c.
    \]

    \noindent
    (c) Wzór (\ref{eq:49}) jest bezpośrednią konsekwencją reguły różniczkowania funkcji złożonej
    (zob. Tw. 55).

\end{proof}

\begin{uwaga}
    (a) Wzór (\ref{eq:49}) pokazuje, że chcąc uzyskać funkcję pierwotną funkcji $t \mapsto f(\phi(t))\phi'(t)$ można postąpić w następujacy sposób:
    \[
        \int f(\phi(t))\phi'(t)dt = \int f(x)dx = F(x) + c = F(\phi(t)) + c,
    \]
    to znaczy najpierw dokonać zmiany $\phi(t) = x$ i przejść do nowej zmiennej $x$, a następnie
    przejść do poprzedniej zmiennej podstawiając $x = \phi(t)$. Wzór ten nazywa się
    wzorem na całkowanie przez podstawienie.

    \noindent
    (b) Jeśli $\phi : P \mapsto I$ jest bijekcją, to aby obliczyć całkę nieoznaczoną $\int f(x)dx$ można obliczyć całke nieoznaczoną
    $\int f(\phi(t))\phi'(t)dt$, a następnie dokonać podstawienia $t = \phi^{-1}(x)$.

    \noindent
    (c) W szczególności wzór (\ref{eq:48}) jest prawdziwy jeśli $u,v \in C^1$.
\end{uwaga}

\begin{ex}
    (a) Obliczmy $\int \arcsin x dx$. Najpierw zastosujmy wzór (\ref{eq:48}) przyjmując $u(x) = \arcsin x$ i $v'(x) = 1$.
    Wówczas $u'(x) = \frac{1}{\sqrt{1 - x^2}}$ (dla $x \in (-1, 1)$) i $v(x) = x$ oraz
    $\int \arcsin x dx = x \arcsin x - \int \frac{x}{\sqrt{1 - x^2}}dx$. Aby
    obliczyć tę ostatnią całkę zastosujmy podstawienie $\sqrt{1 - x^2} = t$. Mamy wówczas
    $\frac{-xdx}{\sqrt{1-x^2}}=dt$ i stąd $\int \frac{x}{\sqrt{1 - x^2}}dx = -\int dt = -t + c = -\sqrt{1 - x^2} + c$, a zatem
    \[
        \int \arcsin x dx = x \arcsin x + \sqrt{1 - x^2} + c.
    \]

    \noindent
    (b) Obliczmy $\int \tg t dt$. Niech np. $P = (-\frac{\pi}{2}, \frac{\pi}{2})$, $I = (0,1]$, $\phi : P \mapsto I$
    będzie określone wzorem $\phi(t) = cost$, natomiast $f(x) = \frac{1}{x}$ dla $x \in (0, 1]$. Mamy
    \[
        \int \tg t dt = \int \frac{1}{\cos t}\sin t dt = -\int \frac{1}{x}dx = -\ln x + c = \ln \cos t + c.
    \]
    W ogólności podstawiamy $\cos t = x$. Wówczas $-\sin t dt = dx$ oraz
    \[
        \int \tg t dt = -\int \frac{dx}{x} = -\ln |x| + c = -\ln |\cos t| + c.
    \]
    Wobec Uwagi $\ref{uwaga:41}$ (b) powyższy wzór nie objemuje wszystkich funkcji pierwotnych funkcji
    $t \mapsto \tg t$, gdzie $t \neq \frac{\pi}{2} + k\pi$, $k \in \mathbb{Z}$.

    \noindent
    (c) Obliczmy $\sqrt{1 - x^2}$. Niech $I = (-1, 1)$, $P = (-\frac{\pi}{2}, \frac{\pi}{2})$,
    $f(x) = \sqrt{1 - x^2}$ dla $x \in I$ oraz $\phi(t) = sint$ dla $t \in P$. Oczywiście $\phi : P \mapsto I$
    jest bijekcją oraz $\phi^{-1}(x) = \arcsin{x}$ dla $x \in I$. Mamy
    \begin{equation*}
        \begin{split}
            \int \sqrt{1-x^2}dx &= \int \sqrt{1-\sin^2{t}}\cos tdt  = \int \cos^{2}{t}dt
            \int \frac{dt}{2} + \frac{1}{2}\int \cos{2t}dt \\
            &= \frac{1}{2}t + \frac{1}{4}\sin{2t} + c = \frac{1}{2}\arcsin{x} + \frac{1}{4}\sin(2\arcsin{x}) + c \\
            &= \frac{1}{2}\arcsin x + \frac{1}{2} \sin(\arcsin x)\cos( \arcsin x) + c \\
            &= \frac{1}{2}\arcsin{x} + \frac{1}{2}x\sqrt{1 - \sin^{2} (\arcsin x)} + c \\
            &= \frac{1}{2}\arcsin{x} + \frac{1}{2}x\sqrt{1 - x^2} + c.
        \end{split}
    \end{equation*}
  
\end{ex}

\begin{uwaga}
    Funkcjami elementarnymi nazywamy funkcje potęgowe, wykładnicze, trygonometryczne, funkcje odwrotne do nich, ich superpozycje oraz funkcje
    powstałe przez wykonanie skończonej ilości działań na nich: dodawanie, odejmowanie, mnożenie,
    dzielenie oraz składanie. 
    Podstawowe metody całkowania funkcji elementarnych oraz pewne tzw. wzory rekurencyjne można znaleźć np.
    w książce [5], s. 260-273. 
\end{uwaga}

\subsection{Rachunek całek (oznaczonych) w sensie Riemanna}
Udowodnimy wpierw następujące twierdzenie nazywane podstawowym twierdzeniem rachunku całkowego
lub wzorem Newtona-Leibniza.

\begin{theorem}
{
    Jeśli funkcja $f$ jest R-całkowalna na przedziale $[a,b]$ oraz $F$ jest funkcją pierwotną funkcji $f$ na tym przedziale, to
    \begin{equation*}
        \int_{a}^{b}f(x)dx = F(b) - F(a).
    \end{equation*}
}
\end{theorem}

\begin{proof}
    Wybierzmy dowolne $\varepsilon > 0$. Istnieje wówczas taki podział $P$ przedziału $[a,b]$, że
    $U(f, P) - L(f, P) < \varepsilon$. Wówczas
    \begin{equation*}
        U(f, P) < \varepsilon + \int_{a}^{b}f(x)dx, \quad \int_{a}^{b}f(x)dx < \varepsilon + L(f, P).
    \end{equation*}
    Na mocy Twierdzenie Lagrange'a o Wartości Średniej istnieją punkty $\xi_{j}$ ($j = 1,\ldots, n$) takie, że
    $x_{j-1} < \xi_{j} < x_{j}$ ($P = \{ x_0, x_1, \ldots, x_n\}$) oraz $F(x_j) - F(x_{j-1}) =
    f(\xi_j)(x_j - x_{j-1})$, a zatem
    \begin{equation*}
        \begin{split}
            F(b) - F(a) &= \sum_{j=1}^{n}(F(x_j) - F(x_{j-1})) \\
                        &= \sum_{j=1}^{n}f(\xi_j)(x_j - x_{j-1}) \leqslant U(f, P) < \varepsilon + \int_{a}^{b}f(x)dx, \\
            F(b) - F(a) &\geqslant L(f, P) > -\varepsilon + \int_{a}^{b}f(x)dx.            
        \end{split}
    \end{equation*}
    Stąd
    \begin{equation*}
        \left|F(b) - F(a) - \int_{a}^{b}f(x)dx \right| < \varepsilon
    \end{equation*}
    i wobec dowolności $\varepsilon > 0$ otrzymujemy żądany wzór.
\end{proof}

\begin{ex}
    Istnieją funkcje, które nie są całkowalne w sensie Riemanna, ale posiadają funkcje pierwotne. Niech
    \begin{equation*}
        f(x) = 
            \begin{cases}
                \begin{aligned}
                    &2x\cos \frac{\pi}{x^2} - \frac{2\pi}{x}\sin \frac{\pi}{x^2} &\text{ dla } x \neq 0&, \\
                    &0 &\text{ dla } x = 0&
                \end{aligned}
            \end{cases}
    \end{equation*}
    oraz
    \begin{equation*}
        F(x) = 
            \begin{cases}
                \begin{aligned}
                    &x^2 \cos \frac{\pi}{x^2} &\text{ dla } x \neq 0&, \\
                    &0 &\text{ dla } x = 0&
                \end{aligned}
            \end{cases}
    \end{equation*}
    Można łatwo sprawdzić, że $F'(x) = f(x)$ dla każdego $x \in [0,1]$. Funkcja $f$ nie jest jednak
    R-całkowalna na przedziale $[0,1]$, ponieważ nie jest na nim ograniczona. Niech bowiem
    $x_k = \frac{1}{\sqrt{\frac{1}{2}+2k}}$, $k \in \mathbb{N}$. Mamy
    \begin{equation*}
        \begin{split}
        \lim\limits_{k \to \infty}\frac{2\pi}{x_k} \sin \frac{\pi}{x^2_k} = \lim\limits_{k \to \infty}\frac{2\pi}{\frac{1}{\sqrt{\frac{1}{2}+2k}}} \sin \frac{\pi}{\frac{1}{\frac{1}{2} + 2k}} \\ 
        = \lim\limits_{k \to \infty}2\pi \sqrt{\frac{1}{2} + 2k}\sin(\frac{\pi}{2} + 2k\pi) = +\infty.
        \end{split}
    \end{equation*}
\end{ex}

Udowodnimy teraz reguły obliczania całek oznaczonych.

\begin{theorem}
{
    (o całkowaniu przez części). Niech pochodne funkcji $u$ i $v$ będą R-całkowalne
    na przedziale $[a,b]$. Wówczas zachodzi wzór (zwany wzorem na całkowanie przez części.)
    \[
        \int_{a}^{b}u(x)v'(x)dx = u(x)v(x) \big|_a^b - \int_{a}^{b}u'(x)v(x)dx,
    \]
    gdzie $u(x)v(x) \big|_a^b = u(b)v(b) - u(a)v(a)$.
}
\end{theorem}

\begin{proof}
    Istotnie, ponieważ $(uv)'(x) = u(x)v'(x) + u'(x)v(x)$, więc
    \[
        \int_{a}^{b}(uv)'(x)dx = \int_{a}^{b}u(x)v'(x)dx + \int_{a}^{b}u'(x)v(x)dx,
    \]
    a zatem na mocy wzoru Newtona-Leibniza otrzymujemy 
    \[
        u(x)v(x) \big|_a^b = \int_{a}^{b}u(x)v'(x)dx + \int_{a}^{b}u'(x)v(x).
    \]
\end{proof}

\begin{wniosek}
{
    Jeśli funkcja $f$ ma na przedziale o końcach $x_0$ i $x$ ciągłe pochodne do rzędu $n+1$ włącznie to
    \begin{equation*}
        f(x) = f(x_0) + \frac{f'(x_0)}{1!}(x-x_0) + \frac{f''(x_0)}{2!}{(x-x_0)}^2 + \ldots \frac{f^{(n)}(x_0)}{n!}{(x-x_0)}^n + r_n(x_0, x),
    \end{equation*}
    gdzie $r_n(x_0, x) = \frac{1}{n!}\int_{x_0}^{x}f^{(n+1)}(t){(x-t)}^n dt$. \\
    (Wzór Taylora dla funkcji $f$ z resztą w postaci całkowej).
}
\end{wniosek}

\begin{proof}
    Stosując Wzór Newtona-Leibniza i wzór na całkowanie przez części wykonujemy następujący ciąg przekształceń, w którym wszystkie różniczkowania i podstawienia
    wykonywane są względem $t$: 
    \begin{multline*}
        f(x) - f(x_0) = \int_{x_0}^{x}f'(t)dt = -\int_{x_0}^{x}f'(t)(x-t)'dt \\
        = -f'(t)(x-t) \big|_{x_0}^x + \int_{x_0}^{x}f''(t)(x-t)dt = f'(x_0)(x-x_0) - \frac{1}{2}\int_{x_0}^{x}f''(t)({(x-t)}^2)'dt \\
        = f'(x_0)(x-x_0) - \frac{1}{2}f''(t){(x-t)}^2 \big|_{x_0}^x + \frac{1}{2}\int_{x_0}^{x}f''(t){(x-t)}^2dt \\
        = f'(x_0)(x-x_0) + \frac{1}{2}f''(x_0){(x-x_0)}^2 - \frac{1}{2 \cdot 3}\int_{x_0}^{x}f'''(t)({(x-t)}^3)'dt \\
        = f'(x_0)(x-x_0) + \frac{1}{2}f''(x_0){(x-x_0)}^2 + \cdots + \frac{1}{n!}f^{(n)}(x_0){(x-x_0)}^n \\ 
        + \frac{1}{n!}\int_{x_0}^{x}f^{(n+1)}(t){(x-t)}^n dt.
    \end{multline*}
\end{proof}

\begin{theorem}
{
    (o całkowaniu przez podstawienie). Jeśli funkcja $f : [a,b] \mapsto \mathbb{R}$ jest ciągła, a $\phi : [\alpha, \beta] \mapsto [a,b]$ ma ciągłą
    pochodną na przedziale $[\alpha, \beta]$ oraz $\phi(\alpha) = a$, $\phi(\beta) = b$, to
    \[
        \int_{a}^{b}f(x)dx = \int_{\alpha}^{\beta}f(\phi(t))\phi'(t)dt.
    \]
}
\end{theorem}
\begin{proof}
    Ponieważ funkcje podcałkowe są ciągłe, zatem całki bo obu stronach powyższej równości istnieją.
    Jeśli $F$ jest funkcją pierwotną funkcji $f$ na przedziale $[a,b]$, to $\Phi = F \circ \phi$ jest funkcją pierwotną funkcji
    $(f \circ \phi)\phi'$ na przedziale $[\alpha, \beta]$. Na mocy Wzoru Newtona-Leibniza mamy zatem
    \[
        \int_{a}^{b}f(x)dx = F(b) - F(a) 
    \]
    oraz
    \[
        \int_{\alpha}^{\beta}f(\phi(t))\phi'(t)dt = F(\phi(\beta)) - F(\phi(\alpha)) = F(b) - F(a),
    \]
    co kończy dowód twierdzenia.
\end{proof}

Powyższe twierdzenie jest wystarczające dla wielu zastosowań. Można udowodnić następujące jego uogólnienie:

\begin{theorem}
{
    Niech $\phi : [\alpha, \beta] \mapsto [a,b]$ będzie ściśle monotinicznym przekształceniem przedziału
    $[\alpha, \beta]$ na przedział $[a,b]$ i niech pochodna $\phi'$ będzie R-całkowalna na $[a,b]$. Wówczas dla dowolnej funkcji $f$ R-całkowalnej
    na przedziale $[a,b]$ funkcja $(f \circ \phi)\phi'$ jest R-całkowalna na $[\alpha, \beta]$ oraz zachodzi równość
    \[
        \int_{\phi(\alpha)}^{\phi(\beta)}f(x)dx = \int_{\alpha}^{\beta}f(\phi(t))\phi'(t)dt.
    \]
}
\end{theorem}

Dowód powyższego twierdzenia można znaleźć w książce [9].

\begin{ex}
    Obliczmy $\int_{\frac{1}{\pi}}^{\frac{2}{\pi}}\frac{1}{x^2}\sin \frac{1}{x}dx$. Niech $f(x) = \frac{1}{x^2}\sin \frac{1}{x}$, $x \in [\frac{1}{\pi}, \frac{2}{\pi}]$
    oraz niech $\phi(t) = \frac{1}{t}$, $t \in [\frac{\pi}{2}, \pi]$. Na mocy twierdzenia $\ref{theorem:90}$ otrzymujemy
    \[
        \int_{\frac{1}{\pi}}^{\frac{2}{\pi}}\frac{1}{x^2}\sin \frac{1}{x}dx = 
        \int_{\pi}^{\frac{\pi}{2}}t^2 \sin t(-\frac{1}{t^2})dt = 
        \int_{\frac{\pi}{2}}^{\pi}\sin t dt = -\cos t \big|_{\frac{\pi}{2}}^{\pi} = 1.
    \]
\end{ex}

Udowodnimy teraz dwa twierdzenia całkowe o wartości średniej.

\begin{theorem}
{
    (I Twierdzenie Całkowe o Wartości Średniej). Niech funkcje $f$, $g$ będą R-całkowalne na przedziale
    $[a,b]$ oraz niech $m = \inf\{f(x) : x \in [a,b]\}$, $ M = \sup\{f(x) : x \in [a,b]\}$. Jeśli funkcja $g$
    jest nieujemna lub niedodatnia na $[a,b]$, to
    \begin{equation}\label{eq:50}
        \int_{a}^{b}(fg)(x)dx = \mu \int_{a}^{b}g(x)dx,
    \end{equation}
    gdzie $\mu \in [m, M]$. Jeśli ponadto funkcja $f$ jest ciągła na przedziale $[a,b]$, to istnieje punkt 
    $\xi \in [a,b]$ taki, że 
    \begin{equation}\label{eq:51}
        \int_{a}^{b}(fg)(x)dx = f(\xi)\int_{a}^{b}g(x)dx.
    \end{equation}
}
\end{theorem}

\begin{proof}
    Dla ustalenia uwagi załóżmy, że $g(x) \geqslant 0$ dla $x \in [a,b]$. Wówczas $mg(x) \leqslant f(x)g(x) \leqslant Mg(x)$ dla $x \in [a,b]$, a zatem
    \begin{equation}\label{eq:52}
        m\int_{a}^{b}g(x)dx \leqslant \int_{a}^{b}(fg)(x)dx \leqslant M\int_{a}^{b}g(x)dx.
    \end{equation}
    Jeśli $\int_{a}^{b}g(x) = 0$, to z (\ref{eq:52}) wynika natychmiast (\ref{eq:50}).
    Natomiast gdy $\int_{a}^{b}g(x) \neq 0$, to biorąc
    $\mu = (\int_{a}^{b}g(x)dx)^{-1} \int_{a}^{b}(fg)dx$ i uwzględniając (\ref{eq:52}) otrzymujemy
    $m \leqslant \mu \leqslant M$. 
    Równość (\ref{eq:51}) wynika z (\ref{eq:50}) i z tego, że funkcja ciągła na przedziale domkniętym osiąga swoje kresy i ma Właśność Darboux.
\end{proof}

\begin{wniosek}
{
    Jeśli $g(x) = 1$ dla $x \in [a,b]$, to wzór (\ref{eq:51}) przyjmuje postać
    \[
        \int_{a}^{b}f(x)dx = f(\xi)(b-a).
    \]
}
\end{wniosek}

\begin{theorem}
{
    (II Twierdzenie Całkowe o Wartości Średniej). Jeśli funkcje $f$ i $g$ są R-całkowalne na przedziale $[a,b]$ i ponadto funkcja $g$ jest monotoniczna,
    to istnieje punkt $\xi \in [a,b]$ taki, że
    \begin{equation}\label{eq:53}
        \int_{a}^{b}(fg)(x)dx = g(a)\int_{a}^{\xi}f(x)dx + g(b)\int_{\xi}^{b}f(x)dx.
    \end{equation}
}
\end{theorem}

\begin{proof}
    Dowód przeprowadzimy w szczęgolnym przypadku, gdy funkcja $f$ jest ciągła, a funkcja $g$ jest klasy $C^1$. Dowód w przypadku ogólnym można znależć np. w [2], t. II, s. 101-102 lub w [11], s. 359-363. \\
    Niech $F$ będzie dowolną funkcją pierwotną funkcji $f$, czyli $F'=f$. Całkując przez części otrzymujemy
    \[
        \int_{a}^{b}f(x)g(x)dx = \int_{a}^{b}F'(x)g(x)dx - F(x)g(x) \big|_a^b - \int_{a}^{b}F(x)g'(x)dx.
    \]
    Ponieważ funkcja $g$ jest monotoniczna, zatem jej pochodna na przedziale $[a,b]$ ma stały znak, zatem na mocy
    Twierdzenia \ref{theorem:92} mamy
    \[
        \int_{a}^{b}F(x)g'(x) = F(\xi)\int_{a}^{b}g'(x)dx
    \]
    dla pewnego $\xi \in [a,b]$. Stąd
    \begin{align*}
        \int_{a}^{b}f(x)g(x)dx &= F(b)g(b) - F(a)g(a) - F(\xi)g(b) + F(\xi)g(a) \\
                               &= g(a)(F(\xi) - F(a)) + g(b)(F(b) - F(\xi)) \\
                               &= g(a)\int_{a}^{\xi}f(x)dx + g(b)\int_{\xi}^{b}f(x)dx.
    \end{align*}
\end{proof}

\begin{uwaga}
    II Twierdzenie Całkowe o Wartości Średniej bywa podawane w różnych postaciach.
    \begin{enumerate}
        \item[(a)] 
            Jeśli w przedziale $[a,b]$ funkcja $g$ jest nierosnąca i nieujemna, a funckja $f$ jest R-całkowalna to
            \begin{equation}\label{eq:54}
                \int_{a}^{b}(fg)(x)dx = g(a)\int_{a}^{\xi}f(x)dx,
            \end{equation}
            gdzie $\xi$ jest pewnym punktem z przedziału $[a,b]$. 
        \item[(b)] 
            Analogicznie, jeśli funkcja $g$ jest niemalejąca i nieujemna, to zachodzi wzór
            \begin{equation}\label{eq:55}
                \int_{a}^{b}(fg)(x)dx = g(b)\int_{\xi}^{b}f(x)dx,
            \end{equation}
            gdzie $\xi \in [a,b]$.
    \end{enumerate}
    Wzory (\ref{eq:53}), (\ref{eq:54}), (\ref{eq:55}) nazywają się wzorami Bonneta.
\end{uwaga}

\subsection{Całka z funkcji o wartościach zespolonych}
\begin{defn}
    Niech $f_1$, $f_2$ będą funkcjami rzeczywistymi określonymi na przedziale $[a,b]$ i niech $f = (f_1, f_2)$ będzie odwzorowaniem przedziału $[a,b]$ w zbiór $\mathbb{C}$.
    Mówimy, że $f$ jest całkowalna w sensie Riemanna na przedziale $[a,b]$, jeśli funkcje $f_1$, $f_2$ są R-całkowalne na tym przedziale.
    W tym wypadku określamy
    \[
        \int_{a}^{b}f(x)dx = \Bigg(\int_{a}^{b}f_1(x)dx, \int_{a}^{b}f_2(x)dx\Bigg),
    \]
    lub równoważnie
    \[
        \int_{a}^{b}f(x)dx = \int_{a}^{b}\text{Re} f(x)dx + i \int_{a}^{b} \text{Im} f(x)dx.
    \]
\end{defn}

Jest oczywiste, że Twierdzenie \ref{theorem:82} (dla sumy oraz iloczynu przez liczbę rzeczywistą funkcji R-całkowalnych)
jest prawdziwe także dla funkcji o wartościach w $\mathbb{C}$. To samo dotyczy twierdzeń 
\ref{theorem:85}, \ref{theorem:86}, \ref{theorem:88}, \ref{theorem:90} (por. [1], s. 272-273).
Aby się o tym przekonać, należy jedynie zastosować poprzednie rezultaty do poszczególnych współrzędnych.
Dla przykłady sformułujemy Podstawowe Twierdzenie Rachunku Całkowego.

\begin{theorem}
{
    Niech $f$ i $F$ będą funkcjami określonymi na przedziale $[a,b]$ o wartościach w $\mathbb{C}$. Jeśli odwzorowanie $f$ jest R-całkowalne na tym przedziale
    oraz $F'(x) = f(x)$ dla $x \in [a,b]$, to
    \[
        \int_{a}^{b}f(x)dx = F(b) - F(a).
    \]
}
\end{theorem}

Prawdziwy jest również analog Twierdzenia \ref{theorem:83} (b), jednakże jego dowód jest bardziej subtelny.

\begin{theorem}
{
    Jeśli odwzorowanie $f : [a,b] \mapsto \mathbb{C}$ jest R-całkowalne, to funkcja $|f|$ jest również R-całkowalna oraz
    \[
        \Big| \int_{a}^{b}f(x)dx \Big| \leqslant \int_{a}^{b} |f(x)|dx.
    \]
}
\end{theorem}

\begin{proof}
    Niech $f_1$, $f_2$ będą składowymi odwzorowania $f$ (to znaczy $f = (f_1, f_2)$).
    Wówczas $|f| = \sqrt{f_1^2+f_2^2}$. Każda z funkcji $f_1^2$, $f_2^2$ jest R-całkowalna, więc z ciągłości
    pierwiastka i Twierdzenia \ref{theorem:81} wynika, że funkcja $|f|$ jest również całkowalna.
    Niech $y = (y_1, y_2)$, gdzie $y_i = \int_{a}^{b}f_i(x)dx$ dla $i = 1,2$. Wówczas
    $y = \int_{a}^{b}f(x)dx$ oraz
    \[
        |y|^2 = y_1^2 + y_2^2 = y_1 \int_{a}^{b}f_1(x)dx + y_2\int_{a}^{b}f_2(x)dx = 
        \int_{a}^{b}(y_1f_1(x) + y_2f_2(x))dx.
    \]
    Na podstawie nierówności Schwarza mamy 
    \[
        y_1f_1(x) + y_2f_2(x) \leqslant |y||f(x)| \text{ dla } x \in [a,b],
    \]
    a zatem
    \[
        |y|^2 \leqslant |y|\int_{a}^{b}|f(x)|dx.
    \]
    Dzieląc ostatnią nierówność przez $|y| \neq 0$, otrzymujemy tezę (dla $y = 0$ twierdzenie jest oczywiste).
\end{proof}

W przypadku funkcji R-całkowalnych określonych na przedziale $[a,b]$ o wartościach w $\mathbb{C}$, Wniosek \ref{wniosek:24}
nie zachodzi (por. Rozdział 7.6). Prawdziwe jest natomiast następujące

\begin{theorem}
{
    Jeśli $f : [a,b] \mapsto \mathbb{C}$ jest funkcja R-całkowalną oraz $f([a,b]) \subset B(x^0, r)$,
    gdzie $B(x^0, r)$ oznacza kule domkniętą o środku w punkcie $x^0 \in \mathbb{C}$ i promieniu $r$, to
    \begin{equation}\label{eq:56}
        \frac{1}{b-a}\int_{a}^{b}f(x)dx \in B(x^0, r).
    \end{equation}
}
\end{theorem}

\begin{proof}
Rozważmy funkcję $g(t) = f(t) - x^0$, $t \in [a,b]$. Mamy
\begin{align*}
    \Bigg|\frac{1}{b-a}\int_{a}^{b}f(t)dt - x^0\Bigg| &= \frac{1}{b-a}\Bigg|\int_{a}^{b}(f(t)-x^0)dt\Bigg| \\
    &\leqslant \frac{1}{b-a}\int_{a}^{b}|f(t)-x^0|dt \leqslant \frac{1}{b-a}(b-a)r = r,
\end{align*}
a zatem (\ref{eq:56}) zachodzi.
\end{proof}

\subsection{Zastosowania całki Riemanna}
Wiele zastosowań całki Riemanna opiera się na następującym twierdzeniu, które podaje warunek na to, aby
addytywna funkcja przedziału była generowana przez całkę. 

\begin{theorem}
{
    Jeśli dla addytywnej funkcji $J(\alpha, \beta)$ określonej dla punktów $\alpha, \beta \in [a,b]$ istnieje
    funkcja R-całkowalna na $[a,b]$ i taka, że 
    \[
        \inf_{x \in [\alpha, \beta]} f(x)(\beta - \alpha) \leqslant J(\alpha, \beta) \leqslant \sup_{x \in [\alpha, \beta]} f(x)(\beta - \alpha)
    \]
    dla dowolnych $a \leqslant \alpha < \beta \leqslant b$, to
    \[
        J(a,b) = \int_{a}^{b}f(x)dx.
    \]
}
\end{theorem}

\begin{proof}
    Niech $ P = \{x_0, \ldots, x_n\}$ będzie dowolnym podziałem przedziału $[a,b]$ i niech
    $m_i = \inf\{f(x) : x \in [x_{i-1}, x_i]\}$ oraz $M_i = \sup\{f(x) : x \in [x_{i-1}, x_i]\}$ ($i = 1, \ldots, n$).
    Dla dowolnego przedziału $[x_{i-1}, x_i]$ mamy 
    \[
        m_i \Delta x_i \leqslant J(x_{i-1}, x_i) \leqslant M_i \Delta x_i.
    \]
    Sumując powyższe nierówności i korzystając z addytywności funkcji $J(\alpha, \beta)$ otrzymujemy
    \[
        L(f, P) = \sum_{i=1}^{n}m_i \Delta x_i \leqslant J(a,b) \leqslant \sum_{i = 1}^{n}M_i \Delta x_i = U(f, P).
    \]
    Niech będzie dane dowolne $\varepsilon > 0$. Wówczas istnieje taki podział $P$ przedziału $[a,b]$, że
    $U(f, P) - L(f, P) < \varepsilon$. Mamy zatem
    \[
        \int_{a}^{b}f(x)dx - \varepsilon < L(f, P) \leqslant J(a,b) \leqslant U(f, P) < \int_{a}^{b}f(x)dx + \varepsilon,
    \]
    czyli $\big|J(a,b) - \int_{a}^{b}f(x)dx \big| < \varepsilon$, a zatem wobec dowolności $\varepsilon > 0$ otrzymujemy tezę.
\end{proof}

Omówimy teraz kilka geometrycznych zastosowań całki Riemanna. Zajmiemy się wpierw zagadnieniem długości krzywej.

\begin{defn}
    Ciągłe odwzorowanie $\gamma$ przedziału $[a,b]$ w zbiór $\mathbb{R}$ lub $\mathbb{C}$ nazywamy krzywą
    lub drogą w $\mathbb{R}$ lub $\mathbb{C}$. Punkty $A = \gamma(a)$, $B = \gamma(b)$, nazywają się odpowiednio
    początkiem i końcem drogi. Jeśli $\gamma(a) = \gamma(b)$, to powiemy, że $\gamma$ jest krzywą zamkniętą.
    Zbiór $\gamma([a,b])$ nazywamy obrazem krzywej $\gamma$. 
\end{defn}

\begin{uwaga}
    Zauważmy, że jeden i ten sam zbiór może być obrazem wielu różnych krzywych.
    Ponadto może on okazać się nie tym, co w naszym potocznym wyobrażeniu jest linią.
    Istnieją przykłady krzywych, które wypełniają cały kwadrat jednostkowy (tak zwane ,,Krzywe Peano'',
    zob. R. Eugelking, K. Sieklucki, Geometria i topologia, cz-II, s. 134-135).
\end{uwaga}

\begin{defn}
Krzywą $\gamma : [a,b] \mapsto \mathbb{R}$ (lub $\mathbb{C}$) nazywamy łukiem, jeśli odwzorowanie $\gamma$ jest wzajemnie
jednoznacznie. Krzywą zamkniętą $\gamma : [a,b] \mapsto \mathbb{R}$ (lub $\mathbb{C}$) nazywamy łukiem zamkniętym, jeśli funkcja $\gamma$ jest wzajemnie jednoznaczna na przedziale $[a,b)$.
\end{defn}

\begin{defn}
    Z każdym podziałem $P = \{x_0, x_1, \ldots x_n\}$ przedziału $[a,b]$ i z dowolną krzywą $\gamma$ w
    $\mathbb{R}$ lub $\mathbb{C}$ wiążemy liczbę
    \[
        V(P, \gamma) = \sum_{i=1}^{n}|\gamma(x_i)-\gamma(x_{i-1})|,
    \]
    gdzie $i$-ty składnik powyższej sumy oznacza odległość pomiędzy puntkami $\gamma(x_{i-1})$ oraz
    $\gamma(x_{i})$. Zatem $V(P, \gamma)$ jest długością łamanej o wierzchołkach $\gamma(x_1), \ldots \gamma(x_n)$ występujacych w takim porządku. \\
    Długością krzywej $\gamma$ nazywamy liczbę
    \[
        L(\gamma) = l(a,b) = \sup_P V(P, \gamma),
    \]
    gdzie kres górny jest wzięty po zbiorze wszytkich podziałów przedziału $[a,b]$.
    Jeśli $L(\gamma) < +\infty$, to powiemy, że $\gamma$ jest krzywą prostowalną. 
\end{defn}

W pewnych przypadkach możemy obliczać $L(\gamma)$ jako całkę Riemanna.

\begin{defn}
Krzywa $\gamma : [a,b] \mapsto \mathbb{R}$ (lub $\mathbb{C})$ nazywa się krzywą danej klasy gładkości,
jeśli funkcja $\gamma$ należy do tej klasy (to znaczy do klasy $C^{(n)}$ przy pewnym $n$).
Krzywą klasy $C^1$ nazywamy krzywą gładką. Natomiast mówimy, że krzywa jest kawałkami gładka, jeśli przedział $[a,b]$ można 
podzielić na skończoną liczbę przedziałów tak, że na każdym z nich funkcja $\gamma$ jest klasy $C^1$.
\end{defn}

\begin{theorem}
{
    Jeśli krzywa $\gamma : [a,b] \mapsto \mathbb{R}$ (lub $\mathbb{C}$) jest gładka, to $\gamma$ jest prostowalna oraz
    \begin{equation}\label{eq:57}
        L(\gamma) = \int_{a}^{b}|\gamma'(t)|dt.
    \end{equation}
}
\end{theorem}

\begin{proof}
    Jest jasne, że dla $a \leqslant \alpha < \beta \leqslant b$ zachodzi równość $L(\alpha, \gamma) = L(\alpha, \beta) + L(\beta, \gamma)$.
    Ponadto wobec Twierdzenia Lagrange'a (Tw. 62) mamy
    \[
        \inf_{t \in [\alpha, \beta]} |\gamma'(t)|(\beta - \alpha) \leqslant L(\alpha, \beta) \leqslant \sup_{t \in [\alpha, \beta]} |\gamma'(t)|(\beta - \alpha).
    \]
    Wzór (\ref{eq:57}) jest więc konsekwencją Twierdzenia \ref{theorem:97} (to, że krzywa $\gamma$ jest prostowalna wynika z Twierdzenia Lagrange'a i założenia, że
    $\gamma \in C^1$; mamy bowiem $L(\gamma) \leqslant \sup\limits_{t \in [a,b]} \gamma'(t)(b-a) < +\infty)$.
\end{proof}

\begin{uwaga}
    \begin{itemize}
        \item [(a)] 
            Niech będzie dana krzywa na płaszczyźnie (w $\mathbb{C}$) o równaniach
            $x = x(t)$, $y = y(t)$ ($\gamma(t) = (x(t), y(t))), t \in [a,b]$. Jeśli krzywa $\gamma$ jest gładka, to na mocy
            \ref{eq:57} jej długość wyrażą się wzorem
            \begin{equation}\label{eq:58}
                L(\gamma) = \int_{a}^{b}\sqrt{x'(t)^2 + y'(t)^2}dt.
            \end{equation}
        \item [(b)] 
            Długość wykresu funkcji $y = f(x)$, $x \in [a,b]$, klasy $C^1$ wyraża się wzorem
            \[
                L(a,b) = \int_{a}^{b}\sqrt{1 + f'(x)^2}dx.
            \]
            Wynika to ze wzoru (\ref{eq:58}) bowiem $\{(x, f(x)) : x \in [a,b]\} = \gamma([a,b])$, gdzie
            $\gamma(t)=(t, f(t))$ dla $t \in [a,b]$. 
        \item [(c)]
            \textbf{Współrzędne biegunowe} \\
            Przez dowolny punkt $O$ zwany biegunem poprowadzimy oś $S$, mającą początek w tym punkcie.
            Współrzędnymi biegunowymi punktu $P$ nazywamy liczbe $r$ będącą długościa wektora $\overrightarrow{OP}$ i liczbę
            $\phi \in [0, 2\pi)$ będącą miarą łukową kąta skierowanego $\angle{(S, \overrightarrow{OP})}$. Liczbę $\phi$ nazywamy amplitudą
            punktu $P$ (biegunowi, to znaczy punktowi $O$, można przyporządkować dowolną amplitudę), natomiast $r$ - promieniem wodzącym punktu $P$. \\
            Obierzmy układ kartezjański prostokątny i układ biegunowy tak, by biegun leżał w początku układu kartezjańskiego, a oś biegunowa pokrywała się z osią odciętych.
            Oznaczmy przez $(x, y)$ współrzędne prostokątne oraz przez $(r, \phi)$ - współrzędne biegunowe tego samego punktu w obu układach. Wówczas zachodzą następujące związki:
            \begin{align*}
                x &= r\cos \phi \\
                y &= r \sin \phi \\
                r &= \sqrt{x^2+y^2} \text{ dla } x^2 + y^2 > 0 \\
                & \cos \phi = \frac{x}{\sqrt{x^2+y^2}}, & \sin \phi = \frac{y}{\sqrt{x^2+y^2}}.
            \end{align*}
            Załóżmy, że funkcja $g$ klasy $C^1$ jest określona we współrzędnych biegunowych, to znaczy $r = g(\phi)$, gdzie $\phi_1 \leqslant \phi \leqslant \phi_2$. 
            Za pomocą wzorów $x = r\cos \phi = g(\phi)\cos \phi$, $y = r\sin \phi = g(\phi)\sin \phi$ otrzymujemy przedstawienie parametryczne funkcji $g$, zatem
            \[
                L(\phi_1, \phi_2) = \int_{\phi_1}^{\phi_2} \sqrt{g(\phi)^2 + g'(\phi)^2}d\phi
            \]
        \item [(d)]
            Gdy krzywa jest kawałkami gładka, to aby obliczyć jej długość dzielimy ją na skończoną liczbę krzywych gładkich i do każdej z nich stosujemy wzór (\ref{eq:57}), a następnie dodajemy otrzymane liczby.
    \end{itemize}
\end{uwaga}

Zajmiemy się teraz zastosowaniem całek Riemanna do obliczania pola powierzchni.

\begin{defn}
    Rozpatrzmy na płaszczyźnie dowolną figurę $P$, która jest obszarem ograniczonym. 
    Załóżmy, że brzeg figury $P$ jest obrazem gładkiej krzywej zamkniętej $\gamma : [a,b] \mapsto \mathbb{C}$ (lub składa się z obrazów kilku takich krzywych).
    Zakładamy, że $\gamma'(t) \neq 0$ dla każdego $t \in [a,b]$. Jordan udowodnił, że rozważana krzywa zamknięta rozcina płaszczyznę na dwa obszary,
    wewnętrzny i zewnętrzny, dla których jest ona wspólnym brzegiem. Rozważmy wszystkie możliwe wielokąty $A$, całkowicie zawarte w $P$ i wielokąty $B$, całkowicie zawierające obszar $P$.
    Jeśli $|A|$ i $|B|$ oznaczają pola tych wielokątów, to $|A| \leqslant |B|$. Zbiór $\{|A| : A \subset P\}$ jest ograniczony z góry przez którąkolwiek z liczb $|B|$, a zatem na mocy
    Aksjomatu Dedekinda posiada on kres górny $|P_*|$. Natomiast zbiór $\{|B| : P \subset B\}$ jest ograniczony z dołu przez liczbę $|P_*|$ i posiada kres dolny $|P^*|$. Kres
    górny $|P_*|$ nazywamy wewnętrzną, a kres dolny $|P^*|$ --- zewnętrzną miarą Jordana figury $P$. \\
    Jeśli $|P_*| = |P^*|$, to mówimy, że figura $P$ jest mierzalna w sensie Jordana, a wspólną wartość tych kresów nazywamy miarą Jordana lub polem figury $P$ i oznaczamy symbolem $|P|$. \\
    Będziemy mówili, że figura ma pole równe zeru, jeśli można pokryć ją obszarem wielokątnym o dowolnie małym polu. 
\end{defn}

\begin{uwaga}
    \begin{itemize}
        \item [(a)] 
            Umieścmy rozpatrywaną figurę $P$ wewnątrz prostokąta $R$ o bokach równoległych do osi układu współrzędnych.
            Prostokąt ten rozbijamy na części za pomocą pewnej liczby prostych równoległych do jego boków.
            Oznaczmy przez $\tilde{A}$ figurę złożoną z prostokątów całkowicie zawartych w $P$, natomiast
            przez $\tilde{B}$ --- figurę złożoną z prostokątów mających punkty wspólne z $P$. Przez $d$ oznaczmy długość najdłuższej z przekątnych
            prostokątów. Można udowodnić, że figura $P$ jest mierzalna w sensie Jordana wtedy i tylko wtedy, gdy
            przy $d \rightarrow 0$ obydwa pola $|\tilde{A}|$ i $|\tilde{B}|$ dążą do wspólnej granicy $|P|$; jeśli 
            ten warunek jest spełniony, to wspólna granica $|P|$ jest równa polu figury $P$ ([2], t. II, s. 163 --- 164).
        \item [(b)]
            Załóżmy, że figura $P$ jest rozcięta na dwie figury $P_1$ i $P_2$ (to znaczy figury $P_1$ i $P_2$ nie mają punktów
            wewnętrznych wspólnych). Można udowodnić, że mierzalność dwóch pośród trzech figur $P$, $P_1$, $P_2$, pociąga za sobą mierzalność trzeciej, 
            przy czym $|P| = |P_1| + |P_2|$, to znaczy pole figury ma własność addytywności. 
            ([2], t. II, s. 162).    
    \end{itemize}
    Można nietrudno udowonić ([2], t. II, s. 161, 164 --- 165) następujące kryteria mierzalności figur w sensie Jordana.
\end{uwaga}

\begin{theorem}
{
    \begin{itemize}
        \item [(a)] 
            Na to, by figura $P$ była mierzalna w sensie Jordana, potrzeba i wystarcza, żeby
            dla każdego $\varepsilon > 0$ można było znaleźć takie dwa wielokąty $A \subset P$ i $P \subset B$, że $|B| - |A| < \varepsilon$.
        \item [(b)]
            Na to, żeby figura $P$ była mierzalna w sensie Jordana potrzeba i wystarcza, żeby jej brzeg miał pole równe zeru.
        \item [(c)]
            Jeśli figura $P$ jest ograniczona wykresami kilku funkcji ciągłych, z których każda jest w postaci $y = f(x)$, $x \in [a,b]$ lub $x = g(y)$, $y \in [c,d]$, to
            figura jest mierzalna w sensie Jordana.
    \end{itemize}
}
\end{theorem}

Niech $f$ będzie nieujemną funkcją ciągłą określoną na przedziale $[a,b]$. 
Niech $A = f(a)$, $B = f(b)$. Rozważmy figurę $aABb$ zwaną trapezem krzywolinowym. 
Jest ona ograniczona pionowymi odcinkami $aA$, $bB$, odcinkiem $[a,b]$ na osi $OX$ oraz wykresem funkcji $f$.
Dla $a \leqslant \alpha < \beta \leqslant b$ oznaczmy przez $S(\alpha, \beta)$ pole trapezu krzywolinowego
$\alpha f(\alpha) f(\beta) \beta$. Ponadto przyjmijmy $S(\beta, \alpha) = -S(\alpha, \beta)$. Wiemy, że pole jest funkcją 
addytywną, to znaczy jeśli $a \leqslant \alpha < \beta < \gamma \leqslant b$, to
\[
    S(\alpha, \beta)  + S(\beta, \gamma) = S(\alpha, \gamma).
\]
Stąd wynika, że $S(\alpha, \beta)$ jest addytywną funkcją przedziału zorientowanego.
Ponieważ pole figury zawartej w innej figurze nie może być większe od pola tej ,,obejmującej'' figury, więc
\[
    \inf\limits_{x \in [\alpha, \beta]} f(x)(\beta - \alpha) \leqslant S(\alpha, \beta) \leqslant \sup\limits_{x \in [\alpha, \beta]} f(x)(\beta - \alpha).
\]
Na mocy Twierdzenia \ref{theorem:97} otrzymujemy następujący wzór na pole trapezu krzywolinowego
\begin{equation}\label{eq:59}
    S(a, b) = \int_{a}^{b} f(x)dx.
\end{equation}
Jest jasne, że jeśli $f(x) \leqslant 0$ dla $x \in [a,b]$, to $S(a,b) = -\int_{a}^{b}f(x)dx$. 
Ponadto, jeśli trapez krzywolinowy $CABD$ jest ograniczony z góry i z dołu wykresami krzywych o równaniach $y = f(x)$ i $y = g(x)$ ($a \leqslant x \leqslant b)$, to znaczy
$f(x) \geqslant g(x)$ dla $x \in [a,b]$, to oznaczając przez $|P|$ pole tego trapezu mamy
\[
    |P| = \int_{a}^{b}|f(x) - g(x)|dx.
\]
Niech teraz będzie dany wycinek $AOB$ ograniczony promieniami $OA$, $OB$ (każdy z promieni
$OA$, $OB$ może być punktem) i wykresem funkcji ciągłej o równaniu biegnowym
$r = g(\phi) > 0$ dla $\phi_1 \leqslant \phi \leqslant \phi_2$ gdzie $\phi_2 - \phi_1 < 2\pi$ ($A = g(\phi_1)$, $B = g(\phi_2)$). Oznaczmy
przez $S(\phi_1, \phi_2)$ pole wycinka $AOB$. Mamy
\[
    \inf\limits_{\phi \in [\phi_\alpha, \phi_\beta]} \frac{1}{2}g^2(\phi)(\phi_\beta - \phi_\alpha)
    \leqslant S(\phi_\alpha, \phi_\beta) \leqslant
    \sup\limits_{\phi \in [\phi_\alpha, \phi_\beta]} \frac{1}{2}g^2(\phi)(\phi_\beta - \phi_\alpha)
\]
(przypomnijmy, że pole wycinka kołowego o promieniu $r$ i kącie środkowym $\Delta \phi$ wyraża się wzorem
$\frac{1}{2}r^2\Delta \phi$). Stąd otrzyjmujemy
\[
    S(\phi_1, \phi_2) = \frac{1}{2} \int_{\phi_1}^{\phi_2}r^2 d\phi.
\]
Roważmy, jeszcze przypadek, gdy mamy krzywą gładką o równaniach $x = x(t)$, $y = y(t)$, $t \in [a,b]$. Jeśli $y(t) \geqslant 0$
oraz $x'(t) > 0$ w przedziale $[a,b]$, to pole obszaru $D$, zawartego między obrazem tej krzywej, osią $OX$ i rzędnymi w punktach końcowych krzywej,
wyraża się wzorem
\[
    |D| = \int_{a}^{b}y(t)x'(t)dt.
\]
Istotnie funkcja $x = x(t)$ jest rosnąca, bowiem $x'(t) > 0$, a zatem posiada funkcję odwrotną $t = t(x)$ w przedziale $[\alpha, \beta]$, gdzie
$\alpha = x(a)$, $\beta = x(b)$. Obraz danej krzywej możemy więc traktować jako wykres funkcji $y = y(t(x))$, $\alpha < x < \beta$, przy czym na mocy wzoru
(\ref{eq:59})
\[
    |D| = \int_{\alpha}^{\beta} y(t(x))dx.
\]
Wykonując w powyższej całce podstawienie $x = x(t), a \leqslant t \leqslant b$, otrzymujemy żądany wzór (por. Tw. 90).

\noindent
Jeśli natomiast $x'(t) < 0$ w przedziale $[a,b]$, to $|D| = -\int_{a}^{b}y(t)x'(t)dt$.

Całka Riemanna posiada również zastosowania do obliczania objętności pewnych brył oraz pól powierzchni obrotowych.
Niech będzie dana bryła $V$, to znaczy obszar ograniczony w przestrzeni trójwymiarowej.
Załóżmy, że brzeg $S$ bryły $V$ jest powierzchnią zamkniętą (lub składa się z kilku takich powierzchni).

\begin{defn}
    Rozpatrzmy wielościany $X$ o objętości $|X|$, zawarte całkowicie w bryle $V$ oraz wielościany
    $Y$ o objętości $|Y|$ zawierjące w sobie całą bryłe $V$. Istnieje kres górny
    $|V_*|$ liczb $|X|$ i kres dolny $|V^*|$ liczb $|Y|$, przy czym $|V_*| \leqslant |V^*|$; kresy te nazywamy
    odpowiednio wewnętrzną i zewnętrzną objetością bryły $V$. Jeśli oba kresy są równe, to ich wspólną
    wartość nazywamy objętością bryły $V$. Bryła ma objętość równą zeru, jeśli można umieścić ją w bryle
    wielościennej o dowolnie małej objętości.
\end{defn}

Również w tym przypadku są prawdziwe odpowiedniki Uwagi \ref{uwaga:47} i Twierdzenia \ref{theorem:99}, w których odpowiednio
zastąpimy figury --- bryłami, wielokąty --- wielościanami, prostokąty --- prostopadłościanami, pola --- objętościami, funkcje ciągłe jednej zmiennej
--- funkcjami ciągłymi dwóch zmiennych. \\
W szczególności można udowodnić (por. [2], t. II, s. 177 --- 178), że jeśli trape
krzywolinowy $aABb$ ,,obrócimy'' względem osi $OX$, to otrzymana w ten sposób bryła posiada objętość.
Istotnie ponieważ wykres funkcji ciągłej $y = f(x)$, $x \in [a,b]$ można pokryć prostokątami o dowolnie małym polu,
zatem brzeg rozważanej bryły można pokryć bryłami (pierścieniami walcowymi) o dowolnie małej objętości (to znaczy suma objętości tych brył jest dowolnie mała). \\
Oznaczmy przez $V(\alpha, \beta)$ objętość bryły otrzymanej przez obrót trapezu krzywolinowego $\alpha f(\alpha) f(\beta) \beta$ dookoła osi $OX$. Z właśności objętości brył
otrzymujemy następujące związki: jeśli $a \leqslant \alpha < \beta < \gamma \leqslant b$, to
\[
    V(\alpha, \gamma) = V(\alpha, \beta) + V(\beta, \gamma)
\]
\begin{center}
    oraz
\end{center}
\[
    \pi\Bigg( \inf\limits_{x \in [\alpha, \beta]}f(x)\Bigg)^2(\beta - \alpha)
    \leqslant V(\alpha, \beta) \leqslant 
    \pi\Bigg( \sup\limits_{x \in [\alpha, \beta]}f(x)\Bigg)^2(\beta - \alpha),
\]
przy czym w powyższych nierównościach zastosowaliśmy wzór na objętość walca,
łatwo wynikający z Definicji \ref{def:83} (zob. [2], t. II, s. 176). Na mocy Twierdzenia \ref{theorem:97} otrzymujemy
\[
    V(a, b) = \pi \int_{a}^{b}f(x)^2dx.
\]
Jest jasne, że jeśli trapez krzywoliniowy jest ograniczony z góry i z dołu wykresami funkcji ciągłych
$y = f_1(x), y = f_2(x)$, $f_1(x) \geqslant f_2(x) \geqslant > 0$ ($x \in [a,b]$), to
\[
    V(a, b) = \pi \int_{a}^{b}|f_1(x)^2 - f_2(x)^2|dx.
\]
Niniejszy paragraf zakończymy krótką informacją o polu powierzchni obrotowej. Ze względu na niewystarczjacy aparat pojęciowy, nie możemy jeszcze
zdefiniować pola powierzchni zakrzywionej. \\
Niech będzie dana krzywa w $\mathbb{C} : \gamma(t) = (x(t), y(t))$, $t \in [a,b]$. Jeśli krzywa $\gamma$ jest gładka,
to pole powierzchni obrotowej powstałej przez obrót obrazu krzywej $\gamma$ dookoła osi $OX$ istnieje i wyraża się wzorem
\[
    |P| = 2\pi \int_{a}^{b} y \sqrt{x'(t)^2 + y'(t)^2}dt.
\]
W szczególności jeśli ,,obracamy'' wykres funkcji $y = f(x)$ ($a \leqslant x \leqslant b$) klasy $C^1$ dookoła osi $OX$, to
\[
    |P| = 2\pi \int_{a}^{b}f(x)\sqrt{1 + f'(x)^2}dx.
\]

\subsection{Całki niewłaściwe}

\begin{defn}
    Niech $f$ będzie funkcją rzeczywistą określoną na przedziale $[a,b)$ i R-całkowalną w każdym przedziale domkniętym $[a, \beta]$, gdzie
    $a < \beta < b$. Wobec tego dla każdego $a < \beta < b$ istnieje całka
    \begin{equation}\label{eq:60}
        J(\beta) = \int_{a}^{\beta} f(x)dx.
    \end{equation}
    Punkt $b$ nazywać będziemy punktem osobliwym funkcji $f$, jeśli albo $b = +\infty$, albo funkcja $f$ nie jest ograniczona na przedziale $[a, b)$. \\
    Jeśli $b$ jest punktem osobliwym funkcji $f$ i całka (\ref{eq:60}) dąży do skończonej granicy, gdy
    $\beta \to b$, to tę granicę nazywamy całką niewłaściwą z funkcji $f$ na przedziale $[a,b)$ i oznaczamy symbolem
    \[
        \int_{a}^{b}f(x)dx.
    \]
    O funkcji $f$ mówimy wówczas, że jest całkowalna w sensie niewłaściwym na przedziale $[a,b)$. 
    Mamy zatem $\int_{a}^{b}f(x)dx = \lim_{\beta \to b}\int_{a}^{\beta}f(x)dx$. Jeśli ta granica nie istnieje, to mówimy, 
    że całka niewłaściwa jest rozbieżna. Analogicznie określamy całkę niewłaściwą z funkcji rzeczywistej określonej na
    przedziale $(a,b]$, gdy a jest punktem osobliwym funkcji $f$, to znaczy $a = -\infty$ albo funkcja $f$ nie jest ograniczona w otoczeniu punktu $a$.
    Zakładąć, że funkcja $f$ jest R-całkowalna na każdym przedziale $[\alpha, b]$, gdzie $a < \alpha < b$, przyjmujemy
    \[
        \int_{a}^{b}f(x)dx = \lim_{\alpha \to a}\int_{\alpha}^{b}f(x)dx
    \]
    (o ile ta granica istnieje). \\
    W przypadku, gdy $b < a$ przyjmujemy
    \[
        \int_{a}^{b}f(x)dx = -\int_{b}^{a}f(x)dx.
    \]
    Jeśli funkcja $f$ ma w przedziale $[a,b]$ skończenie wiele punktów osobliwych,
    to dzielimy ten przedział na przedziały mające po jednym punkcie osobliwym na początku lub na końcu przedziału i wówczas sumę całek
    niewłaściwych odpowiadających tym podprzedziałom nazywamy całką niewłaściwą funkcji $f$ na przedziale $[a,b]$. 
\end{defn}

\begin{ex}
    \begin{itemize}
        \item [(a)]
            Zbadamy, dla jakich $p > 0$ istnieje całka niewłaściwa $I_p = \int_{a}^{b}\frac{dx}{(b-x)^p}$.
            Badamy całki
            \begin{align*}
                J_p(\beta) &= \int_{a}^{\beta}\frac{dx}{(b-x)^p} = 
                    \begin{cases}
                        -\ln(b-x) \big|_a^\beta & \text{dla } p = 1, \\
                        -\frac{(b-x)^{-p+1}}{-p+1}\big|_a^\beta & \text{dla } p \neq 1.
                    \end{cases} \\
                    &= 
                    \begin{cases}
                        -\frac{(b-\beta)^{1-p}}{1-p} + \frac{(b-a)^{1-p}}{1-p} & \text{dla } 0 < p < 1, \\
                        \ln{\frac{b-a}{b-\beta}}& \text{dla } p = 1, \\
                        -\frac{1}{(1-p)(b-\beta)^{p-1}} + \frac{1}{(1-p)(b-a)^{p-1}} & \text{dla } p >  1.
                    \end{cases}
            \end{align*}
            Stąd $\lim_{\beta \to b}J_p(\beta) = 
            \begin{cases}
                (b-a)^{1-p} & \text{dla } 0 < p < 1, \\
                +\infty & \text{dla } p \geqslant 1,
            \end{cases}$
            czyli całka niewłaściwa $I_p = \int_{a}^{b}\frac{dx}{(b-x)^p}$ istnieje dla $0 < p < 1$ i nie istnieje dla $p \geqslant 1$. 
            W szczególności $\int_{0}^{1}\frac{dx}{\sqrt{x}}$ istnieje, natomiast $\int_{0}^{1}\frac{dx}{x}$ nie istnieje.
        \item [(b)]
            Zbadamy teraz, dla jakich $p > 0$ istnieje całka niewłaściwa $K_p = \int_{a}^{\infty} \frac{dx}{x^p}$. Badamy całki
            \begin{align*}
                K_p(\beta) = \int_{a}^{\beta}\frac{dx}{x^p} &= 
                \begin{cases}
                    \ln x \big|_a^\beta & \text{dla } p = 1, \\
                    \frac{x^{-p+1}}{-p+1}\big|_a^\beta & \text{dla } p \neq 1,
                \end{cases} \\
                &= 
                \begin{cases}
                    \frac{\beta^{1-p}}{1-p} - \frac{a^{1-p}}{1-p} & \text{dla } 0 < p < 1, \\
                    \ln \frac{\beta}{a} & \text{dla } p = 1, \\
                    \frac{1}{(1-p)\beta^{p-1}} - \frac{1}{(1-p)a^{p-1}} &\text{dla } p > 1.
                \end{cases}
            \end{align*}
            Zatem $\lim\limits_{\beta \to +\infty} K_p(\beta) = 
                \begin{cases}
                    +\infty &\text{dla } 0 < p \leqslant 1, \\
                    \frac{1}{(p-1)a^{p-1}} &\text{dla } p > 1,
                \end{cases}$ a więc całka niewłaściwa $K_p$ istnieje dla $p > 1$ i nie istnieje, gdy $0 < p \leqslant 1$.
    \end{itemize}
\end{ex}

Następujące twierdzenie podaje podstawowe własności całek niewłaściwych.

\begin{theorem}
{
    \begin{itemize}
        \item [(a)]
            Jeśli funkcja $f$ jest ograniczona na przedziale skończonym $[a,b)$ i jeśli całka $I(\beta) = \int_{a}^{\beta} f(x)dx$ istnieje dla
            każdego $a < \beta < b$, to granica $\lim\limits_{\beta \to b} \int_{a}^{\beta}f(x)dx$ istnieje.
        \item [(b)]
            Jeśli funkcja $f$ jest R-całkowalna na przedziale $[a,b]$, to całka Riemanna i całka niewłaściwa funkcji $f$
            (na przedziałach $[a,b]$ oraz $[a,b)$, odpowiednio) pokrywają się.
        \item [(c)]
            Jeśli funkcje $f, g : [a,b] \mapsto \mathbb{R}$ są całkowalne (w sensie niewłaściwym) na przedziale $[a,b)$, to dla dowolnych
            $\lambda_1, \lambda_2 \in \mathbb{R}$ funkcja $(\lambda_1f + \lambda_2 g)(x)$ jest całkowalna (w sensie całki niewłaściwej) oraz
            \[
                \int_{a}^{b}(\lambda_1 f + \lambda_2 g)(x)dx = \lambda_1 \int_{a}^{b}f(x)dx + \lambda_2 \int_{a}^{b}g(x)dx.
            \]
            Ponadto, jeśli $c \in [a,b]$, to $\int_{a}^{b}f(x)dx = \int_{a}^{c}f(x)dx + \int_{c}^{b}f(x)dx$.
        \item [(d)]
            Załóżmy, że funkcja $f$ jest całkowalna w sensie niewłaściwym na przedziale $[a,b)$ oraz $\phi : [\alpha, \gamma) \mapsto [a,b)$ jest ściśle monotoniczną
            funkcją klasy $C^1$, przy czym $\phi(\alpha) = a$ oraz $\phi(\beta) \to b$ jeśli $\beta \to \gamma$, $\beta \in [\alpha, \gamma)$. 
            Wówczas całka niewłaściwa funkcji $t \to (f \circ \phi)(t)\phi'(t)$ na przedziale $[\alpha, \gamma)$ istnieje oraz
            \[
                \int_{a}^{b}f(x)dx = \int_{a}^{\gamma}(f \circ \phi)(t)\phi'(t)dt
            \]
            (zamiana zmiennych w całkach niewłaściwych).
        \item [(e)]
            Jeśli $f, g \in C^1[a,b)$, to
            \[
                \int_{a}^{b}(fg')(x)dx = (fg)(x)\big|_a^b - \int_{a}^{b}(f'g)(x)dx,
            \]
            gdzie $(fg)(x)\big|\fracnoline{b}{a} = \lim\limits_{x \to b}(fg)(x) - (fg)(x)$, o ile dwa spośród trzech występujących w równości wyrażeń mają sens.
            Stąd już wynika istnienie trzeciego (całkowanie przez części dla całek niewłaściwych). 
    \end{itemize}    
}
\end{theorem}

\newpage

\begin{proof}
    \begin{itemize}
        \item [(a)]
            Istotnie, jeśli $a < \beta < \beta' < b$, to $|I(\beta) - I(\beta')| = |\int_{\beta}^{\beta'}f(x)dx| \leqslant M(\beta' - \beta)$,
            a zatem jeśli $\varepsilon > 0$, $b - \beta < \frac{\varepsilon}{M}$, $b - \beta' < \frac{\varepsilon}{M}$, to $|I(\beta) - I(\beta')| < \varepsilon$.
            Na mocy Kryterium Cauchy'ego (Tw. 36) wnioskujemy, że $\lim\limits_{\beta \to b} I(\beta)$ istnieje.
        \item [(b)]
            Wynika z ciągłości funkcji $I(\beta) = \int_{a}^{\beta}f(x)dx$, na przedziale $[a,b]$, na którym funkcja $f$ jest R-całkowalna.
        \item [(c)]
            Pierwsza część tezy wynika, z tego, że dla $\beta \in [a,b)$ mamy
            \[
                \int_{a}^{\beta}(\lambda_1f + \lambda_2g)(x)dx = \lambda_1 \int_{a}^{\beta}f(x)dx + \lambda_2 \int_{a}^{\beta} g(x)dx.
            \]
            Druga część tezy wynika z równości
            \[
                \int_{a}^{\beta}f(x)dx = \int_{a}^{c}f(x)dx + \int_{c}^{\beta}f(x)dx,
            \]
            która zachodzi dla dowlnych $c, \beta \in [a,b)$. 
        \item [(d)]
            Na podstawie Twierdzenia (\ref{theorem:91}) otrzymujemy
            \[
                \int_{a=\phi(\alpha)}^{\phi(\beta)}f(x)dx = \int_{\alpha}^{\beta}(f \circ \phi)(t)\phi'(t)dt \text{ dla } \beta \in [a,\gamma),
            \]
            a następnie przechodzimy do granicy przy $\beta \to \gamma$. 
    \end{itemize}
\end{proof}

\begin{uwaga}
    \begin{itemize}
        \item [(a)]
            Jeżeli przy założeniach punktu (a) powyższego twierdzenia nadamy funkcji $f$ dowolną wartość, to otrzymamy funkcję R-całkowalną na przedziale $[a,b]$
            oraz całka Riemanna z tej funkcji na przedziale $[a,b]$ jest równa $\lim\limits_{\beta \to b}\int_{a}^{\beta}f(x)dx$. 
            Aby to udowodnić, wystarczy dla przykładu wykorzystać Uwagę \ref{uwaga:38}, a następnie własności funkcji górnej granicy całkowania (dla całki Riemanna).
        \item [(b)]
            Iloczyn dwóch funkcji $f, g : [a,b] \mapsto \mathbb{R}$ całkowalnych w sensie niewłaściwym nie musi być funkcją całkowalną w sensie niewłaściwym. Dla przykładu
            $\int_{0}^{1}\frac{dx}{\sqrt{1-x}}$ istnieje natomiat $\int_{0}^{1}\frac{dx}{\sqrt{1-x}\sqrt{1-x}}$ nie istnieje.
        \item [(c)]
            Dla przykładu wobec Tw. \ref{theorem:100} (a) całka niewłaściwa $\int_{0}^{1}\frac{\sin x}{x}dx$ istnieje, bowiem funkcja $f(x) = \frac{\sin x}{x}$ jest ograniczona
            na przedziale $(0, 1]$ oraz R-całkowalna na każdym przedziale $[\alpha, 1]$, $0 < \alpha < 1$. 
    \end{itemize}
\end{uwaga}

\begin{defn}
    Niech funkcja $f$ ma w przedziale $[a,b]$ skończoną liczbę punktów osobliwych. 
    Jeśli całka niewłaściwa $\int_{a}^{b}f(x)dx$ istnieje, to mówimy, że jest ona zbieżna, natomiast gdy istnieje całka niewłaściwa $\int_{a}^{b}|f(x)|dx$, to o całce
    $\int_{a}^{b}f(x)dx$ mówimy, że jest zbieżna bezwględnie. Całka niewłaściwa, która jest zbieżna, ale nie jest zbieżna bezwględnie, nazywa się całką warunkowo zbieżną.
\end{defn}

Udowodnimy teraz podstawowe kryteria zbieżności całki niewłaściwej.

\begin{theorem}
{
    (Kryterium Cauchy'ego) Niech $b$ będzie jedynym punktem osobliwym funkcji $f$ w przedziale $[a,b]$. Całka niewłaściwa $\int_{a}^{b}f(x)dx$ jest zbieżna wtedy i tylko wtedy, gdy
    dla każdego $\varepsilon > 0$ istnieje taka liczba $\beta_0 \in (a,b)$, że
    \[
        \Bigg|\int_{\beta}^{\beta'}f(x)dx \Bigg| < \varepsilon
    \]
    dla dowolnych liczb $\beta$, $\beta'$, spełniających nierówności $\beta_0 < \beta < \beta' < b$.
}
\end{theorem}

\begin{proof}
    Zbieżność całki niewłaściwej funkcji $f$ oznacza, że $\int_{a}^{b}f(x)dx = \lim\limits_{\beta \to b}I(\beta)$, gdzie $I(\beta) = \int_{a}^{\beta}f(x)dx$.
    Na mocy Twierdzenia 36 jest to równoważne następującemu warunkowi: \\
    Dla dowolnego $\varepsilon > 0$ istnieje $\beta_0 \in (a,b)$ takie, że $|I(\beta) - I(\beta')| = |\int_{\beta}^{\beta'}f(x)dx| < \varepsilon$ 
    dla dowolnych $\beta, \beta'$ spełniających nierówności $\beta_0 < \beta < \beta' < b$, co kończy dowód.
\end{proof}

Odpowiednik Twierdzenia 36 jest prawdziwy również wtedy, gdy $p = +\infty$ lub $p = -\infty$, wtedy
warunek Cauchy'ego przyjmuje postać: \\ 
dla dowolnego $\varepsilon > 0$ istnieje $M \in \mathbb{R}$ takie, że dla dowolnych $x', x'' \in E$, jeśli $x' > M$ oraz $x'' > m$, to
$|f(x')-f(x'')| < \varepsilon$.

\begin{theorem}
{
    (Kryterium Porównawcze) Jeśli funkcje $f$ i $F$ spełniają nierówność $|f(x)| \leqslant F(x)$ dla $a \leqslant x < b$ i $b$ jest
    jedynym punktem osobliwym dla obu funkcji w przedziale $[a,b]$ oraz jeśli funkcja $f$ jest R-całkowalna na każdym przedziale 
    $[a, \beta], \beta < b$ i istnieje całka niewłaściwa $\int_{a}^{b}F(x)dx$, to istnieje również całka niewłaściwa $\int_{a}^{b}f(x)dx$ i jest ona
    bezwględnie zbieżna.
}
\end{theorem}

\begin{proof}
Z Kryterium Cauchy'ego wynika, że dla dowolnego $\varepsilon > 0$ istnieje $\beta_0 \in (a,b)$ takie, że
\[
    \Bigg| \int_{\beta}^{\beta'}F(x)dx \Bigg| < \varepsilon \text{ dla } \beta_0 < \beta < \beta' < b.
\]
Mamy więc
\[
    \Bigg| \int_{\beta}^{\beta'}f(x)dx \Bigg| \leqslant \int_{\beta}^{\beta'}\big|f(x)\big|dx \leqslant \int_{\beta}^{\beta'}F(x)dx = \Bigg|\int_{\beta}^{\beta'}F(x)dx\Bigg| < \varepsilon. 
\]
Z powyższych równości i z Kryterium Cauchy'ego wynika istnienie całek niewłaściwych
$\int_{a}^{b}f(x)dx$, $\int_{a}^{b}|f(x)|dx$.

\end{proof}

\begin{wniosek}
{
    \begin{itemize}
        \item [(a)] 
            Jeśli całka niewłaściwa $\int_{a}^{b}|f(x)|dx$ jest zbieżna, to zbieżna jest również całka niewłaściwa
            $\int_{a}^{b}f(x)dx$ (czyli bezwględna zbieżność pociąga zbieżność).
        \item [(b)]
            Jeśli w przedziale $[a,b)$ funkcje $f$ i $F$ spełniają nierówność $0 \leqslant f(x) \leqslant F(x)$ i całka
            $\int_{a}^{b}f(x)dx$ nie istnieje, to całka niewłaściwa $\int_{a}^{b}F(x)dx$ jest rozbieżna. 
    \end{itemize}
}
\end{wniosek}

\begin{theorem}
{
    (Kryterium Dirichleta) Niech punkt $b$ będzie jedynym punktem osobliwym iloczynu funkcji $f$ i $g$ w przedziale $[a,b]$.
    Jeśli funkcja $F(x) = \int_{a}^{x}f(t)dt$ istnieje i jest ograniczona na przedziale $[a,b)$, a funkcja $g(x)$ dąży monotonicznie do zera, gdy 
    $x \to b$, to całka niewłaściwa $\int_{a}^{b}(fg)(x)dx$ jest zbieżna. 
}
\end{theorem}

\begin{proof}
    Na mocy II Twierdzenia Całkowego o Wartości Średniej mamy
    \[
        \int_{\beta}^{\beta'}(fg)(x)dx = g(\beta)\int_{\beta}^{\xi}f(x)dx + g(\beta')\int_{\xi}^{\beta'}f(x)dx,
    \]
    dla dowolnych $a < \beta < \beta' < b$, gdzie $\xi$ jest pewnym punktem leżącym między $\beta$ i $\beta'$.
    Wobec przyjętych założeń dla dowolnego $\varepsilon > 0$ istnieje $\beta_0 \in (a,b)$ takie, że
    \[
        \Bigg|\int_{\beta}^{\beta'}(fg)(x)dx\Bigg| < \varepsilon \text{ dla } \beta_0 < \beta < \beta' < b.
    \]
    Na mocy kryterium Cauchy'ego całka niewłaściwa $\int_{a}^{b}(fg)(x)dx$ jest zbieżna.
\end{proof}

\begin{theorem}
{
    (Kryterium Abela) Niech $b$ będzie jedynym punktem osobliwym dla iloczynu funkcji $f$ i $g$ w przedziale $[a,b]$.
    Jeśli całka niewłaściwa $\int_{a}^{b}f(x)dx$ jest zbieżna, a funkcja $g$ jest monotoniczna i ograniczona, to całka niewłaściwa
    $\int_{a}^{b}(fg)(x)dx$ jest zbieżna.
}
\end{theorem}

\begin{proof}
    Analogiczny do dowodu Kryterium Dirichleta.
\end{proof}

\begin{uwaga}
    Nietrudno zauważyć, że kryterium Abela wynika z kryterium Dirichleta.
    Niech funkcje $f$ i $g$ spełniają bowiem założenia kryterium Abela. Wówczas istnieje granica
    $\lim\limits_{x \to b}g(x) = g(b)$. Mamy $f(x)g(x) = f(x)g(b) + f(x)[g(x) - g(b)]$. Jest jasne, że drugi składnik tej sumy
    spełnia założenia kryterium Dirichleta. Stąd całka niewłaściwa $\int_{a}^{b}(fg)(x)dx$ jest zbieżna. 
\end{uwaga}

\newpage

\begin{ex}
\begin{itemize}
    \item [(a)]
        Rozważmy całkę niewłaściwą $\int_{0}^{1}\frac{\cos x}{x}dx$. Ponieważ $\frac{\cos x}{x} \geqslant \frac{\cos1}{x} \geqslant 0$ dla $x \in (0, 1]$ oraz $\cos 1 \int_{0}^{1}\frac{dx}{x}$ jest całką niewłaściwą rozbieżną, zatem na mocy Kryterium Porównawczego całka niewłaściwa $\int_{0}^{1}\frac{\cos x}{x}dx$ jest rozbieżna. 
    \item [(b)]
        Rozważmy całkę niewłaściwą $\int_{0}^{\infty}\frac{\sin x}{x}dx$. Niech $g(x) = \frac{1}{x}$ oraz $f(x) = \sin x$ dla $x > 0$. Mamy $\int_{0}^{\infty}\frac{\sin x}{x}dx = \int_{0}^{1}\frac{\sin x}{x}dx + \int_{1}^{\infty}\frac{\sin x}{x}dx$, zatem wobec Uwagi \ref{uwaga:48} (c) wystarczy zbadać zbieżność całki niewłaściwej $\int_{1}^{\infty}\frac{\sin x}{x}dx$. Ponieważ $\int_{1}^{x}\sin t dt = \cos1 - \cos x$ jest funkcją ograniczoną na przedziale $[1, +\infty)$ oraz $\frac{1}{x} \to 0$ monotonicznie przy $x \to \infty$, zatem na mocy Kryterium Dirichleta całka niewłaściwa $\int_{1}^{\infty}\frac{\sin x}{x}dx$ jest zbieżna. \\
        Pokażemy teraz, że całka niewłaściwa $\int_{0}^{\infty}\frac{\sin x}{x}dx$ nie jest zbieżna bezwględnie. Mamy
        \begin{align*}
            \lim\limits_{N \to \infty}\int_{0}^{N\pi}\Big|\frac{\sin x}{x}\Big|dx &= \lim\limits_{N \to \infty} \sum_{k=1}^{N}\int_{(k-1)\pi}^{k\pi}\frac{|\sin x|}{x}dx \\
            & \geqslant \lim\limits_{N \to \infty}\sum_{k=1}^{N}\frac{1}{k\pi}\int_{(k-1)\pi} ^{k \pi}|\sin x|dx \\
            &= \lim\limits_{N \to \infty}\sum_{k=1}^{N}\frac{1}{k\pi}\int_{0}^{\pi}\sin x dx \\
            &= \frac{2}{\pi}\lim\limits_{N \to \infty}\sum_{k=1}^{N}\frac{1}{k} = +\infty,
        \end{align*}
        co kończy dowód.
\end{itemize}
\end{ex}

\subsection{Całka Riemanna-Stieltjesa (względem funkcji monotonicznej)}
Całka Riemanna-Stieltjesa jest bezpośrednim uogólnieniem całki Riemanna.

\begin{defn}
    Niech $f, g : [a,b] \mapsto \mathbb{R}$ będą funkcjami ograniczonymi. Dla dowolnego podziału 
    $P = \{x_0, \ldots, x_n\}$ przedziału $[a,b]$ tworzymy następujące sumy
    \[
        S = \sum_{i=1}^{n}f(\xi_i)(g(x_i)-g(x_{i-1})), \text{ gdzie } \xi_i \in [x_{i-1}, x_i], i = 1, \ldots, n.
    \]
    Sumy te nazywamy sumami Riemanna-Stieltjesa odpowiadąjącymi podziałowi $P$ przy ustalonym wyborze punktów $\xi_i$.
    Przez $S(f, P)$ oznaczać będziemy zbiór wszystkich możliwych sum Riemanna-Stieltjesa odpowiadących podziałowi $P$. \\
    Jeśli dla dowolnego ciągu normalnego podziałów $(P_k)$ i dla dowolnych sum Riemanna-Stieltjesa $S_k \in S(f, P_k)$ istnieje
    skończona granica $I = \lim\limits_{k \to \infty} S_k$, to tę granicę nazywamy całką Riemanna-Stieltjesa funkcji $f$ względem
    funkcji $g$ na przedziale $[a,b]$. O funkcji $f$ mówimy wówczas, że jest całkowalna w sensie Riemanna-Stieltjesa (lub krótko: (R-S) całkowalna) względem
    funkcji $g$ na przedziale $[a,b]$. Całkę Riemanna-Stieltjesa oznaczamy symbolem.
    \[
        \int_{a}^{b}fdg \text{ lub } \int_{a}^{b}f(x)dg(x).
    \]
\end{defn}

\begin{defn}
    Niech $g$ będzie monotonicznie rosnącą funkcją określoną na przedziale $[a,b]$ (ponieważ $g(a)$ i $g(b)$ są skończone, więc funkcja jest
    ograniczona na $[a,b]$). Jeśli $P$ jest jakimś podziałem przedziału $[a,b]$, to określamy $\Delta{g_i} = g(x_i) - g(x_{i-1})$ (oczywiście $\Delta{g_i} \geqslant 0$).
    Dla ograniczonej funkcji $f : [a,b] \mapsto \mathbb{R}$ piszemy
    \[
        U(f, g, P) = \sum_{i=1}^{n}M_i\Delta{g_i}, \quad L(f, g, P) = \sum_{i=1}^{n}m_i\Delta{g_i},
    \]
    gdzie $m_i$ oraz $M_i$ mają ten sam sens, co w Definicji \ref{def:71}. Liczby $U(f, g, P)$ i $L(f, g, P)$
    nazywać będziemy górną i dolną sumą Darboux-Stieltjesa. Dalej
    \[
        \overline{\int_{a}^{b}fdg} = \inf\limits_{P}U(f, g, P), \quad \underline{\int_{a}^{b}fdg} = \sup\limits_{P}L(f, g, P),
    \]
    gdzie kres górny i dolny są wzięte ze względu na wszystkie możliwe podziały przedziału $[a,b]$.
\end{defn}

Rozumując analogicznie jak w dowodzie Twierdzeń \ref{theorem:76} i \ref{theorem:77} otrzymujemy dwa kryteria całkowalności funkcji
w sensie Riemanna-Stieltjesa.

\begin{theorem}
{
    Na to, by ograniczona funckja $f$ była (R-S)-całkowalna względem funkcji rosnącej $g$ na przedziale $[a,b]$
    potrzeba i wysarcza, aby był spełniony jeden z następujących warunków:
    \begin{itemize}
        \item [(a)]
            dla dowolnego $\varepsilon > 0$ istnieje taki podział $P$ przedziału $[a,b]$, że
            \[
                U(f, g, P) - L(f, g, P) < \varepsilon.
            \]
        \item [(b)]
            \[
                \overline{\int_{a}^{b}fdg} = \underline{\int_{a}^{b}fdg}.
            \]
    \end{itemize}
}
\end{theorem}

Zbadamy teraz klasy funkcji (R-S)-całkowalnych. Załóżmy wpierw, że $g$ jest funkcją rosnącą na $[a,b]$.
Wówczas rozumując podobnie jak w dowodzie Twierdzeń \ref{theorem:78} oraz \ref{theorem:80} wnioskujemy, że
jeśli $f$ jest funkcją ciągła lub monotoniczną (w tym przypadku zakładamy dodatkowo, że $g$ jest funkcją ciągła na $[a,b]$), to
jest ona (R-S)-całkowalna względem funkcji $g$ na $[a,b]$.

\begin{theorem}
{
    Niech $f : [a,b] \mapsto \mathbb{R}$ będzie funkcją ograniczoną i majacą tylko skończoną ilość punktów nieciągłości
    na przedziale $[a,b]$ i niech funkcja rosnąca $g$ będzie ciągła w każdym z punktów, w których nieciągła jest funkcja $f$.
    Wtedy $f$ jest (R-S)-całkowalna względem funkcji $g$.
}
\end{theorem}

\begin{proof}
    Niech będzie dane $\varepsilon > 0$. Ponadto niech $M = \sup\limits_{x \in [a,b]}|f(x)|$ oraz niech $E$
    oznacza zbiór punktów nieciągłości $f$. Ponieważ $E$ jest zbiorem skończonym i $g$ jest ciągła
    w każdym z punktów $E$, więc możemy pokryć zbiór $E$ skonczoną liczbą przedziałów rozłącznych $[u_j, v_j] \subset [a,b]$ tak,
    że suma różnic $g(v_j) - g(u_j) < \varepsilon$. Możemy poza tym tak umieścić te przedziały, aby każdy z punktów zbioru $E \cap (a,b)$
    leżał we wnętrzu któregoś z przedziałów $[u_j, v_j]$. Usuńmy przedziały $(u_j, v_j)$ z odcinka $[a,b]$. Pozostały zbiór $K$
    jest skończoną sumą przedziałów domkniętych. Wobec tego na mocy Twierdzenia Cantora funkcja $f$ jest jednostajnie ciągła na $K$,
    a więc istnieje $\delta > 0$ takie, że $|f(s) - f(t)| < \varepsilon$, jeśli tylko $s, t \in K$ oraz $|s-t| < \delta$. \\
    Utwórzmy teraz podział $P = \{x_0, \ldots, x_n\}$ przedziału $[a,b]$ tak, aby każdy z punktów $u_j, v_j$ występował w $P$ oraz aby żaden z punktów przedziału $(u_j, v_j)$
    nie należał do $P$. Dalej jeśli $x_{i-1}$ nie jest żadnym z punktów $u_j$, to ma być $\Delta x_i < \delta$. Zauważmy, że 
    $M_i - m_i \leqslant 2M$ dla dowolnego $i$ oraz, że $M_i - m_i \leqslant \varepsilon$ o ile $x_{i-1}$ nie jest żadnym z punktów $u_j$. Wobec tego mamy
    \[
        U(f, g, P) - L(f, g, P) \sum_{i=1}^{n}(M_i - m_i)\Delta g_i \leqslant [g(b) - g(a)]\varepsilon + 2M\varepsilon.
    \]
    Ponieważ $\varepsilon > 0$ było dowolne, zatem na mocy Twierdzenia \ref{theorem:105} (a) dowód jest skończony.
\end{proof}

\begin{uwaga}
    Jeśli funkcje $f$ i $g$ posiadają w przedziale $[a,b]$ wspólny punkt nieciągłości to może się zdarzyć, że $f$ nie jest (R-S)-całkowalna względem funkcji $g$ (zob. Przykład \ref{ex:33}).
\end{uwaga}

Powtarzając rozumowanie z dowodu Twierdzenia \ref{theorem:81} otrzymujemy następujące.

\begin{theorem}
{
    Niech $f$ będzie funkcją (R-S)-całkowalną względem funkcji rosnącej $g$ na przedziale $[a,b]$, $m \leqslant f(x) \leqslant M$ i niech funkcja
    $\phi$ będzie ciągła na przedziale $[m, M]$. Wówczas funkcja $h(x) : [a,b] \mapsto \mathbb{R}$ określona wzorem $h(x) = \phi(f(x))$ dla $x \in [a,b]$
    jest funkcją (R-S)-całkowalną względem funkcji $g$.
}
\end{theorem}

\begin{theorem}
    Jeśli funkcja $f : [a,b] \mapsto \mathbb{R}$ jest R-całkowalna, a funkcja $g : [a,b] \mapsto \mathbb{R}$ spełnia
    warunek Lipschitza ze stałą $L > 0$, to funkcja $f$ jest (R-S)-całkowalna względem funkcji $g$.
\end{theorem}

\newpage
\begin{proof}
    Załóżmy wpierw dodatkowo, że funkcja $g$ jest rosnąca. Wówczas $\Delta g_i \leqslant \Delta x_i$, skąd
    \[
        U(f, g, P) - L(f, g, P) \leqslant L(U(f, P) - L(f, P)).
    \]
    Ponieważ funkcja $f$ jest R-całkowalna, zatem wobec Twierdzenia \ref{theorem:105} (a) funkcja $f$ jest (R-S)-całkowalna względem funkcji $g$. \\
    Niech teraz $g$ będzie dowolną funkcją spełniającą warunek Lipschitza ze stałą $L > 0$. Przedstawmy funkcję $g$ w postaci
    \[
        g(x) = Lx - (Lx - g(x)) = g_1(x) - g_2(x).
    \]
    Funkcja $g_1(x) = Lx$, $x \in [a,b]$ spełnia oczywiście Warunek Lipschitza, a jednocześnie jest rosnąca. Te same własności ma funkcja $g_2(x) = Lx - g(x)$, $x \in [a,b]$,
    bowiem dla $a \leqslant x < x' \leqslant b$ mamy
    \[
        |g_2(x')-g_2(x)| = L(x'-x) + |g(x') - g(x)| \leqslant 2L(x'-x).
    \]
    Stąd 
    \[
        U(f, g_2, P) - L(f, g_2, P) \leqslant 2L(U(f, P) - L(f, P))
    \]
    i dalej rozumujemy jak wyżej.
\end{proof}

Można również udowodnić (zob. [2], t. III, s. 74---75) następujące

\begin{theorem}
{
    Jeśli funkcja $f : [a,b] \mapsto \mathbb{R}$ jest R-całkowalna, a funkcję $g : [a,b] \mapsto \mathbb{R}$ można przedstawić w następującej postaci:
    \[
        g(x) = c + \int_{a}^{x}\phi(t)dt,
    \]
    gdzie $c$ jest stałą, a $\phi$ --- funkcją bezwględnie całkowalną (w sensie Riemanna lub w sensie niewłaściwym) w przedziale $[a,b]$ (lub $[a,b)$),
    to funkcja $f$ jest (R-S)-całkowalna względem funkcji $g$.
}
\end{theorem}

\begin{theorem}
{
    Niech $f,g,g_1,g_2,f_1,f_2 : [a,b] \mapsto \mathbb{R}$ będą funkcjami ograniczonymi. Wówczas
    \begin{itemize}
        \item [(a)]
            $\int_{a}^{b}dg = g(b) - g(a)$;
        \item [(b)]
            $\int_{a}^{b}[f_1 \pm f_2]dg = \int_{a}^{b}f_1dg \pm \int_{a}^{b}f_2dg$;
        \item [(c)]
            $\int_{a}^{b}fd[g_1 \pm g_2] = \int_{a}^{b}fdg_1 \pm \int_{a}^{b}fdg_2$;
        \item [(d)]
            $\int_{a}^{b}kfd[lg] = kl\int_{a}^{b}fdg$ ($k,l = const.)$;
    \end{itemize}
    (w punktach (b), (c), (d) zakładamy istnienie całek Riemanna-Stieltjesa po prawych stronach nierówności)
    \begin{itemize}
        \item [(e)]
            $\int_{a}^{b}fdg = \int_{a}^{c}fdg + \int_{c}^{b}fdg$, przy założenium że $a < c < b$ oraz, że istnieje całka po lewej stronie
            nierówności (zob. [2], t. III, s.75---76).
    \end{itemize}
}
\end{theorem}

\begin{ex}
    Z istnienia całek $\int_{a}^{c}fdg$ i $\int_{c}^{b}fdg$ nie wynika na ogół istnienie całki $\int_{a}^{b}fdg$.
    Niech w przedziale $[-1, 1]$ funkcje $g$ i $g$ będą określone następująco:
    \[ 
        f(x) = 
        \begin{cases}
            0 \text{ dla} \quad -1 \leqslant x \leqslant, 0 \\
            1 \text{ dla} \quad 0 < x \leqslant 1,
        \end{cases}
    \quad 
        g(x) = 
        \begin{cases}
            0 \text{ dla} \quad -1 \leqslant x < 0, \\
            1 \text{ dla} \quad 0 \leqslant x \leqslant 1,
        \end{cases}
    \]
    Można łatwo sprawdzić, że obie całki $\int_{-1}^{0}fdg$, $\int_{0}^{1}fdg$ istnieją i są równe zeru, bowiem
    odpowiadające im sumy Riemanna-Stieltjesa są równe zeru, bowiem $f(x) = 0$ dla $x \in [-1, 0]$; druga, bowiem dla $x \in [0,1]$ funkcja
    $g$ jest stała, czyli $\Delta g_i = 0$. Natomiast $\int_{-1}^{1}fdg$ nie istnieje. Istotnie, dokonajmy podziału przedziału
    $[-1, 1]$ na podprzedziały tak, żeby punkt $0$ nie był punktem podziału i utwórzmy sumę
    \[
        S = \sum_{i=1}^{n}f(\xi_i)\Delta g_i.
    \]
    Jeśli punkt $0$ należy do przedziału $[x_{k-1}, x_k]$, to przy $x_{k-1} < 0 < x_k$ w sumie $S$ jest
    różny tylko $k$-ty składnik bowiem $\Delta g_i = 0$ dla $i \neq k$. W takim razie $S = f(\xi_k)(g(x_k) - g(x_{k-1})) = f(\xi_k)$.
    W zależności od tego czy $\xi_k \leqslant 0$ czy $\xi_k > 0$ mamy $S = 0$ lub $S = 1$, 
    zatem $\int_{-1}^{1}fdg$ nie istnieje.
\end{ex}

\begin{theorem}
{
    Niech $g : [a,b] \mapsto \mathbb{R}$ będzie funkcją rosnącą. Wówczas
    \begin{itemize}
        \item [(a)]
            Jeżeli funkcje $f_1, f_2 : [a,b] \mapsto \mathbb{R}$ są (R-S)-całkowalne względem funckji $g$ oraz $f_1(x) < f_2(x)$ dla każdego $x \in [a,b]$, to
            \[
                \int_{a}^{b}f_1dg \leqslant \int_{a}^{b}f_2dg.
            \]
        \item [(b)]
            Jeśli $f : [a,b] \mapsto \mathbb{R}$ jest funkcją (R-S)-całkowalną względem funkcji $g$, to $|f|$ jest funkcją (R-S)-całkowalna względem funkcji $g$ oraz
            \[
                \Bigg|\int_{a}^{b}fdg\Bigg| \leqslant \int_{a}^{b}|f|dg.
            \]
            Ponadto jeśli $|f(x)| \leqslant M$ na $[a,b]$, to $\Big|\int_{a}^{b}fdg\Big| \leqslant M(g(b) - g(a))$.
        \item [(c)]
            Jeśli $f,h : [a,b] \mapsto \mathbb{R}$ są funkcjami (R-S)-całkowalnymi względem $g$, to iloczyn $fh$ jest funkcją (R-S)-całkowalną względem funkcji $g$.
    \end{itemize}
}
\end{theorem}

Dowód powyższego twierdzenia jest analogiczny do dowodów twierdzeń \ref{theorem:82} (dla iloczynu funkcji) oraz \ref{theorem:83}. \\
Udowodnimy teraz twierdzenia o zamianie zmiennych oraz o całkowaniu przez części dla całek Riemanna-Stieltjesa.

\begin{theorem}
{
    Niech $\phi$ będzie funkcją ściśle rosnącą odwzorowującą przedział $[\alpha, \beta]$ na przedział $[a,b]$.
    Niech $g$ będzie również funkcją rosnącą na $[a,b]$ i niech $f$ będzie funkcją (R-S)-całkowalną względem funkcji $g$ na $[a,b]$.
    Określmy na przedziale $[\alpha, \beta]$ funckje $F$ i $G$ wzorami
    \[
        G(y) = g(\phi(y)), \quad F(y) = f(\phi(y)),
    \]
    Wówczas $F$ jest funkcją (R-S)-całkowalną względem funkcji $G$ na $[\alpha, \beta]$ oraz
    \[
        \int_{\alpha}^{\beta}FdG = \int_{a}^{b}fdg.
    \]
}
\end{theorem}

\begin{proof}
    Każdemu podziałowi $P = \{x_0, \ldots, x_n \}$ przedziału $[a,b]$ odpowiada podział 
    $Q = \{y_0, \ldots, y_n \}$ przedziału $[\alpha, \beta]$ taki, że $y_i = \phi(x_i)$, $i = 0, \ldots, n$.
    Dowolny podział przedziału $[\alpha, \beta]$ możemy otrzymać w ten sposób. Ponieważ wartości przyjmowane przez $f$ na przedziale $[x_{i-1}, x_i]$ są dokładnie takie same,
    jak wartości przyjmowane przez $F$ na $[y_{i-1}, y_i]$, zatem
    \begin{equation}\label{eq:61}
        U(f, g, P) = U(F, G, Q), \quad L(f, G, P) = L(F, G, Q).
    \end{equation}
    Ponieważ funkcja $f$ jest (R-S)-całkowalna względem $g$, więc możemy tak wybrać $P$, aby
    zarówno $U(f, g, P)$ i $L(f, g, P)$ były bliskie $\int_{a}^{b}fdg$. Wówczas (\ref{eq:61}) w
    połączeniu z Twierdzeniem \ref{theorem:105} (a) pokazuje, że funkcja $F$ jest (R-S)-całkowalna
    względem funkcji $G$ i że zachodzi wzór z tezy twierdzenia.
\end{proof}

\begin{theorem}
{
    Niech $f, g : [a,b] \mapsto \mathbb{R}$ będą funkcjami ograniczonymi. Wówczas 
    \begin{equation}\label{eq:62}
        \int_{a}^{b}fdg = (fg)(x) \big|_a^b - \int_{a}^{b}gdf,
    \end{equation}
    przy założeniu, że przynajmniej jedna z tych całek istnieje.
}
\end{theorem}

\begin{proof}
    Niech istnieje całka $\int_{a}^{b}gdf$ i niech $P = \{x_0, \ldots, x_n\}$ będzie podziałem przedziału $[a,b]$.
    Sumę Riemanna-Stieltjesa odpowiadającą podziałowi $P$ przy ustalonym wyborze punktów $\xi_i$ możemy przedstawić w postaci
    \begin{align*}
        &S = \sum_{i=1}^{n}f(x_i)(g(x_i) - g(x_{i-1})) = f(b)g(b) - f(a)g(a) - \\
        &\Bigg\{ g(a)(f(\xi_1) - f(a)) + \sum_{i=2}^{n}g(x_{i-1})[f(\xi_i) - f(\xi_{i-1})] + g(b)(f(b) - f(\xi_n))   \Bigg\}.
    \end{align*}
    Wyrażenie w nawiasach sześciennych przedstawia pewną sumę Riemanna-Stieltjesa dla całki
    $\int_{a}^{b}gdf$, której istnienie zakładamy. Odpowiada ona podziałowi $\tilde{P} = \{a, \xi_1, \ldots, \xi_n, b\}$
    przedziału $[a,b]$. Ponadto $\delta(\tilde{P}) \leqslant 2\delta(P)$. Jeśli $\delta(P) \to 0$ to $\delta(\tilde{P}) \to 0$, a zatem
    istnieje również granica dla sum $S$, to jest $\int_{a}^{b}fdg$ i całka ta dana jest wzorem (\ref{eq:62}).
\end{proof}

\begin{defn}
    Jednostkową funkcją schodkową naywamy funkcję
    \[
        f(x) = 
        \begin{cases}
            0 \text{ dla} \quad x \leqslant 0, \\
            1 \text{ dla} \quad x > 0.
        \end{cases}
    \]
\end{defn}

\begin{theorem}
{
    Jeśli $a \leqslant x \leqslant b$, $f$ jest funkcją ograniczoną na $[a,b]$ oraz ciągłą w punkcie $s$, a $g(x) = I(x - s)$, to
    \[
        \int_{a}^{b}fdg = f(s).
    \]
}
\end{theorem}

\begin{proof}
    Niech $P = \{x_0,\ldots, x_n \}$ będzie podziałem przedziału $[a,b]$. Niech $s$ należy na przykład do $k$-tego podprzedziału;
    jest więc $x_{k-1} \leqslant s \leqslant x_k$. Wówczas $\Delta I_k = 1$, a przy $i \neq k$ mamy $\Delta I_i = 0$. Suma Riemanna-Stieltjesa $S$
    sprowadza się więc do jednego składnika: $S = f(\xi_k)$. Niech teraz $\delta(P) \to 0$. Na mocy ciągłości funkcji $f$ 
    w punkcie $s$, $f(\xi_k) \to f(s)$, czyli 
    \[
        \int_{a}^{b}fdg = f(s).
    \]
\end{proof}

\begin{ex}
    \[
        \int_{1}^{\pi}\ln x d[x] = (\ln 2) \cdot 1 + (\ln 3) \cdot 1 = \ln 6.
    \]
\end{ex}

\begin{theorem}
{
    Jeśli $f : [a.b] \mapsto \mathbb{R}$ jest funkcja R-całkowalną oraz pochodna funkcji $g : [a,b] \mapsto \mathbb{R}$ jest całkowalna w sensie Riemanna, to
    \begin{equation}\label{eq:63}
        \int_{a}^{b}fdg = \int_{a}^{b}fg'dx.
    \end{equation}
}
\end{theorem}

\begin{proof}
    Na mocy twierdzeń \ref{theorem:82} i \ref{theorem:108} obydwie całki występujące we wzorze (\ref{eq:63}) istnieją.
    Niech teraz $ P = \{x_0, \ldots, x_n\}$ będzie podziałem przedziału $[a,b]$. Na mocy Twierdzenia
    Lagrange'a o Wartośći Średniej sumę Riemanna-Stieltjesa odpowiadającą podziałowi $P$ możemy zapisać w postaci
    \[
        S = \sum_{i=1}^{n}f(\xi_i)(g(x_i) - g(x_{i-1})) = \sum_{i=1}^{n}f(\xi_i)g'(\mu_i)(x_i - x_{i-1}),
    \]
    gdzie $x_{i-1} < \mu_i < x_i$, $i = 1, \ldots, n$. Podstawiając w powyższej równości $\xi_i = \mu_i$ oraz
    zakładając, że $\delta(P) \to 0$ otrzymujemy wzór (\ref{eq:63}).
\end{proof}

\begin{wniosek}
{
    Niech $g(x) = x$ dla $x \in [a,b]$ i niech $\phi$ będzie ściśle rosnącą funkcją odwzorowującą przedział
    $[\alpha, \beta]$ na przedział $[a,b]$ i taką, że funkcja $\phi'$ jest funkcją R-całkowalną na $[\alpha, \beta]$.
    Wobec twierdzeń \ref{theorem:112} i \ref{theorem:115} otrzymujemy 
    \[
        \int_{a}^{b}f(x)dx = \int_{\alpha}^{\beta}f(\phi(y))\phi'(y)dy;
    \]
    powyższy wzór jest treścią twierdzenia \ref{theorem:91}, jeśli $\phi$ jest funkcją rosnącą. Dobierając $\phi(y) = \phi(-y + \alpha + \beta)$
    możemy uzyskać tezę twierdzenia dla funkcji malejącej.
}
\end{wniosek}

Podamy jeszcze trzy twierdzenia dotyczące całek Riemanna-Stieltjesa. Dowody tych twierdzeń można znaleźć w [2], t. III, s. 79---82.

\begin{theorem}
{
    Jeśli $f : [a,b] \mapsto \mathbb{R}$ jest R-całkowalna, a funkcję $g : [a,b] \mapsto \mathbb{R}$ można przedstawić w postaci 
    \[
        g(x) = c + \int_{a}^{x}\phi(t)dt, \quad c=const, \quad x \in [a,b],
    \]
    gdzie funkcja $\phi$ jest bezwględnie całkowalna w $[a,b]$, to 
    \[
        \int_{a}^{b}fdg = \int_{a}^{b}f\phi dx
    \]
}
\end{theorem}

\begin{theorem}
{
    Jeśli $f : [a,b] \mapsto \mathbb{R}$ jest R-całkowalna, a funkcja $g : [a,b] \mapsto \mathbb{R}$ jest ciągła w całym przedziale
    $[a,b]$ i ma w nim, poza conajmniej skończoną liczbą punktów pochodną $g'$ bewględnie całkowalną w $[a,b]$, to
    \[
        \int_{a}^{b}fdg = \int_{a}^{b}fg'dx.
    \]
}
\end{theorem}

\begin{theorem}
{
    Niech $f : [a,b] \mapsto \mathbb{R}$ będzie funkcją ciągła,a funkcja $g : [a,b] \mapsto \mathbb{R}$
    niech ma w przedziale $[a,b]$ z pominięciem co najwyżej skończonej liczby punktów pochodną $g'$, bezwzględnie całkowalną w tym przedziale.
    Niech ponadto funkcja $g$ ma w skończonej liczbie punktów
    \[
        a = c_0 < c_1 < \cdots < c_n = b
    \]
    nieciągłość pierwszego rodzaju. Wówczas istnieje całka Riemanna-Stieltjesa funkcji $f$ względem funkcji $g$ i wyraża się wzorem
    \begin{align*}
        \int_{a}^{b}fdg &= \int_{a}^{b}fg'dx + f(a)[g(a + 0) -g(a)] \\
                        &+ \sum_{k=1}^{n-1}f(c_k)(g(c_k + 0) - g(c_k - 0)) + f(b)[g(b) - g(b - 0)].
    \end{align*}
}
\end{theorem}

\subsection{Całka krzywoliniowa skierowana}

\begin{defn}
    Niech będzie dana krzywa $\gamma : [a,b] \mapsto \mathbb{C}$ ($\gamma(t) = (\phi(t), \psi(t)))$, $t \in [a,b]$ i niech
    będą dane wzdłuż obrazu tej krzywej pewne funkcje $f(x,y)$, $g(x,y)$ ($f, g : \gamma([a,b]) \mapsto \mathbb{R}$). Niech
    $P = \{t_0, \ldots, t_n\}$ będzie podziałem przedziału $[a,b]$. Tworzymy sumy
    \[
        \sigma = \sum_{i=1}^{n}[f(\phi(\xi_i),\psi(\xi_i))(\phi(t_i) - \phi(t_{i-1})) + g(\phi(\xi_i), \psi(\xi_i))(\psi(t_i) - \psi(t_{i-1}))],
    \]
    gdzie $\xi_i \in [t_{i-1}, t_i]$, $i = 1, \ldots, n$. \\
    Jeśli dla dowolnego ciągu normalnego podziałów $(P_k)$ i dla dowolnych sum $\sigma_k$ odpowiadających podziałowi $P_k$ istnieje 
    skończona granica $I = \lim\limits_{k \to \infty}\sigma_k$, to tę granicę nazywamy całką krzywoliniową skierowaną $fdx + gdy$ po drodze 
    $\gamma$ i oznaczamy ją symbolem
    \[
        \int_{\gamma}fdx + gdy.
    \]
\end{defn}

Bezpośrednio z definicji wynika następujące

\begin{theorem}
{
    \begin{itemize}
        \item [(a)]
            Niech $-\gamma(t) = (\phi(-t), \psi(-t))$, $t \in [-b,-a]$. Wówczas, jeśli istnieje całka krzywoliniowa skierowana dla 
            $fdx + gdy$ po krzywej $\gamma$, to istnieje też całka po krzywej $-\gamma$ i oraz
            \[
                \int_{-\gamma}fdx + gdy = -\int_{\gamma}fdx + gdy.
            \]
        \item [(b)]
            Niech $\gamma = \gamma_1 + \gamma_2$ (to znaczy istnieje punkt $c \in [a,b]$ taki, że $\gamma(t)=\gamma_1(t)$ dla $t \in [a,c]$ oraz $\gamma_2(t)=\gamma(t)$ dla $t \in [c,b]$). Wówczas całka dla $fdx + gdy$ po krzywej $\gamma$ istnieje wtedy i tylko wtedy, gdy istnieją całki dla $fdx + gdy$ po krzywych $\gamma_1$, $\gamma_2$ przy czym
            \[
                \int_{\gamma}fdx + gdy = \int_{\gamma_1}fdx+gdy + \int_{\gamma_2}fdx+gdy
            \]
    \end{itemize}
}
\end{theorem}

Udowodnimy teraz twierdzenie o zamianie całki krzywoliniowej skierowanej na całkę Riemanna.

\begin{theorem}
{
    Jeżeli funkcje $f, g$ są ciągłe na obrazie krzywej gładkiej $\gamma : [a,b] \mapsto \mathbb{C}$ to $\int_{\gamma}fdx + gdy$ istnieje oraz zachodzi wzór
    \begin{equation}\label{eq:64}
        \int_{\gamma}fdx + gdy = \int_{a}^{b}[f(\phi(t), \psi(t))\phi'(t) + g(\phi(t), \psi(t))\psi'(t)]dt.
    \end{equation}
}
\end{theorem}

\begin{proof}
    Sumę $\sigma$ występującą w Definicji \ref{def:89} możemy zapisać w postaci
    \[
        \sigma = \sum_{i=1}^{n}[f(\phi(\xi_i), \psi(\xi_i))\phi'(\nu_i) + g(\phi(\xi_i), \psi(\xi_i))\psi'(\mu_i)](t_i - t_{i-1}),
    \]
    gdzie $\nu_i, \mu_i \in [t_{i-1}, t_i]$, bowiem do różnic $\phi(t_i) - \phi(t_{i-1})$ oraz  $\psi(t_i) - \psi(t_{i-1})$ można zastosować Twierdzenie Lagrange'a o Wartości Średniej. Całka po prawej stronie równości (\ref{eq:64}) istnieje, bowiem funkcja podcałkowa jest ciągła oraz jest granicą ciagu sum postaci
    \[
        \tilde{\sigma} = \sum_{i=1}^{n}[f(\phi(\xi_i), \psi(\xi_i))\phi'(\xi_i) + g(\phi(\xi_i), \psi(\xi_i))\psi'(\xi_i)](t_i - t_{i-1})
    \]
    (dla dowolnego ciągu normalnego podziałów ($P_k)$). Wystarczy zatem wykazać, że sumy $\sigma$ i $\tilde{\sigma}$ różnią się o dowolnie mało, jeśli średnica podziału jest dostatecznie mała, lub, że różnice $|\phi'(\nu_i) - \phi'(\xi_i)|$, $|\psi'(\mu_i) - \psi'(\xi_i)|$ są dowolne małe. To zaś wynika z jednostajnej ciągłości pochodnych $\phi'$, $\psi'$ w przedziale $[a,b]$. 
\end{proof}

\begin{ex}
    Korzystając ze wzoru \ref{eq:64} obliczmy wartość całki krzywoliniowej $H = \int_{L}2xydx + x^2dy$ wziętej po drodze $L$ łączącej punkty $O(0,0)$ i $A(1,1)$ jeśli droga $L$ jest: 
    \begin{itemize}
        \item [(a)]
            prostą $y = x$ ($\gamma(x) = (x,x)$, $x \in [0,1]$)
            \[
                \int_{L}2xydx + x^2dy = \int_{0}^{1}3x^2dx = 1,
            \]
        \item [(b)]
            parabolą $y = x^2$ ($\sigma(x)=(x, x^2)$, $x \in [0,1]$)
            \[
                \int_{L}2xydx + x^2dy = \int_{0}^{1}4x^3dx = 1.
            \]
    \end{itemize}
\end{ex}

% END HERE 

\newpage

\noindent
\textbf{Twierdzenie 131.}\label{theorem:131}
\textit
{
    (Całkowe Kryterium Zbieżności Szeregów). Jeśli $f$ jest funkcją dodatnią i nierosnącą na przedziale $[1, +\infty)$ oraz
    $f(n) = a_n$ dla $n \in \mathbb{N}$, to:
    \begin{itemize}
        \item [(a)]
            szereg $\sum_{n=1}^{\infty}a_n$ jest zbieżny wtedy i tylko wtedy, gdy zbieżna jest całka niewłaściwa $I = \int_{1}^{+\infty}f(x)dx$;
        \item [(b)]
            ciąg $(s_n - I_n)$, gdzie $s_n = \sum_{k=1}^{n}a_k$ i $I_n = \int_{1}^{n}f(x)dx$ jest zbieżny i jego granica należy
            do przedziału $[0, a_1]$. 
    \end{itemize}
}

\begin{proof}
    Dla $k \leqslant x \leqslant k + 1$ mamy $a_{k+1} \leqslant f(x) \leqslant a_k$, a więc $a_{k+1} \leqslant \int_{k}^{k+1}f(x)dx \leqslant a_k$, skąd 
    \[
        a_2 + \cdots + a_n \leqslant \int_{1}^{n}f(x)dx \leqslant a_1 + \cdots + a_{n-1},
    \]
    czyli 
    \[
        s_n - a_1 \leqslant I_n \leqslant s_{n-1}.
    \]
    Jeśli całka $I$ istnieje, to ciąg $(I_n)$ jest zbieżny, więc i ograniczony. Z powyższej nierówności wynika, że ograniczony jest również
    ciąg $(s_n)$. Stąd na mocy Twierdzenia 127 wnioskujemy, że szereg $\sum_{n=1}^{\infty}a_n$ jest zbieżny. \\
    Załóżmy teraz, że całka $I$ jest rozbieżna. Ponieważ ciąg $(I_n)$ jest niemalejący, więc
    $\lim\limits_{n \to \infty}I_n = +\infty$. Wówczas również $\lim\limits_{n \to \infty}s_n = +\infty$, czyli 
    $\sum_{n=1}^{\infty}a_n$ jest rozbieżny. \\ 
    Ciąg $(s_n - I_n)$ jest nierosnący, bowiem
    \[
        (s_{n-1} - I_{n-1}) - (s_n - I_n) = (I_n - I_{n-1}) - (s_n - s_{n-1}) = \int_{n-1}^{n}f(x)dx - a_n \geqslant 0.
    \]
    Ponadto jest on ograniczony, bowiem
    \[
        0 < a_n \leqslant s_n - I_n \leqslant a_1.
    \]
    Wobec tego jest on ciągiem zbieżnym $\lim\limits_{n \to \infty}(s_n - I_n) \in [0, a_1]$. 

\end{proof}

\newpage

\setcounter{section}{9}
\setcounter{defcounter}{97}
\setcounter{equation}{72}
\setcounter{uwagacounter}{56}
\setcounter{excounter}{40}
\setcounter{thcounter}{143}
\setcounter{wniosekcounter}{29}

% Ciągi i szeregi funkcyjne

\section{Ciągi i szeregi funkcyjne}
\subsection{Zbieżność punktowa i jednostajna}

\begin{defn}
    Niech $(f_n)$, $n \in \mathbb{N}$, będzie ciągiem funkcji o wartościach rzeczywistych
    lub zespolonych, określonych na zbiorze $E \subset \mathbb{R}$ (lub $E \subset \mathbb
    {C}$). Jeśli ciąg liczb $(f_n(x))$ jest zbieżny dla każdego $x \in E$, to mówimy że
    ciąg $(f_n)$ jest zbieżny lub zbieżny punktowo na zbiorze $E$ do funkcji $f$, gdzie
    \[
        f(x) = \lim\limits_{n \to \infty}f_n(x) \quad (x \in E).
    \]

    Analogicznie, jeśli ciąg $(s_n)$ sum częsciowych szeregu 
    $\sum\limits_{n=1}^{\infty}f_n$ ($s_n = \sum\limits_{k=1}^{n}f_k$, $k \in \mathbb{N}$)
    jest punktowo zbieżny na zbiorze $E$ do funkcji 
    $s$ ($s(x) = \lim\limits_{n \to \infty} s_n(x)$, $x \in E$), to mówimy, że szereg
    $\sum\limits_{n=1}^{\infty}f_n$ jest punktowo zbieżny na zbiorze $E$ do funkcji $s$.
    Funkcję $s$ nazywamy wówczas sumą szeregu $\sum\limits_{n=1}^{\infty}f_n$ i piszemy
    \[
        s(x) = \sum\limits_{n=1}^{\infty}f_n(x) \quad (x \in E).
    \]
\end{defn}

\begin{defn}
    Mówimy, że ciąg funkcji $(f_n)$, $n \in \mathbb{N}$, jest zbieżny jednostajnie na
    zbiorze $E$ do funkcji $f$, jeżeli dla dowolnego $\varepsilon > 0$ istnieje takie $N
    \in \mathbb{N}$, że dla $n \geqslant N$ zachodzi
    \begin{equation}\label{eq:73}
        |f_n(x) - f(x)| < \varepsilon, \quad \text{dla każdego } x \in E.
    \end{equation}

    Analogicznie, jeśli ciąg $(s_n)$ sum częsciowych szeregu $\sum\limits_{n=1}^{\infty}$
    jest zbieżny jednostajnie na zbiorze $E$ do funkcji $s$ to mówimy, że szereg
    $\sum\limits_{n=1}^{\infty}$ jest zbieżny jednostajnie na zbiorze $E$ do funkcji $s$. 
\end{defn}

\begin{uwaga}
    Jest oczywiste, że każdy ciąg zbieżny jednostajnie jest także zbieżny punktowo;
    odwortnie zachodzić nie musi (zob. Przykład \ref{ex:45} (b)). \\
    Róznica między zbieżnością punktową, a jednostajną polega na tym, że w pierwszym
    przypadku dla dowolnego $\varepsilon > 0$ i dowolnego ustalonego $x \in E$ można dobrać
    takie $N$ (które zależy i od $\varepsilon$ i od $x$), że dla $n \geqslant N$ będzie
    spełniona nierówność (\ref{eq:73}); jeżeli ciąg $(f_n)$ jest zbieżny jednostajnie, to
    przy każdym $\varepsilon > 0$ można dobrać jedną wspólną dla wszystkich punktów $x \in
    E$ liczbę $N$.
\end{uwaga}

\newpage 

\begin{ex}
    \begin{itemize}
        \item [(a)] 
            Niech $f_n(x) = \frac{x}{1+n^2x^2}$, $x \in \mathbb{R}$. \\
            Wówczas $\lim\limits_{n \to \infty}f_n(x) = 0$ dla każdego $x \in \mathbb{R}$. \\
            Dalej, ponieważ
            \[
                {(1-nx)}^2 \geqslant 0 \quad \text{oraz} \quad {(1 + nx)}^2 \geqslant 0
            \]
            zatem
            \[
                \frac{2nx}{1+n^2x^2}\leqslant 1 \quad \text{oraz} \quad \frac{2nx}{1+n^2x^2} \geqslant -1.
            \]
            Stąd
            \[
                -\frac{1}{2n} \leqslant f_n(x) = \frac{1}{2n}\frac{2nx}{1+n^2x^2} \leqslant \frac{1}{2n},
            \]
            zatem, aby nierówność $|f_n(x)| \leqslant \varepsilon$ była spełniona dla
            wszystkich $x$ wystarczy przyjąć $n > \frac{1}{2\varepsilon}$. Tak, więc liczba
            $N = [\frac{1}{2\varepsilon}] + 1$ nadaje się dla wszystkich $x$ jednocześnie.
        \item [(b)]
            Niech $f_n(x) = n^2{(1-x^2)}^n$, $x \in [0,1]$, $n \in \mathbb{N}$. \\
            Jeśli $0 < x \leqslant 1$, to $\lim\limits_{n \to \infty} f_n(x) = 0$, na mocy Twierdzenia 34 (c). \\
            Ponieważ $f_n(0) = 0$ dla dowolnego $n \in \mathbb{N}$, zatem $\lim\limits_{n \to \infty}f_n(x) = 0$ dla $x \in [0,1]$. \\
            Ponieważ $f_n(\frac{1}{n})=n^2\frac{1}{n}{(1-(\frac{1}{n})^2)}^n = n\frac{{(n+1)}^n}{n^n}\frac{{(n-1)}^n}{n^n}$, a więc nie możemy osiągnąc tego, by
            $|f_n(x)| < 1$ dla każdego $x \in [0,1]$ i prawie wszystkich $n$. 
        \end{itemize}
\end{ex}

Następujące twierdzenie jest przeniesiem kryterium Cauchy'ego na przypadek zbieżnośći jednostajnej.

\begin{theorem}
{
    Ciąg funkcji $(f_n)$ określonych na zbiorze $E$, jest na tym zbiorze jednostajnie zbieżny wtedy i tylko wtedy, gdy
    dla dowolnego $\varepsilon > 0$ istnieje liczba naturalna $N$ taka, że przy $m, n \geqslant N$, $x \in E$ mamy
    \begin{equation}\label{eq:74}
        |f_n(x) - f_m(x)| < \varepsilon.
    \end{equation}
}
\end{theorem}

\begin{proof}
    Niech ciąg $(f_n)$ będzie zbieżny jednostajnie na zbiorze $E$ do funkcji $f$. Istnieje wtedy liczba
    naturalna $N$, taka, że jeżeli $n \geqslant N$ i $x \in E$, to zachodzi nierówność
    \[
        |f_n(x) - f(x)| < \frac{\varepsilon}{2}.
    \]
    Wtedy
    \[
        |f_n(x) - f_m(x)| \leqslant |f_n(x) - f(x)| + |f(x) - f_m(x)| < \varepsilon
    \]
    jeżeli $n,m \geqslant N$, $x \in E$. \\
    Na odwrót, niech będzie spełniony warunek Cauchy'ego. Na mocy Twierdzenia 33 ciąg $(f_n(x))$ jest przy
    każdym $x$ zbieżny do granicy, którą oznaczmy przez $f(x)$. Pokażemy, że w tym przypadku zbieżność jest jednostajna. \\
    Niech będzie dana liczba $\varepsilon > 0$. Niech $N \in \mathbb{N}$ będzie takie, aby spełniona
    była nierówność (\ref{eq:74}). Ustalmy $n$ i przejdźmy w nierówności (\ref{eq:74}) z $m$ do granicy ($m \to \infty)$.
    Ponieważ $f_m(x) \to f(x)$ przy $m \to \infty$, zatem
    \[
        |f_n(x) - f(x)| < \varepsilon
    \]
    dla dowolnego $n > N$ i dowolnego $x \in E$. Dowód jest zakończony.
\end{proof}

Następujące kryterium jest bezpośrednią konsekwencją Definicji \ref{def:99}.

\begin{theorem}
{
    Niech $\lim\limits_{n \to \infty}f_n(x) = f(x)$ dla $x \in E$. Określmy $M_n = \sup\limits_{x \in E}|f_n(x) - f(x)|$.
    Wówczas $f_n \to f(x)$ jednostajnie na $E$ wtedy i tylko wtedy, gdy $M_n \to 0$ przy $n \to \infty$.
}
\end{theorem}

Podamy teraz wygodnie kryterium zbieżności jednostajnej szeregów funkcyjnych pochodzące od Weierstrassa.

\begin{theorem}
{
    (Twierdzenie o Majorancie). Niech $(f_n)$ będzie ciągiem funkcyjnym określonym na zbiorze $E$ i niech
    \[
        |f_n(x)| \leqslant M \quad (x \in E, n \in \mathbb{N}).
    \]
    Wtedy szereg $\sum\limits_{n=1}^{\infty}f_n$ jest zbieżny jednostajnie na $E$, jeśli zbieżny jest 
    szereg liczbowy $\sum\limits_{n=1}^{\infty}M_n$.
}
\end{theorem}

\begin{proof}
    Jeżeli $\sum\limits_{n=1}^{\infty}M_n$ jest zbieżny to przy dowolnym $\varepsilon > 0$
    \[
        \Bigg|\sum\limits_{i=n}^{m}f_i(x)\Bigg| \leqslant \sum\limits_{i=n}^{m}M_i < \varepsilon \quad (x \in E),
    \]
    jeżeli tylko $m$ i $n$ są dostatecznie duże (zob. Tw. 121). Wobec Twierdzenia \ref{theorem:144} wnosimy, 
    więc, że szereg funkcyjny $\sum\limits_{n=1}^{\infty}f_n$ jest zbieżny jednostajnie.
\end{proof}

\begin{uwaga}
    Jest jasne, że jeżeli do danego szeregu funkcyjnego można zastosować Kryterium Weierstrassa, to szereg ten
    jest zbieżny bezwględnie. Możliwe są jednak przypadki gdy dany szereg jest zbieżny jednostajnie, nie będąc zbieżnym bezwględnie.
    (np. $\sum\limits_{n=1}^{\infty}\frac{(-1)^{n-1}}{x^2+n}$ ([2], t. II, przykład 7/367)). Możliwe są również przypadki gdy szereg
    funkcyjny $\sum\limits_{n=1}^{\infty}f_n$ jest zbieżny bezwględnie i jednostajnie, a szereg $\sum\limits_{n=1}^{\infty}|f_n|$ jest zbieżny
    niejednostajnie (np. $\sum\limits_{n=1}^{\infty}\frac{x^2(-1)^{n-1}}{{(1+x^2)}^n}$ ([2], t. II. przykład. 8 s. 367)).
\end{uwaga}

Rozumiejąc analogicznie jak w dowodzie Twierdzenia 134 otrzymujemy nastepujące kryterium zbieżności jednostajnej szeregów funkcyjnych zwane
\textbf{Kryterium Dirichleta}.

\begin{theorem}
{
    Niech sumy częsciowe $s_n$ szeregu funkcyjnego $\sum\limits_{n=1}^{\infty}f_n$ będą wspólnie ograniczone dla dowolnych $x$ i $n$: 
    $|s_n(x)| \leqslant M$, a funkcje $g_n(x)$ (dla każdego $x$) tworzą ciąg monotoniczny zbieżny do $0$ jednostajnie na zbiorze $E$.
    Wówczas szereg funkcyjny $\sum\limits_{n=1}^{\infty}f_n g_n$ jest także zbieżny jednostajnie w tym zbiorze.
}
\end{theorem}

\begin{theorem}
{
    (Kryterium Abela). Niech szereg funkcyjny $\sum\limits_{n=1}^{\infty}f_n$
    będzie zbieżny jednostajnie na zbiorze $E$ i niech funkcje $g_n(x)$ tworzą (dla każdego $x$) ciąg monotoniczny
    i są wspólnie ograniczone dla dowolnych $x$ i $n$: $|g_n(x)| \leqslant K$. Wówczas szereg
    funkcyjny $\sum\limits_{n=1}^{\infty} f_n g_n$ jest zbieżny jednostajnie w zbiorze $E$.
}
\end{theorem}

\begin{proof}
    Z uwagi na jednostajną zbieżność szeregu funkcyjnego $\sum\limits_{n=1}^{\infty}f_n$ znajdziemy wskaźnik $N$, niezależny od $x$ taki, że dla $N \leqslant p \leqslant q$ mamy
    \[
        \Bigg|\sum_{n=p}^{q}f_n(x)\Bigg| < \frac{\varepsilon}{3^k}.
    \]
    Wówczas dla $N \leqslant p \leqslant q$, stosując przekstałcenie Abela (Lemat \ref{lemat:3}, w miejsce $A_n$ wstawiamy $s_{p,n} = f_p + f_{p+1} + \cdots + f_n$; $s_{p,{p-1}} = 0$) mamy
    \begin{align*}
        \Bigg|\sum_{n=p}^{\infty}f_n(x)g_n(x)\Bigg| &= \Bigg|\sum_{n=p}^{q-1}s_{p,n}(x)(g_n(x) - g_{n+1}(x)) + s_{p, q}(x)g_q(x) - s_{p, p-1}(x)g_p(x) \Bigg| \\
                                                    &< |g_p(x)|\frac{\varepsilon}{3^k} + \frac{\varepsilon}{3^k}(|g_p(x) + g_q(x)|) \leqslant \varepsilon
    \end{align*}
    dla każdego $x \in E$. Wobec Twierdzenia \ref{theorem:144} dowód jest zakończony.
\end{proof}

\noindent
\textbf{Lemat 3.}\label{lemat:3}
Niech $a_m, a_{m+1}, \ldots, a_n$ i $b_m, b_{m+1}, \ldots, b_n$ będą liczbami rzeczywistymi, $m \leqslant n$. Oznaczmy
\[
    s_{m,k} = b_m + b_{m+1} + \cdots + b_k \quad \text{ dla } \quad k = m, m+1, \ldots, n, \quad s_{m,{m-1}} = 0. 
\]
Wówczas
\[
    \sum_{k=m}^{n}a_k b_k = a_n s_{m,n} + \sum_{k=m}^{n-1}(a_k - a_{k+1})s_{m, k}
\]
(zob. [6], s.65-66).

\newpage
\subsection{Zbieżność jednostajna a ciągłość, różniczkowalność i całkowalność}

Powstaje naturalne pytanie, czy przy założeniu, że funkcje $f_n$ określone na zbiorze $E$ są
ciągłe lub różniczkowalne, lub całkowalne w sensie Riemanna, to analogiczną własność będzie miała funkcja graniczna?.
Jaki związek zachodzi na przykład pomiędzy $f_n' i f'$ lub pomiędzy całkami Riemanna z $f_n$ a całką Riemanna z $f$? \\
Pytanie czy granica ciągu funkcyjnego jest funkcją ciągłą jest pytaniem czy zachodzi równość 
\[
    \lim\limits_{t \to x} \lim\limits_{n \to \infty}f_n(t) = \lim\limits_{n \to \infty}\lim\limits_{t \to x}f_n(t) = \lim\limits_{n \to \infty}f_n(x) 
    \quad \text{(z ciągłośći } f_n; \quad (t, x \in E) \text{)},
\] 
tj. czy istotna jest kolejnośc, w jakiej dokunuje się przejść granicznych.

\begin{ex}
    Niech $f_n(x) = \frac{x^2}{{(1+x^2)}^n}$, $x \in \mathbb{R}$, $n \in \mathbb{N}$ i niech
    $f(x) = \sum\limits_{n=0}^{\infty}f_n(x) = \sum\limits_{n=0}^{\infty}\frac{x^2}{{(1+x^2)}^n}$.
    Ponieważ $f_n(0) = 0$, więc i $f(0) = 0$. Suma rozważanego szeregu jako szeregu geometrycznego jest równa $1 + x^2$ dla $x \neq 0$. \\
    Zatem $f(x) = \begin{cases} 0 & \text{dla } x = 0 \\ 1 + x^2 & \text{dla } x \neq 0 \end{cases}$, a
    więc szereg, którego wyrazami są funkcję ciągłe, może mieć sumę nieciągłą.
\end{ex}

\begin{theorem}
{
    Niech $f_n \to f$ jednostajnie na zbiorze $E$ (zawartym w $\mathbb{R}$ lub $\mathbb{C}$).
    Niech $x$ będzie punktem skupienia zbioru $E$ i niech 
    \[
        \lim\limits_{t \to x}f_n(t) = A_n \quad (n \in \mathbb{N}).
    \]
    Wówczas ciąg $(A_n)$ jest zbieżny i $\lim\limits_{t \to x}f(t) = \lim\limits_{n \to \infty}A_n$ \\ 
    (tj. $\lim\limits_{t \to x} \lim\limits_{n \to \infty}f_n(t) = \lim\limits_{n \to \infty}\lim\limits_{t \to x}f_n(t)$).
}
\end{theorem}

\begin{proof}
    Niech będzie dana liczba $\varepsilon > 0$. Ponieważ ciąg $(f_n)$ jest zbieżny jednostajnie, więc istnieje
    takie $N$, że jeżeli $m,n \geqslant N$, $t \in E$ to zachodzi nierówność 
    \[
        |f_n(t) - f_m(t)| < \varepsilon.
    \]
    Przechodząc w ostatniej nierówności do granicy z $t$ ($t \to x$), otrzymujemy \\
    $|A_n - A_m| \leqslant \varepsilon$ dla $m, n \geqslant N$. Stąd wynika, że ciąg $(A_n)$ jest ciągiem Cauchy'ego, a zatem ma granicę. 
    Oznaczmy tę granicę przez $A$. Wówczas mamy
    \begin{equation}\label{eq:75}
        |f(t) - A| \leqslant |f(t) - f_n(t)| + |f_n(t) - A_n| + |A_n + A|.
    \end{equation}
    Wybierzmy $n$ tak duże, żeby nierówność 
    \begin{equation}\label{eq:76}
        |f(t) - f_n(t)| < \frac{\varepsilon}{3}
    \end{equation}
    była spełniona dla każdego $t \in E$ (jest to możliwe dzięki jednostajnej zbieżności ciągu $(f_n)$) i żeby
    \begin{equation}\label{eq:77}
        |A_n - A| < \frac{\varepsilon}{3}.
    \end{equation}
    Następnie dla tego $n$ weźmy takie otoczenie $V$ punktu $x$, aby dla $t \in V \cap E$, $t \neq x$ zachodziło
    \begin{equation}\label{eq:78}
        |f_n(t) - A_n| < \frac{\varepsilon}{3}.
    \end{equation} 
    Podstawijać zależności (\ref{eq:76}), (\ref{eq:77}), (\ref{eq:78}) do nierówności (\ref{eq:75}) otrzymujemy
    \[
        |f(t) - A| < \varepsilon \text{ dla } t \in V \cap E, \quad t \neq x.
    \]
    Stąd $\lim\limits_{t \to x}f(t) = \lim\limits_{n \to \infty}A_n$. 
\end{proof}

\begin{theorem}
{
    Jeżeli $(f_n)$ jest ciągiem funkcji ciągłych na zbiorze $E$ i jeżeli $f_n \to f$ jednostajnie na $E$, to funkcja $f$ jest ciągła na zbiorze $E$.
}
\end{theorem}

\begin{proof}
    Jest to bezpośrednia konsekwencja Twierdzenia \ref{theorem:149}.
\end{proof}

\begin{uwaga}
    Twierdzenie odwrotne do Twierdzenia \ref{theorem:150} nie jest prawdziwe, to znaczy ciąg funkcji 
    ciągłych może być niejednostajnie zbieżny do funkcji ciągłej (zob. Przykład \ref{eq:41} (b)). 
\end{uwaga}

\begin{wniosek}
{
    Jeżeli $(f_n)$ jest ciągiem funkcji ciągłych na zbiorze $E$ i jeżeli ciąg $(f_n)$ jest 
    zbieżny jednostajnie na zbiorze $E$, to jego suma $s(x) = \sum\limits_{n=1}^{\infty}f_n(x)$, $x \in E$ jest funkcją
    ciągłą na tym zbiorze. 
}
\end{wniosek}

\begin{ex}
    Roważmy ponownie funkcje z przykładu \ref{ex:41} (b). Mamy $\int\limits_{0}^{1}f_n(x)dx =
    n^2 \int\limits_{0}^{1}x{(1-x^2)}^n dx = n^2 \int\limits_{0}^{1}\frac{t^n}{2}dt = \frac{n^2 t^{n+1}}{2(n+1)} \Big|_0^1 =
    \frac{n^2}{2n+2}$ ($1 - x^2 = t$). Wobec tego $\lim\limits_{n \to \infty} \int\limits_{0}^{1}f_n(x)dx = 
    \lim\limits_{n \to \infty}\frac{n^2}{2n+2} = + \infty$, natomiast $\int\limits_{0}^{1}\lim\limits_{n \to \infty}f_n(x)dx = 0$.
\end{ex}

\begin{theorem}
{
    Jeżeli $(f_n)$ jest ciągiem funkcji R-całkowalnych na przedziale $[a,b]$ oraz $f_n \to f$ jednostajnie na $[a,b]$, to $f$ jest funkcją
    R-całkowalną na $[a,b]$ oraz
    \[
        \int\limits_{a}^{b}fdx = \lim_{n \to \infty}\int\limits_{a}^{b}f_n dx.
    \]
}
\end{theorem}

\begin{proof}
    Wystarczy udowodnić twierdzenie dla funkcji $f_n$ o wartościach rzeczywistych (analogicznie dla zespolonych). \\
    Niech 
    \[
        \varepsilon_n = \sup\limits_{x \in [a,b]}|f_n(x) - f(x)|.
    \]
    Wówczas $f_n(x) - \varepsilon_n \leqslant f(x) \leqslant f_n(x) + \varepsilon_n$ dla $x \in [a,b]$, zatem wobec
    Definicji \ref{def:71} oraz Twierdzenia \ref{theorem:77} mamy
    \begin{equation}\label{eq:79}
        \int\limits_{a}^{b}(f_n - \varepsilon_n)dx \leqslant \underline{\int\limits_{a}^{b}fdx} \leqslant \overline{\int\limits_{a}^{b}fdx} \leqslant \int\limits_{a}^{b}(f_n + \varepsilon_n)dx/
    \end{equation}
    Stąd $0 \leqslant \overline{\int\limits_{a}^{b}fdx} - \underline{\int\limits_{a}^{b}fdx} \leqslant 2\varepsilon_n(b-a)$.
    Ponieważ na mocy Twierdzenia \ref{theorem:145}, $\varepsilon_n \to 0$ przy $n \to \infty$, a więc całki górna i dolna funkcji $f$ są równe.
    Wobec tego funkcja $f$ jest R-całkowalna na $[a,b]$. Przekształcając (\ref{eq:79}) inaczej otzymujemy
    \[
        \Bigg|\int\limits_{a}^{b}fdx -  \int\limits_{a}^{b}f_n dx \Bigg| \leqslant \varepsilon_n(b-a).
    \]
    więc $\lim\limits_{n \to \infty}\int\limits_{a}^{b}f_n dx = \int\limits_{a}^{b}fdx$.
\end{proof}

\begin{wniosek}
{
    Jeżeli funkcje $f_n$ są R-całkowalne na przedziale $[a,b]$ i szereg $\sum\limits_{n=1}^{\infty}f_n$ jest jednostajnie zbieżny 
    na $[a,b]$, to
    \[
        \int\limits_{a}^{b}sdx = \sum\limits_{n=1}^{\infty}\int\limits_{a}^{b}f_n dx, \text{ gdzie } s(x) = \sum\limits_{n=1}^{\infty}f_n(x) \text{ dla } x \in [a,b].
    \]
}
\end{wniosek}

Inaczej mówiąc jednostajnie zbieżny szereg funkcyjny można całkować wyraz po wyrazie. 

\begin{ex}
    Niech $f_n(x) = \frac{\sin(nx)}{\sqrt{n}}$ ($x \in \mathbb{R}, n \in \mathbb{N}$). Wówczas 
    $f(x) = \lim\limits_{n \to \infty}f_n(x) = 0$, czyli $f'(x) = 0$ dla $x \in \mathbb{R}$. Z
    drugiej strony $f'_n(x) = \sqrt{n}\cos(nx)$, więc $(f'_n)$ nie jest zbieżny do $f'$. Na przykład
    $f'_n(0) = \sqrt{n} \to + \infty$ przy $n \to \infty$, podczas gdy $f'(0) = 0$. Zbieżność jednostajna ciągu $(f_n)$ nie pociąga
    za soba zbieżności punktowej ciągu $(f'n)$.
\end{ex}

\begin{theorem}
{
    Niech $(f_n)$ będzie ciągiem funkcji różniczkowalnych na $[a,b]$ takim, że ciąg $f_n(x_0)$ jest zbieżny 
    dla pewnego punktu $x_0 \in [a,b]$. Jeżeli ciąg $(f'_n)$ jest zbieżny jednostajnie na $[a,b]$, to także
    ciąg $(f_n)$ jest zbieżny jednostajnie do pewnej funkcji $f$ i zachodzi równość 
    \[
        f'(x) = \lim\limits_{n \to \infty}f'_n(x). \quad x \in [a,b].
    \]
}
\end{theorem}

\begin{proof}
    Niech będzie dana liczba $\varepsilon > 0$. Niech $N \in \mathbb{N}$ bedzie takie, aby dla $n,m \in \mathbb{N}$ zachodziło
    \begin{equation}\label{eq:80}
        |f_n(x_0) - f_m(x_0)| < \frac{\varepsilon}{2}
    \end{equation}
    oraz 
    \begin{equation}\label{eq:81}
        |f'_n(t) - f'_m(t)| < \frac{\varepsilon}{2(b-a)} \quad (a \leqslant t \leqslant b).
    \end{equation}
    Jeżeli do funkcji $f_n - f_m$ zastosujemy Twierdzenie 73 o Wartości Średniej, to dzięki (\ref{eq:81}) mamy
    \begin{equation}\label{eq:82}
        |f_n(x) - f_m(x) - f_n(t) + f_m(t)| \leqslant \frac{|x-t|\varepsilon}{2(b-a)} \leqslant \frac{\varepsilon}{2}
    \end{equation}
    dla dowolnych wartośći $x$ i $t$ z przedziału $[a,b]$ oraz $n,m \geqslant N$. Z nierówności
    \[
        |f_n(x) - f_m(x)| \leqslant |f_n(x) - f_m(x) - f_n(x_0) + f_m(x_0)| + |f_n(x_0) - f_m(x_0)|
    \]
    wynika, na mocy (\ref{eq:80}) i (\ref{eq:82}), że 
    \[
        |f_n(x) - f_m(x)| < \varepsilon \quad (a \leqslant x \leqslant b, \quad n,m \geqslant N)
    \]
    i wobec tego ciąg $(f_n)$ jest zbieżny jednostajnie na $[a,b]$. Niech 
    \[
        f(x) = \lim\limits_{n \to \infty}f_n(x) \quad (a \leqslant x \leqslant b).
    \]
    Ustalmy punkt $x$ z przedziau $[a,b]$ i określmy 
    \begin{equation}\label{eq:83}
        \phi_n(t) = \frac{f_n(t) - f_n(x)}{t-x}, \quad \phi(t) = \frac{f(t) - f(x)}{t-x} \text{ dla } t \in [a,b], \quad t \neq x.
    \end{equation}
    Wtedy
    \begin{equation}\label{eq:84}
        \lim\limits_{t \to x}\phi_n(t) = f_n'(x) \quad (n \in \mathbb{N})
    \end{equation}
    Z pierwszej nierówności (\ref{eq:82}) wynika, że 
    \begin{equation*}
        |\phi_n(t) - \phi_m(t)| \leqslant \frac{\varepsilon}{2(b-a)} \quad (n,m \geqslant N)
    \end{equation*}
    i wobec tego ciąg $(\phi_n)$ jest jednostajnie zbieżny przy $t \neq x$. Ponieważ $(f_n)$ jest zbieżny do $f$, zachodzi 
    \begin{equation}\label{eq:85}
        \lim\limits_{n \to \infty}\phi_n(t) = \phi(t)
    \end{equation}
    jednostajnie na zbiorze tych $t$, że $a \leqslant t \leqslant b$, $t \leqslant x$. Stosując do ciągu $(\phi_n)$
    Twierdzenie \ref{theorem:149} wnioskujemy, że na podstawie (\ref{eq:84}) i (\ref{eq:85}), że 
    \[
        \lim\limits_{t \to x} \lim\limits_{n \to \infty}\phi_n(t) = \lim\limits_{n \to \infty}\lim\limits_{t \to x}\phi_n(t) = \lim\limits_{n \to \infty}f'_n(x), 
    \]
    a zatem $f(x) = \lim\limits_{n \to \infty}f'_n(x)$.
\end{proof}

\begin{wniosek}
{
    Niech $(f_n)$ będzie ciągiem funkcji rózniczkowalnych na $[a,b]$ takim, że szereg 
    $\sum\limits_{n=1}^{\infty}f_n(x_0)$ jest zbieżny dla pewnego punktu $x_0 \in [a,b]$. Jeżeli
    szereg $\sum\limits_{n=1}^{\infty}f'_n$ jest zbieżny jednostajnie na $[a,b]$, to także szereg 
    $\sum\limits_{n=1}^{\infty}f_n$ jest zbieżny jednostajnie na $[a,b]$ oraz
    \[
        \Bigg( \sum\limits_{n=1}^{\infty}f_n(x)\Bigg)' = \sum\limits_{n=1}^{\infty}f'_n(x), \quad x \in [a,b]. 
    \]
}
\end{wniosek}

\end{justify}
\end{document}